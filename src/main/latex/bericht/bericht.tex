\documentclass[
		paper=a4,
	  twoside=false,
	 fontsize=10pt,
			  titlepage,
			  plainheadsepline,
			  plainfootsepline,
			  headsepline,
			  footsepline,
		index=totoc,	% fügt Inhaltsverzeichnis in Inhaltsangabe ein
	   listof=totoc,	% fügt das Sourcenverzeichnis in Inhaltsangabe ein
 bibliography=totoc,	% fügt Literaturverzeichnis in Inhaltsangabe ein
	 abstract=false,		%needed?
	titlepage=false
]{scrreprt}

\raggedright

%\parskip2ex plus1ex minus0.5ex
\setlength{\parindent}{0.05\textwidth}

\usepackage[
		 left=3cm,
		right=3cm,
		  top=2.5cm,
	   bottom=2.5cm,
		includeheadfoot
]{geometry}

\usepackage[
		pdftex,
		colorlinks=true,
		 linkcolor=black,
		 citecolor=black,
		 filecolor=black,
		  urlcolor=black
]{hyperref}
\usepackage[all]{hypcap}

\usepackage[english, ngerman]{babel}\selectlanguage{ngerman}
\usepackage[ngerman]{translator}
\usepackage[acronym, toc, numberedsection=false]{glossaries}

\usepackage[utf8]{inputenc} %Kodierung, statt latin1 auch utf8 möglich
%\usepackage{setspace} \onehalfspacing

\usepackage{lmodern}
%\usepackage{ae,aeguill}
\usepackage{ae}

\usepackage{amsfonts}
\usepackage{amsmath}
\usepackage[automark]{scrpage2}
\pagestyle{scrheadings}
%\setlength{\headheight}{1.5\baselineskip}
\clearscrheadfoot

\usepackage[round,numbers,square]{natbib}
\bibliographystyle{alphadin}

\usepackage{graphicx}
\usepackage{epsfig}
\setlength{\unitlength}{10mm} %sets unitlength for picture-environment

\usepackage{rotating}
\usepackage{longtable}
\usepackage{multirow}
\usepackage{afterpage}

% \usepackage{eurosym} % euro sign

\usepackage{color}
\definecolor{blue}{rgb}{0.1647,0,1}
\definecolor{darkred}{rgb}{0.498,0,0.33333}
\definecolor{green}{rgb}{0.247,0.498,0.3725}
\definecolor{lightgray}{rgb}{0.95,0.95,0.95}

\usepackage{listings}
\lstset{
		language=Java,									% the language for the code
	escapeinside={(*@}{@*)},							% the comments in code that are not printed
	  basicstyle=\scriptsize\ttfamily\color{black},		% the style of the code font
	keywordstyle=\bfseries\color{darkred},				% the style for keywords
	 stringstyle=\color{blue},							% the style for strings
	commentstyle=\color{green},							% the style for comments
 identifierstyle=\color{black},							% the style for identifiers
		 numbers=left,									% where to put the line numbers
	 numberstyle=\tiny,									% the style of line numbers
showstringspaces=false,									% underline spaces in strings
	  stepnumber=1,										% the step between two numbered lines
	   numbersep=10pt,									% space between numbers and code
	  captionpos=b,										% where the caption should be printed (bottom)
		   frame=single,  								% if a frame should be printed around the code
		 tabsize=4,										% how many spaces one tab contains
       aboveskip=1cm,									% the space above the displayed listing
 backgroundcolor=\color{lightgray},							% background color to fill the listings with
			breaklines
}
% \renewcommand{\lstlistlistingname}{Quellcodeverzeichnis}
%\renewcommand{\baselinestretch}{1.5}\normalsize

% Schusterjungen/Hurenkinder ausschließen
\clubpenalty = 10000
\widowpenalty = 10000
\displaywidowpenalty = 10000

\newcommand{\figref}[1]{Abbildung~\ref{fig:#1}}
\newcommand{\secref}[1]{Abschnitt~\ref{sec:#1}}
\newcommand{\tabref}[1]{Tabelle~\ref{tbl:#1}}
\newcommand{\lstref}[1]{Listing~\ref{lst:#1}}
\newcommand{\srcref}[1]{Zeile~\ref{src:#1}}
\newcommand{\chpref}[1]{Kapitel~\ref{chp:#1}}
\newcommand{\itmref}[1]{Punkt~\ref{itm:#1}}

\newcommand{\figlbl}[1]{\label{fig:#1}}
\newcommand{\seclbl}[1]{\label{sec:#1}}
\newcommand{\tablbl}[1]{\label{tbl:#1}}
\newcommand{\lstlbl}[1]{lst:#1}
\newcommand{\srclbl}[1]{\label{src:#1}}
\newcommand{\chplbl}[1]{\label{chp:#1}}
\newcommand{\itmlbl}[1]{\label{itm:#1}}

\newcommand{\icon}{icon~Systemhaus~GmbH}

\newcommand{\code}[1]{\texttt{#1}}
\newcommand{\notiz}[1]{\marginpar{\raggedright\tiny #1}}
\newcommand{\glentry}[3]{\newglossaryentry{#1}{name={#2},description={{#3}}}}

\newcommand{\md}{\emph{micro-debug}}
\newcommand{\mdg}{\md{}\emph{-gui}}
\newcommand{\mic}{\emph{Mic-1}}
\newcommand{\datei}[1]{\texttt{.#1}-Datei}
\newcommand{\reg}[1]{\texttt{#1}}
\newcommand{\package}{Package}
\newcommand{\packages}{Packages}
\newcommand{\subpackage}{Subpackage}
\newcommand{\subpackages}{Subpackages}
\newcommand{\pck}[1]{\texttt{#1}}
\newcommand{\mvn}{\emph{maven}}
\newcommand{\klasse}[1]{\texttt{#1.java}}


% \usepackage{ifthen}
% \newcommand{\vgl}[2][]{% #1 = optional, #2 = notwendig
%   \ifthenelse{\equal{#1}{}}{
%        % Autor hat kein oder leeres optionales Argument angegeben
% 		(Vgl. \cite{#2})
%   }{%
%        % Autor hat optionales Argument angegeben
% 		(Vgl. \cite[#1]{#2})
%   }
% }
% \newcommand{\vglausf}[2][]{% #1 = optional, #2 = notwendig
%   \ifthenelse{\equal{#1}{}}{
%        % Autor hat kein oder leeres optionales Argument angegeben
% 		(Vgl. dazu ausführlich \cite{#2})
%   }{%
%        % Autor hat optionales Argument angegeben
% 		(Vgl. dazu ausführlich \cite[#1]{#2})
%   }
% }

\newcommand{\code}[1]{\texttt{#1}}
\newcommand{\notiz}[1]{\marginpar{\tiny#1}}
\newcommand{\glentry}[3]{\newglossaryentry{#1}{name={#2},description={{#3}}}}

\usepackage[texcoord]{eso-pic}
%\newcommand{\startprintinglogo}{
%	\AddToShipoutPicture{
%		\parbox[t]{\paperwidth}{
%			\vspace{0.5cm}
%			\hspace{16.2cm}
%			\includegraphics[width=5.36cm,height=2.0cm]{includes/dhbw}
%		}
%	}
%}

\renewcommand{\notiz}[1]{} %uncomment for _NOT_ showing sidenotes!
\newcommand\thema{Design und Implementierung eines Debuggers für Mikro-Assembler-Programme}

\hypersetup{
	pdftitle={Studienarbeit}, pdfauthor={Christian Rösch, \icon},
	pdfsubject={\thema}, pdfkeywords={}
}

%\(i|c|o)head[plainpage]{otherpage} % i..links, o..rechts, c..zentriert
\ihead[Studienarbeit]{Studienarbeit}
\ohead[Christian Rösch]{Christian Rösch}
\ifoot[\rightmark]{\rightmark} 
\ofoot[\pagemark]{\pagemark} 

\loadglsentries{glossar}
\loadglsentries[\acronymtype]{abkuerzungen}
\makeglossaries


\begin{document}

% \nocite{*} % Alle Einträge der .bib-Datei werden auch wenn unzitiert ins Literaturverzeichnis übernommen
	
% don't write the text of these acronyms \glsunset{asf}
% \glsunset{html}

\thispagestyle{empty}

\AddToShipoutPicture*{
	\parbox[t]{\paperwidth}{
		\vspace{2.0cm}
		\hspace{12.0cm}
		\includegraphics[height=2.5cm]{includes/dhbw.jpg}
	}
}
\parbox[t]{\paperwidth}{}
\begin{center}
	\vspace{15mm}	{\LARGE\bf\thema}\\
	\vspace{20mm}	{\Large\textsc{Studienarbeit}}\\
	\vspace{15mm}	des Studiengangs Angewandte Informatik\\der Dualen Hochschule Baden-Württemberg Stuttgart\\
	\vspace{7,5mm}	von\\
	\vspace{7,5mm}	{\textbf{Christian Rösch}}\\
	\vspace{10mm}	Oktober 2011 - Mai 2012
\end{center}

\vfill

\begin{tabular*}{\textwidth}{p{7cm}l}
  \textbf{Bearbeitungszeitraum} &  Oktober 2011 - Mai 2012\\
  \textbf{Bearbeitungsdauer}    &  300~Stunden\\
  \textbf{Matrikelnummer}       &  0487930\\
  \textbf{Kurs}                	&  TIT09AIC\\
  \textbf{Ausbildungsfirma}     &  \icon{} -- Stuttgart\\
  % \textbf{Gutachter der Ausbildungsfirma} &  Dipl.-Inf. Steffen Huber\\
  \textbf{Gutachter der Studienakademie} & Prof.~Dr.~Karl~Stroetmann\\
\end{tabular*}\clearpage

\pagenumbering{roman}

\selectlanguage{ngerman}
\begin{abstract}
  In dieser Arbeit stelle ich den \md und die \mdg vor. Der \md ist ein konsolenbasierter Debugger für die \mic, die \name{Tanenbaum} in \cite{Tanenbaum1998} beschreibt. Die \mdg ist eine \gls{gui} für den \md.

Sowohl für den \md als auch die \mdg beschreibe ich, wie sie bedient werden. Ich zeige ein Tutorial, wie mit dem \md ein Fehler im \ac gefunden werden kann und weise auf die Unterschiede zwischen dem \md und der \mdg hin.

Im zweiten Teil der Arbeit konzentriere ich mich auf die Entwicklung des \md und der \mdg. Die Werkzeuge, die ich zur Entwicklung genutzt habe, stelle ich vor und zeige, wie ein Außenstehender sich an der Entwicklung des \md oder der \mdg beteiligen kann. Auch den Aufbau des Codes beschreibe ich und beschreibe, welche Funktionalität des \md wo implementiert ist.
\end{abstract}

\bigskip

\selectlanguage{english}
\begin{abstract}
  In this seminar paper I present the \md and the \mdg. The \md is a console based debugger for the \mic, described by \name{Tanenbaum} in \cite{Tanenbaum1998}. The \mdg is a \gls{gui} for the \md.

  For the \md and the \mdg I describe how to use them. Also there is a tutorial where I explain, how you can find bugs in an assembler code with the \md. Furthermore I show the differences between the \md and the \mdg.

  In the second part I concentrate on the development of the \md and the \mdg. I describe the tools I've used for development and explain how you can use them to participate in development of the \md and the \mdg. Also I describe the structure of the code and where which functionality of the \md has been implemented.
\end{abstract}

\selectlanguage{ngerman}


% \include{Erklaerung}\clearpage

%	\startprintinglogo

\tableofcontents\clearpage

\printglossary\clearpage

\printglossary[type=\acronymtype,title={Abkürzungsverzeichnis}]\clearpage

\listoffigures\clearpage

\listoftables\clearpage

\lstlistoflistings\clearpage

\pagenumbering{arabic}

\chapter*{Einleitung}
Mit dieser Arbeit stelle ich \md vor, einen Debugger für die \mic. Die \mic ist ein Prozessor und wird von \name{Tanenbaum} in \cite{Tanenbaum1998} vorgestellt -- sie ist ein Prozessor der \gls{cisc} Architektur.

Im Gegensatz zur \gls{risc} Architektur haben \gls{cisc}-Prozessoren einen komplexeren Befehlssatz. Bei der \mic wird dies durch einen \mac realisiert, der für die begrenzte Hardware einen komplexen Befehlssatz bereitstellt. Dieser \mac ermöglicht, Maschinenbefehle auf einem hohen Abstraktionsniveau zu definieren und ohne Veränderung der Hardware anzupassen.

\section*{Aufgabenstellung}
Im Skript der Vorlesung Rechnertechnik \cite[Kapitel 8]{Stroetmann2009} wird von \name{Stroetmann} ein Mikro-Assembler für die \mic vorgestellt. Ziel dieser Arbeit ist der Entwurf und die Implementierung eines Debuggers für diesen Mikro-Assembler. Der Debugger soll die Entwicklung von \ma und \ac für die \mic erleichtern.

Um entsprechenden \ma und \ac debuggen zu können, muss der Debugger die \mic simulieren können. In seiner Studienarbeit \cite{Kutzer2004} beschreibt \name{Kutzer} einen Simulator für die \mic. Der hier entwickelte Debugger soll diesen Simulator ersetzen und um Debug-Funktionen erweitern.

In dieser Arbeit sollen dabei hauptsächlich zwei Teilaufgaben bearbeitet werden.

\subsection*{Debugger für die Konsole}
Zunächst soll eine Version des Debuggers entwickelt werden, die über die Konsole bedient wird. Das Ziel ist die Implementierung des Debuggers unabhängig von der Darstellung der Ein- und Ausgabe.

Der Debugger soll in der Lage sein zwei Binärdateien einzulesen: eine Datei, die den \ma und eine, die den \ac enthält. Der \ac sollte grundsätzlich nur vom \mac abhängig und durch den Benutzer frei definierbar sein; als Referenz für diese Arbeit kann aber der \gls{ijvm}-Assembler dienen.

Der Benutzer muss Breakpoints (Haltepunkte) definieren können, die die Simulation des \mac unterbrechen. Der Debugger ermöglicht dem Benutzer dabei, Breakpoints für bestimmte \mais aber auch für bestimmte \ais zu definieren.

Neben den Werten der Register kann der Benutzer sich auch den Inhalt des Stacks, die gesetzten Breakpoints, den \ma und den \ac anzeigen lassen. Dem Benutzer soll es möglich sein, lokale Variablen, Register, den ausgeführten \ma sowie \ac zu beobachten.

\subsection*{GUI für den Debugger}
Im zweiten Teil der Aufgabe wird für den Debugger eine \gls{gui} entwickelt werden. Die \gls{gui} soll sowohl den \ma als auch den \ac anzeigen; Auch die Register mit ihren Werten und der Stack sollen sichtbar sein.

Die \gls{gui} ermöglicht dem Benutzer, Breakpoints zu setzen und zeigt an, welche \ma und \ai gerade abgearbeitet wird.

Da in dieser Arbeit ein allumfassender Debugger weder geschaffen werden kann noch soll, ist besonders darauf zu achten, die Wartung und Erweiterung des Debuggers außenstehenden Personen zu ermöglichen. Besonders der Code des Debuggers soll daher gut dokumentiert werden; es sollen sich andere Studenten in den Code einarbeiten und neue Funktionalitäten implementieren oder bestehende Funktionalitäten anpassen können.

\section*{Aufbau der Arbeit}
Die Arbeit ist in zwei Teile geteilt: Im ersten Teil erläutere ich die Bedienung und im zweiten Teil die Entwicklung des \md. Ich beschreibe also eher den derzeitigen Zustand des \md, als den Weg der Entstehung des \md. Denn diese Arbeit soll vor allem dazu dienen, dass Außenstehende sich in den \md einarbeiten können und den jetzigen Zustand nachvollziehen können.

Ich werde im Folgenden \md für die Konsolenvariante des Debuggers verwenden und \mdg für die \gls{gui}. Diese beiden Bezeichnungen sind gleichzeitig die Namen der Code-Projekte -- so möchte ich eine saubere Trennung zwischen den Projekten erreichen. Die Bezeichnung \emph{Debugger} werde ich verwenden, um allgemein über den \md und die \mdg zu sprechen; oder für Erklärungen, die beide Projekte betreffen.

\subsection*{Bedienung}
Im \chpref{allgemein} beantworte ich einige allgemeine Fragen über den Debugger: Welche Systemvoraussetzungen gibt es? Wie konfiguriert man den Debugger? Wie kann man den Debugger in andere Sprachen übersetzen? Produziert der Debugger Log-Ausgaben? Wenn ja, kann man diese konfigurieren?

Die Bedienung des \md beschreibe ich im \chpref{bed-konsole}; dort erkläre ich auch die Parameter des \md. Außerdem erläutere ich die verschiedenen Befehle und zeige ein Beispiel, wie mit dem \md ein Fehler im \ac gefunden werden kann.

Im \chpref{bed-gui} beschreibe ich die Bedienung der \mdg; dort nenne ich die Unterschiede zum \md und erkläre die verschiedenen Oberflächenelemente.% und zeige ein Beispiel, wie mit der \mdg ein Fehler gefunden werden kann.

\subsection*{Entwicklung}
In \chpref{werkzeuge} beschreibe ich einige Werkzeuge, die ich zur Entwicklung des Debuggers genutzt habe und deren Verständnis nötig ist, um selbst Änderungen am Projekt vorzunehmen. In \chpref{entw-allg} gehe ich kurz auf die Notwendigkeit der automatisierten Tests ein.

In den nachfolgenden Kapiteln beschreibe ich, welche Funktionalität des Debuggers sich in welchen Klassen befinedt. \chpref{prozessor} nutze ich, um zu beschreiben, wie die \mic simuliert wird und wie ich die verschiedenen Hardwarekomponten implementiert habe. Im \chpref{konsole} erkläre ich den Code des \md und in \chpref{gui} den Code der \mdg.\clearpage

\part{Bedienung}
\chapter{Allgemein}
\chplbl{allgemein}
Der Debugger -- sowohl der \md als auch die \mdg{} -- benötigt keine Installation; er kann als \texttt{.zip}-Archiv heruntergeladen werden und ist direkt nach dem Entpacken nutzbar. Die Datei heißt üblicherweise \texttt{micro-debug-version.zip}\footnote{Die Datei der \mdg heißt üblicherweise \texttt{micro-debug-gui-version.zip}.} und enthält mehrere Dateien:

\begin{description}
\item[micro-debug-version.jar] enthält den Code des \md und kann mit dem Befehl \texttt{java~-jar} ausgeführt werden. Diese Datei ist auch in der \mdg enthalten, dort gibt es aber zusätzlich noch die Datei \texttt{micro-debug-gui-version.jar}, die den Code der \mdg enthält.
\item[micro-debug.sh und micro-debug.bat] sind die Startskripte des \md für Windows und Linux\footnote{Auch hier ist die Namensgebung für die Startskripte der \mdg entsprechend: \texttt{micro-debug-gui.sh} und \texttt{micro-debug-gui.bat}}. Der Debugger kann durch Ausführen dieser Skripte gestartet werden -- der Benutzer muss sich dadurch nicht um die Konfiguration des Klassenpfades kümmern. Ich habe die Startskripte so geschrieben, dass sie von beliebigen Orten ausgeführt werden können: Der Benutzer kann den Debugger von vielen Arbeitsverzeichnissen aus aufrufen und hat die Konfigurationsdateien nur an einer Stelle zu pflegen.
\item[config/micro-debug.properties] ist die Konfigurationsdatei für den \md und die \mdg, hier kann beispielsweise die Größe des Hauptspeichers oder Tastenkombinationen konfiguriert werden -- siehe \secref{konfiguration}.
\item[config/logging.properties] ist die Konfigurationsdatei für das Logging des Debuggers -- hier kann der Benutzer definieren, ob und welche Ausgaben auf der Konsole oder in einer Datei erscheinen sollen -- siehe \secref{logs}.
\item[config/lang/] enthält die Textressourcen des Debuggers -- siehe \secref{lokalisierung}.
\item[lib/] enthält die verwendeten Bibliotheken des Debuggers; zur Zeit nur in der \mdg verwendet.
\end{description}

\section{Systemvoraussetzungen}
\seclbl{systemvoraussetzungen}
Der Debugger ist in \gls{java} geschrieben und daher prinzipiell an keine spezielle Plattform gebunden. Voraussetzung für die Nutzung des \md ist lediglich \gls{java}~5\notiz{Referenz?}.

In der Praxis differenzieren sich die verschiedenen Plattformen, auf denen \gls{java} verfügbar ist, in einigen Feinheiten. Daher ist es prinzipiell möglich, dass sich der Debugger auf gewissen Plattformen unerwünscht verhält. Wie anhand der Startskripte zu erkennen ist, habe ich den Debugger im Blick auf \emph{Linux} und \emph{Windows} entwickelt und auf diesen Plattformen\footnote{Ausführlich habe ich mit \emph{Windows~7} (in der 64-Bit-Variante) und \emph{Ubuntu~12.04} (auch 64-Bit) getestet.} getestet. Um den Debugger auf einer weiteren Plattform zu nutzen, muss zumindest das Startskript eigenständig geschrieben werden.

Der \md nutzt bislang einen Thread, die \mdg hingegen mindestens zwei (wie ich in \secref{verarbeitung-benutzeraktionen-threads} näher beschreibe), die allerdings selten parallel arbeiten. Für die Ausführungsgeschwindigkeit des Debuggers ist die Anzahl der verfügbaren Prozessorkerne daher irrelevant.

\section{Konfiguration}
\seclbl{konfiguration}
Die Konfiguration des Debuggers findet in \datei{properties}en statt. Die Konfigurationsdatei \texttt{config/micro-debug.properties} enthält die Konfigurationen sowohl für den \md als auch die \mdg. Wie in \datei{properties}en üblich, werden dort Schlüsselwert-Paare hinterlegt.

\begin{lstlisting}[language=sh,caption={Eintrag in \texttt{conf/micro-debug.properties}},label=\lstlbl{md-props-entry}]
# default value for the register CPP
mic1.register.cpp.defval = 0x4000
\end{lstlisting}

\lstref{md-props-entry} zeigt einen Eintrag aus der Konfigurationsdatei: Den Startwert für das Register \reg{CPP}. Diese Datei kann für jeden Schlüssel genau einen Wert enthalten. Ist für einen Schlüssel kein Wert konfiguriert oder ist ein ungültiger Wert konfiguriert, nutzt der Debugger den internen Standardwert für den jeweiligen Schlüssel.\notiz{Prüfen und ggf. einfügen, dass fallback anständig geloggt wird und das dann hier auch erwähnen.}

Dadurch ist es möglich ein Update des Debuggers durchzuführen, und weiterhin die alte Konfiguration zu nutzen. Ersetzt der Benutzer die \datei{jar} durch eine neuere Version, die zusätzliche Konfigurationsoptionen benötigt, nutzt der Debugger die Standardwerte dieser Konfigurationsoptionen.

Der Benutzer kann seine eigene Konfiguration dadurch von Version zu Version des Debuggers behalten; er muss geänderte Konfigurationsoptionen nicht mühsam nachpflegen.\notiz{Überlegen, ob nicht irgendwo ein Update-Abschnitt sinnvoll ist}

\subsection{Zahlenformat}
\seclbl{zahlenformat}
\notiz{Wahrscheinlich ist es sinnvoll in den properties Dateien einzutragen, von welchem Typ ein Schlüssel sein soll und dies dann hier auch zu notieren.}
In \lstref{md-props-entry} ist zu sehen, wie dem Register \reg{CPP} der Standardwert \texttt{0x4000} zugewiesen wird -- hier hexadezimal notiert. Andere Konfigurationseinträge sind dezimal eingetragen; welches Zahlenformat ist nun wo anzuwenden?

Für den Debugger ist es irrelevant, welches Zahlenformat bei welcher Konfigurationsoption verwendet wird. Er liest die gegebene Zahl und wandelt sie in einen \emph{Integer} um; dadurch kann der Benutzer für jeden Eintrag das passende Zahlenformat wählen. Nicht nur bei Konfigurationsoptionen, sondern bei allen eingegebenen Zahlen des Benutzers ist das Format variabel.

Welche Zahlenformate gibt es außer dezimal und hexadezimal? Der Debugger unterstützt alle Zahlensysteme mit der Grundzahl von $b=2$ bis $b=36$. Generell werden die Zahlen in der Art \texttt{ZAHL\_BASIS} angegeben -- \texttt{ZAHL} wird in dem jeweiligen Zahlensystem und \texttt{BASIS} im Dezimalsystem angegeben. In \tabref{verschiedene-zahlenformate} sind verschiedene Eingabemöglichkeiten für die Zahl $10$ gegeben.

\begin{table}[h]
  \centering
  \begin{tabular}[h]{|rl|}
    \hline
    \textbf{Zahlensystem} & \textbf{Darstellung}\\
    \hline
    Dualsystem        & \texttt{1010\_2}\\
    Ternärsystem      & \texttt{101\_3}\\
    Dezimalsystem     & \texttt{10\_10}\\
    Hexadezimalsystem & \texttt{A\_16}\\
    \hline
  \end{tabular}
  \caption{Auswahl möglicher Eingabearten der Zahl $10$ im Debugger}
  \tablbl{verschiedene-zahlenformate}
\end{table}


Für die meiner Meinung am häufigsten verwendeten Zahlensysteme gibt es die in \tabref{vereinfachungen-zahlenformate} dargestellten Vereinfachungen. Der Benutzer \emph{muss} diese Vereinfachungen aber nicht nutzen, er kann sowohl \texttt{A\_16} als auch \texttt{0xA} eingeben -- oder aber \texttt{10}.

\begin{table}[h]
  \centering
  \begin{tabular}[h]{|rll|}
    \hline
    \textbf{Zahlensystem} & \textbf{ausführlich} & \textbf{vereinfacht}\\
    \hline
    Dualsystem        & \texttt{1010\_2} & \texttt{0b1010} \\
    Oktalsystem       & \texttt{12\_8}   & \texttt{0o12}   \\
    Dezimalsystem     & \texttt{10\_10}  & \texttt{10}     \\
    Hexadezimalsystem & \texttt{A\_16}   & \texttt{0xA}    \\
    \hline
  \end{tabular}
  \caption{Vereinfachungen typischer Zahlenformate im Debugger am Beispiel der Zahl $10$}
  \tablbl{vereinfachungen-zahlenformate}
\end{table}

\subsection{Logs}
\seclbl{logs}
Sucht der Benutzer die Ursache für einen Fehler, den er oder der Debugger begangen hat, sind Log-Ausgaben ein guter Ansatzpunkt. Denn dort sind die vermeintlich wichtigen Vorgänge protokolliert und im Hinblick auf solche Analysen geschrieben.

Der Debugger wird im Startskript so gestartet, dass die Log-Konfiguration aus der Datei \texttt{conf/logging.properties} genutzt wird. Da die Syntax und Semantik der Konfigurationseinträge von \name{Oracle} \cite{Oracle2004} und \name{Vogel} \cite{Vogel2012} detailliert beschrieben ist, lasse ich sie hier aus.

Die Startskripte sind so konfiguriert, dass der Debugger alle Log-Ausgaben mit dem Log-Level \texttt{INFO} oder höher in Dateien schreibt. Diese Dateien liegen im \emph{home}-Verzeichnis des Benutzers und sind von der Form \texttt{micro-debugX.log} -- wobei \texttt{X} ein Laufindex ist. Es existieren neben dem genannten noch weitere Log-Level; je nach Situation kann der Debugger beispielsweise für ausführlichere Log-Ausgaben konfiguriert werden. Die verschiedenen Log-Level sind mit absteigender Wichtigkeit: \texttt{SEVERE}, \texttt{WARNING}, \texttt{INFO}, \texttt{CONFIG}, \texttt{FINE}, \texttt{FINER} und \texttt{FINEST}.

\subsection{ijvm.conf}
\seclbl{ijvm-conf}
Ich habe den Debugger so konstruiert, dass er prinzipiell für jeden beliebigen \ac genutzt werden kann. Allerdings habe ich den Debugger nur mit dem \gls{ijvm}-Assembler getestet, da der Debugger in der Praxis wohl vorwiegend mit dem \gls{ijvm}-Assembler genutzt wird.

Um den \ac disassemblieren zu können, gibt es eine Konfigurationsdatei -- die \texttt{ijvm.conf}. Damit der Debugger einen gewissen Standardsatz an \gls{ijvm}-Befehlen versteht, wird diese Datei in der \datei{jar} des Debuggers mit ausgeliefert.

Möchte der Benutzer die Datei \texttt{ijvm.conf} anpassen, genügt es, die Datei im Verzeichnis \texttt{conf/} abzulegen. Durch das Startskript wird sie dann auf den Klassenpfad geladen; vor der internen \texttt{ijvm.conf}-Datei in der \datei{jar} und ersetzt diese somit.

Die Datei \texttt{ijvm.conf} ist zeilenweise zu lesen, wobei eine Zeile nach folgendem Format aufgebaut ist:\notiz{Beispiel!?}

\setlength{\grammarparsep}{3pt plus 1pt minus 1pt}
\setlength{\grammarindent}{10em} % increase separation between LHS/RHS 
\begin{grammar}

<line>     ::= <comment>
  \alt         <address> <white>+ <identifier> <white>* <argumentlist> <comment>?

<comment>  ::= '//' <text>

<address> ::= '0x' <hexchar> <hexchar>

<hexchar> ::= '0' | '1' | '2' | '3' | '4' | '5' | '6' | '7' | '8' | '9' | 'A' | 'B' | 'C' | 'D' | 'E' | 'F'

<identifier> ::= <word>

<argumentlist> ::= ( <white> <argument> <white>* )*

<argument> ::= 'byte' | 'const' | 'index' | 'label' | 'offset' | 'varnum'

<white>    ::= ' ' | '$\backslash$t'
\end{grammar}

\section{Lokalisierung}
\seclbl{lokalisierung}
Der Debugger wurde in Englisch geschrieben -- weitere Sprachen können über das Verzeichnis \texttt{conf/lang/} hinzugefügt werden.

In diesem Verzeichnis liegen verschiedene \datei{xml}en, die dazu dienen, dem Benutzer Ausgaben in seiner Sprache anzuzeigen. Die Sprache, die der Benutzer angezeigt bekommt, hängt vom \emph{Locale} (siehe dazu ausführlich \cite{Oracle2010Loc}) der \gls{jvm} ab.

Der Debugger sucht für diese Sprache folgende vier Dateien, die Schlüsselwert-Paare für die Textkonstanten enthalten:
\begin{description}
\item[text_lang_CT_var1.xml] wobei \textbf{lang} die Sprache, \textbf{CT} die Länderkennung und \textbf{var1} die Variante ist, die vom \emph{Locale} gegeben sind. Diese Datei ist die spezifischste und wird sofern sie existiert zuerst geladen. Schlüssel, die hier nicht gefunden werden, werden in den folgenden Dateien gesucht.
\item[text_lang_CT.xml] in der Praxis existieren selten Informationen über die Variante des \emph{Locale}, daher kann \texttt{_var1} weggelassen werden.
\item[text_lang.xml] häufig existieren zwar verschiedene Länderkennungen für die gleiche Sprache, aber meist unterscheiden sich die Übersetzungsdateien für die Länderkennungen nicht. Daher werden in der Praxis viele Dateien in dieser Form angelegt werden.
\item[text.xml] die Basis-Datei für die Lokalisierung. Wenn ein Schlüssel in den anderen Dateien nicht gefunden wurde oder die anderen Dateien nicht existieren, dann wird er in dieser Datei gesucht.
\end{description}

Wie ich gerade angedeutet habe, ist der Aufbau dieser vier Dateien hierarschich: ein Schlüssel, der in der spezifischsten Datei gefunden wurde, wird aus den anderen Dateien nicht mehr ausgewertet.

\begin{lstlisting}[language=XML,caption={Beispiel für Datei \texttt{text_de_DE.xml}},label=\lstlbl{locale-file-de-de}]
<?xml version="1.0" encoding="UTF-8"?>
<!DOCTYPE properties SYSTEM "http://java.sun.com/dtd/properties.dtd">
<properties>
	<entry key="border">+*+*+*+*+*</entry>
	<entry key="border2">---</entry>
</properties>
\end{lstlisting}

Anhand eines Beispiels möchte ich dieses Verhalten erläutern: Gegeben seien die drei Dateien, die in \lstref{locale-file-de-de}, \lstref{locale-file-de} und \lstref{locale-file} aufgeführt sind. Die Datei der Form \texttt{text_lang_CT_var1.xml} ist in diesem Beispiel nicht vorhanden.

\begin{lstlisting}[language=XML,caption={Beispiel für Datei \texttt{text_de.xml}},label=\lstlbl{locale-file-de}]
<?xml version="1.0" encoding="UTF-8"?>
<!DOCTYPE properties SYSTEM "http://java.sun.com/dtd/properties.dtd">
<properties>
	<entry key="file-not-found">Datei nicht gefunden {0}</entry>
	<entry key="version">Micro-Debug - Version {0}</entry>
</properties>
\end{lstlisting}

\lstref{locale-result} zeigt das Ergebnis nach der Auswertung dieser drei Dateien am Beispiel des deutschen \emph{Locale}. Es ist zu sehen, wie die Einträge sich überschreiben -- ein Benutzer mit englischem Locale würde in diesem Beispiel den Eintrag für \texttt{border2} nicht erhalten und hätte dafür keinen definierten Wert.

\begin{lstlisting}[language=XML,caption={Beispiel für Datei \texttt{text.xml}},label=\lstlbl{locale-file}]
<?xml version="1.0" encoding="UTF-8"?>
<!DOCTYPE properties SYSTEM "http://java.sun.com/dtd/properties.dtd">
<properties>
	<entry key="version">Micro-Debug: Version {0}</entry>
	<entry key="file-not-found">file not found {0}</entry>
	<entry key="border">----------------------------------------</entry>
</properties>
\end{lstlisting}

Die Reihenfolge der Schlüsselwert-Paare ist willkürlich und hat für den Programmverlauf keine Auswirkung.

\begin{lstlisting}[language=XML,caption={Beispiel für Ergebnis nach Verarbeitung der Lokalisierungsdateien},label=\lstlbl{locale-result}]
<?xml version="1.0" encoding="UTF-8"?>
<!DOCTYPE properties SYSTEM "http://java.sun.com/dtd/properties.dtd">
<properties>
	<entry key="version">Micro-Debug - Version {0}</entry>
	<entry key="file-not-found">Datei nicht gefunden {0}</entry>
	<entry key="border">+*+*+*+*+*</entry>
	<entry key="border2">---</entry>
</properties>
\end{lstlisting}

Dieses System ermöglicht das einfache Hinzufügen neuer Sprachen. Dieser Vorgang ist zwar bisher nicht für den Benutzer vorgesehen, aber prinzipiell kann er in diesen Dateien Anpassungen vornehmen und sich dadurch die Ausgaben des Debuggers an seine Bedürfnisse anpassen. Auch ist denkbar, dass Übersetzungen des Debuggers von Dritten angeboten wird -- der Benutzer kann dann die Sprachdateien für seine Sprache herunterladen und in dem Verzeichnis \texttt{conf/lang/} ablegen.

Im Gegensatz zur Konfigurationsdatei gibt es bei den Lokalisierungsdateien keinen individuellen Fallback für jeden Schlüssel, sodass fehlende Werte mit einem Fehlertext belegt werden. Die Lokalisierungsdateien sind daher auch von einem Update des Debuggers betroffen -- mögliche Änderungen des Benutzers gehen verloren oder müssen in die neuen Dateien übernommen werden.

Einige Einträge in den Lokalisierungsdateien können Platzhalter enthalten. Platzhalter gelten nur für ein bestimmtes Schlüsselwert-Paar und sind von $0$ aufsteigend nummeriert in der Form \texttt{{0}} zu verwenden. Der Inhalt mit dem der jeweilige Platzhalter gefüllt wird kann in den Kommentaren in der Datei nachgelesen werden.

Zusätzlich zu den \texttt{text...xml}-Dateien gibt es bei der \mdg \texttt{text-gui...xml}-Dateien. Die Lokalisierungsdateien des \md und der \mdg sind also voneinander unabhängig.

Die Log-Ausgaben aus \secref{logs} sind immer auf Englisch und können über die \datei{xml}en nicht verändert werden.
\chapter{Interaktion per Konsole}
\chplbl{bed-konsole}
Wie in \chpref{allgemein} beschrieben, muss der \md{} nicht installiert werden: Vom Herunterladen bis zum Starten des \md{} genügen folgende Schritte.

\begin{enumerate}
\item Die Datei \texttt{micro-debug-version.zip} von der Projektseite\notiz{Verweis auf Seite} herunterladen
\item Die \datei{zip} in ein beliebiges Verzeichnis entpacken (bspw. \texttt{/opt/micro-debug/})
\item Das Verzeichnis des \md{} dem \texttt{PATH} hinzufügen
\item Den \md{} starten -- mit \texttt{\$ micro-debug.sh -{}-help}
\end{enumerate}

Der \md{} kann durch einige Parameter gesteuert werden und wird nach dem Start durch Befehle gesteuert. Sowohl die Parameter als auch die Befehle werden nun erklärt und anschließend wird ein beispielhafter Ablauf mit dem \md{} beschrieben.

\section{Parameter}
Der Standardaufruf für den \md{} ist in \lstref{aufruf-konsolenversion} zu sehen. Es gibt zwei verpflichtende Parameter: Die Pfade zur Mikro-Assembler- und zur Assembler-Bytecode-Datei\notiz{Irgendwo den Aufbau der beiden Dateien erwähnen}. Die beiden Pfade können sowohl relativ als auch absolut angegeben werden, wichtig ist allerdings, dass zuerst der Pfad zur Mikro-Assembler- und dann der Pfad zur assembler-Datei gegeben wird. Werden die beiden Pfade vertauscht, so startet der \md{} nicht und bricht mit einer Fehlermeldung ab.

\begin{lstlisting}[language=sh,caption={Aufruf des \md{} -- Konsolenversion},label=\lstlbl{aufruf-konsolenversion}]
  micro-debug.sh [PARAMETER]... MIC1 IJVM
\end{lstlisting}

Neben den beiden Dateipfaden gibt es im folgende optionale Parameter. Jeder Parameter kann sowohl in der langen (mit doppeltem Minus) als auch in der kurzen (einfaches Minus gefolgt von einem Zeichen) angegeben werden.

\begin{description}
\item[-h, -{}-help]
  ist dieser Parameter gegeben, so wird die Hilfe angezeigt, die neben den verschiedenen Aufrufmöglichkeiten die möglichen Parameter erklärt. Zusätzlich werden noch einige andere Informationen, wie Kontaktmöglichkeiten, angezeigt.

\item[-o, -{}-output-file FILE]
  ermöglicht die Umlenkung der Ausgabe der \mic{} (nicht des \md{}) in eine Datei. Normalerweise wird die Ausgabe der \mic{} auf der Konsole ausgegeben.

  Mit dem Argument \texttt{FILE} wird hier der Pfad zur Datei gegeben, in die die Ausgabe geschrieben werden soll. Wenn die Datei bereits existiert, wird die Ausgabe an das Ende der Datei angehängt.

  Unter Linux kann man dann in einer zweiten Konsole diese Datei beispielsweise mit \texttt{tail -f} anzeigen und die Ausgabe der \mic{} komfortabel von der Ausgabe des \md{} trennen. In diesem Szenario ist auch der Parameter \texttt{-{}-unbuffered-output} sinnvoll.
\item[-u, -{}-unbuffered-output]
  verhindert die Pufferung der Ausgabe der \mic{}. Normalerweise gibt der \md{} die Ausgabe der \mic{} zeilenweise aus, also erst bei der Ausgabe eines Zeilenumbruchs. Verwendet man den Parameter \texttt{-{}-output-file} oder möchte man aus sonstigen Gründen jedes ausgegebene Zeichen der \mic{} direkt auf der Konsole sehen, so ist dieser Parameter die Lösung.

  \emph{Vorsicht!} Wird dieser Parameter ohne \texttt{-{}-output-file} genutzt, so kann es sehr schwer werden, die ausgegebenen Zeichen der \mic{} ausfindig zu machen, da sie womöglich in der Menge der Ausgaben von \md{} untergehen.

\item[-v, -{}-version]
  gibt die Version des \md{} aus.
\end{description}

Wenn einer der Parameter \texttt{-{}-help} oder \texttt{-{}-version} angegeben wurden, so startet der Debugger nicht. Dies kann genutzt werden, um ohne vorhandene Bytecode-Dateien Informationen über den \md{} anzeigen zu können. \lstref{aufrufe-ohne-start} zeigt, dass der \md{} daher auch ohne Bytecode-Dateien aufgerufen werden kann.

\begin{lstlisting}[language=sh,caption={Aufruf des \md{} ohne Start -- Konsolenversion},label=\lstlbl{aufrufe-ohne-start}]
  micro-debug.sh --help
  micro-debug.sh --version
\end{lstlisting}

Bei der Abarbeitung der gegebenen Parameter hängt die Reihenfolge der Parameter nicht von der Reihenfolge ab, in der sie dem \md{} als Parameter übergeben wurden. Wichtig ist nur, dass die beiden Bytecode-Dateien die letzten beiden Parameter und in der richtigen Reihenfolge aufgeführt sind.

\section{Befehle}
Ist der \md{} gestartet lässt er sich durch verschiedene Befehle steuern, die am Ende dieses Abschnitt ausführlich beschrieben werden.

Bei der Bedienung des \md{} gibt es die Möglichkeit, dass zwischen den verschiedenen Befehlen auch Eingaben für die \mic{} zu liefern sind. Damit der Benutzer sieht, ob die Eingabe vom \md{} oder der \mic{} erwartet wird, schreibt der \md{} '\texttt{micro-debug> }' und die \mic{} '\texttt{mic1> }' bevor eine Eingabe erwartet wird. Wenn die \mic{} eine Eingabe von Dir erwartet, kannst du mehrere Zeichen auf einmal eingeben -- an die \mic{} werden die von dir eingegebenen Zeichen plus ein Zeilenumbruch gesendet. Sollte dein Assembler-Code also in einer Schleife einzelne Zeichen einlesen, solltest du die gesamte Zeile direkt eingeben.

Die verschiedenen Befehle können ein oder mehrere Argumente benötigen. Die Argumente können einem der folgenden Datentypen gehören:
\begin{description}
\item[Register] ist der Name eines Registers, also \reg{CPP}, \reg{H}, \reg{LV}, \reg{MAR}, \reg{MBR}, \reg{MBRU}, \reg{MDR}, \reg{OPC}, \reg{PC}, \reg{SP} oder \reg{TOS}.
\item[Zahl] ist eine Zahl im Wertebereich eines Integers\notiz{Verweis?}. Die Formate für die Eingabe der Zahl sind in \secref{zahlenformat} beschrieben.
\end{description}

Je nach Befehl kann der zulässige Wertebereich jedoch noch weiter eingeschränkt sein -- eine solche Einschränkung ergibt sich aus der Beschreibung des jeweiligen Befehls.

\begin{description}
\item[break] \emph{Register [Zahl]}

  setzt einen Breakpoint für das gegebene \emph{Register}: Sobald das Register den Wert \emph{Zahl} erhält, hält der \md{}. Der \md{} hält erst, nachdem das Register den Wert erhalten hat -- da dies unter Umständen zu spät ist, gibt es die Möglichkeit, das Argument \emph{Zahl} wegzulassen. Wird nur ein Register angegeben, hält der \md{} bevor das Register einen neuen Wert zugewiesen bekommt.

\item[exit] \hspace*{\fill}\\

  beendet den \md{} und gibt eventuell belegte Ressourcen wieder frei.

\item[help] \hspace*{\fill}\\

  zeigt die verfügbaren Befehle mit einer kurzen Beschreibung an. Auch hier sind noch einige weitere Informationen über das Projekt enthalten.

\item[ls-break] \hspace*{\fill}\\

  zeigt alle Breakpoints mit deren jeweiligen Bedingung an. Jeder Breakpoint erhält eine Identifikationsnummer, die bei anderen Operationen, wie dem Entfernen, angegeben werden müssen.

\item[ls-macro-code] \emph{[Zahl1 [Zahl2]]}

  zeigt den disassemblierten Assembler-Code an. Dabei gibt es drei mögliche Konstellationen der Parameter:
  \begin{itemize}
  \item Wird kein Parameter gegeben, wird der vollständige Assembler-Code angezeigt.
  \item Wird ein Parameter \emph{Zahl1} gegeben, wird die angegebene Anzahl an Zeilen vor und nach der nächsten auszuführenden Codezeile angezeigt.
  \item werden beide Parameter gegeben, wird der Assembler-Code von Zeile \emph{Zahl1} bis zur Zeile \emph{Zahl2} (inklusive) angezeigt.
  \end{itemize}

  Da der \md{} nur den Bytecode kennt, gibt es eigentlich keine Zeilen. Der \md{} schreibt pro Zeile einen Befehl (inklusive Argumente), somit existieren Zeilennummern, die für den \md{} als Referenz dienen.

\item[ls-micro-code] \emph{[Zahl1 [Zahl2]]}

  zeigt den disassemblierten Mikro-Assembler-Code an und arbeitet wie \texttt{ls-macro-code}. Es gibt drei mögliche Konstellationen der Parameter:
  \begin{itemize}
  \item Wird kein Parameter gegeben, wird der vollständige Mikro-Assembler-Code angezeigt.
  \item Wird ein Parameter \emph{Zahl1} gegeben, wird die angegebene Anzahl an Zeilen vor und nach der nächsten auszuführenden Codezeile angezeigt.
  \item werden beide Parameter gegeben, wird der Mikro-Assembler-Code von Zeile \emph{Zahl1} bis zur Zeile \emph{Zahl2} (inklusive) angezeigt.
  \end{itemize}

  Da der \md{} nur den Bytecode kennt, gibt es eigentlich keine Zeilen. Der \md{} schreibt pro Zeile eine Mikroinstruktion (die $36~Bit$, die die Instruktion spezifizieren), somit existieren Zeilennummern, die für den \md{} als Referenz dienen.

\item[ls-mem] \emph{Zahl1 Zahl2}

  zeigt den Inhalt des Hauptspeichers zwischen den Adressen (inklusive) \emph{Zahl1} und \emph{Zahl2} an. Die Adressen des Speichers sind Wortadressen -- jedes Wort enthält $32~Bit$.

\item[ls-reg] \emph{[Register]}

  zeigt den Wert von dem gegebenen \emph{Register} an; wird das optionale Argument weggelassen, werden alle Register und deren Werte angezeigt.

\item[ls-stack] \hspace*{\fill}\\

  zeigt den aktuellen Stack an.

  \emph{Hinweis:} Dieser Befehl wird durch die Konfigurationsoption \texttt{stack.elements.to.hide} beeinflusst. Normalerweise wird der Stack von dem initialen Stackpointer bis zum aktuellen Stackpointer ausgegeben. Da dies unter Umständen mehr Elemente liefert, als der Stack tatsächlich enthält, gibt es die Möglichkeit über die Konfiguration die ersten Elemente nicht auszugeben.

Möchte man den realen (im Speicher vorhandenen) Stack sehen, sollte man sicherstellen, dass \texttt{stack.elements.to.hide = 0} konfiguriert ist.

\item[macro-break] \emph{Zahl}

  fügt einen Breakpoint hinzu, der den \md{} anhält, sobald der Bytecode des Assemblers an der Adresse \emph{Zahl} ausgeführt werden soll. \emph{Adresse} muss Element aus der Menge der von \texttt{ls-macro-code} angezeigten Zeilennummern sein.

\item[micro-break] \emph{Zahl}

  fügt einen Breakpoint hinzu, der den \md{} anhält, sobald der Bytecode des Mikro-Assemblers an der Adresse \emph{Zahl} ausgeführt werden soll. \emph{Adresse} muss Element aus der Menge der von \texttt{ls-micro-code} angezeigten Zeilennummern sein.

\item[micro-step] \emph{[Zahl]}

  führt die nächsten \emph{Zahl} Mikro-Instruktionen aus; wird kein Argument gegeben, so wird eine Instruktion ausgeführt.

\item[reset] \hspace*{\fill}\\

  die \mic{} wird in den Anfangszustand zurückgesetzt: Der Hauptspeicher und die Register werden auf die initialen Werte zurückgesetzt. Auch die Ein- und Ausgabe der \mic{} wird geleert. Die Informationen des \md{}, vor allem die Breakpoints, bleiben allerdings erhalten und müssen vom Benutzer nicht erneut gesetzt werden.

\item[rm-break] \emph{Zahl}

  entfernt den Breakpoint mit der Nummer \emph{Zahl}. Die Nummer des Breakpoints ist die Identifikationsnummer, die mit \texttt{ls-break} angezeigt wird.

\item[run] \hspace*{\fill}\\
  
  führt alle Instruktionen bis zum Programmende oder bis zum nächsten Breakpoint aus.

  \emph{Hinweis:} Unter Umständen und ungünstigem Programmcode kann das zu debuggende Programm in eine Schleife geraten, welche ohne Breakpoints nur durch den Programmabbruch beendet werden kann.

\item[set] \emph{Register Zahl}

  weist dem \emph{Register} den Wert \emph{Zahl} zu.

\item[set-mem] \emph{Zahl1 Zahl2}

  schreibt den Wert \emph{Zahl2} an die Wortadresse \emph{Zahl1} im Hauptspeicher.

  \emph{Hinweis:} Auch wenn dieser Befehl offensichtlich dazu genutzt werden könnte den Assembler-Code zu manipulieren, ist er für diesen Zweck nicht vorgesehen.

\item[step] \emph{[Zahl]}

  führt die nächsten \emph{Zahl} Instruktionen aus; wird kein Argument gegeben, so wird eine Instruktion ausgeführt.

\item[trace-mac] \hspace*{\fill}\\
  
  der Assembler-Code wird nun beobachtet. Dadurch wird jede Assembler-Instruktion angezeigt, nachdem sie ausgeführt wurde.

\item[trace-mic] \hspace*{\fill}\\

  der Mikro-Assembler-Code wird nun beobachtet. Dadurch wird jede Mikro-Assembler-Instruktion angezeigt, nachdem sie ausgeführt wurde.

\item[trace-reg] \emph{[Register]}

  das gegebene \emph{Register} wird nun beobachtet. Dadurch wird der Wert des Registers angezeigt, wenn er sich ändert. Wird das optionale Argument weggelassen, werden alle \emph{Register} beobachtet.

\item[trace-var] \emph{Zahl}

  die Variable \emph{Zahl} wird nun beobachtet. Dadurch wird der Inhalt der Variable angezeigt, wenn er sich ändert.

  \emph{Zahl} ist die Nummer der lokalen Variable. Wird eine Methode im Assembler-Code aufgerufen, ändert sich der Zeiger \reg{LV} und damit auch die Identität der lokalen Variablen. Wird also Variable Nummer~1 in Methode X beobachtet und führt der \md{} gerade Methode Y aus, wird eine Änderung der jetzigen lokalen Variable~1 nicht ausgegeben (sofern diese nicht auch beobachtet wird).

\item[untrace-mac] \hspace*{\fill}\\
  
  beendet das Beobachten des Assembler-Codes. Dadurch werden ausgeführte Assembler-Instruktionen nun nicht mehr ausgegeben.

\item[untrace-mic] \hspace*{\fill}\\

  beendet das Beobachten des Mikro-Assembler-Codes. Dadurch werden ausgeführte Mikro-Assembler-Instruktionen nun nicht mehr ausgegeben.

\item[untrace-reg] \emph{[Register]}

  beendet das Beobachten des gegebenen \emph{Register}s. Wird das optionale Argument weggelassen, wird nun kein Register mehr beobachtet.

\item[untrace-var] \emph{Zahl}

  beendet das Beobachten der lokalen Variable Nummer \emph{Zahl}.
\end{description}

\section{Tutorial}
Du solltest den \md{}, wie zu Beginn dieses Kapitels beschrieben, nun entpackt haben und am besten schon der \texttt{PATH}-Variable hinzugefügt haben. Denn dann kannst du in einem Verzeichnis deiner Wahl die Dateien für dieses Tutorial anlegen.

Wir möchten nun ein kleines Programm schreiben und dies testen: ein Programm zum Einlesen von Binärzahlen. \lstref{binary-read-c} zeigt den Code für das Programm in C\notiz{C referenz? erklärn?} und soll hier als Verständnis des Algorithmuses dienen.

\begin{lstlisting}[language=c,caption={C-Programm zum Einlesen einer Binärzahl},label=\lstlbl{binary-read-c}]
int main() {
  int character = 0;
  int result = 0;
  while(1) {
    character = getchar();
    if( c == '\n' ) {
      return result;
    }
    c = c - '0';
    result = 2 * result + c;
  }
}
\end{lstlisting}

Das entsprechende Assembler-Programm findest Du in \lstref{binary-read-jas}. Dieses Programm solltest du nun kompilieren -- \emph{Ray Ontko} stellt dafür in \cite{Ontko1999} einige Programme bereit: \texttt{mic1asm} zum Kompilieren des Mikro-Assembler-Codes und \texttt{ijvmasm} zum Kompilieren des Assembler-Codes. In \cite{Ontko1999} findest du auch den Mikro-Assembler-Code und eine entsprechende \texttt{ijvm.conf}-Datei.

\begin{lstlisting}[language=,caption={IJVM-Assembler zum Einlesen einer Binärzahl},label=\lstlbl{binary-read-jas}]
.main
.var
    c
    result
.end-var
    bipush      0
    istore      result
loop:
    in
    istore      c
    iload       c
    bipush      10
    if_icmpeq   finish
    iinc        c       -48
    iload       result
    dup
    iadd
    iload       c
    iadd
    goto        loop
finish:
    iload       result
    halt
.end-main
\end{lstlisting}

Zum Debuggen des Mikro-Assembler-Codes und Assembler-Codes solltest Du die \texttt{ijvm.conf}-Datei verwenden, die Du zum Kompilieren der \datei{mic1} genutzt hast. Diese \date{conf} kannst Du entweder in das \texttt{conf/} Verzeichnis des Debuggers legen, oder jeweils in dem Verzeichnis, in dem Du gerade arbeitest. Dass Du nicht die korrekte \texttt{ijvm.conf}-Datei verwendest, siehst Du beim Ausführen des Befehls \texttt{ls-macro-code} -- zeigt dieser unbekannte Assembler-Befehle, enthält der Assembler-Code Befehle, die in der \texttt{ijvm.conf} nicht oder nicht an dieser Adresse definiert sind.

Bevor wir nun beginnen solltest Du auch die Konfigurationsoption \texttt{mic1.micro.address.ijvm} überprüfen. Diese Option enthält die Adresse der Mikro-Instruktion, die von allen Mikro-Code-Methoden angesprungen wird, um die nächste Mikro-Instruktion zu \emph{laden}. Ist diese Option falsch konfiguriert, funktioniert später der Befehl \texttt{step} nicht wie erwartet.

Die in \lstref{binary-read-jas} aufgeführte Methode soll später eine eigene Methode werden und gibt daher keinen Wert aus, sondern legt das Ergebnis am Ende auf den Stack. \lstref{tutorial-start} zeigt in \srcref{tutorial-startbefehl} den Befehl, um den \md{} zu starten -- im aktuellen Verzeichnis liegen die Dateien \texttt{mic1ijvm.mic1}, \texttt{binary-read.ijvm} und \texttt{ijvm.conf}.

\begin{lstlisting}[language=,caption={Start des \md{}},label=\lstlbl{tutorial-start}]
micro-debug.sh mic1ijvm.mic1 binary-read.ijvm(*@\srclbl{tutorial-startbefehl}@*)
MicroDebug - Copyright (C) 2011-2012 Christian Roesch AND 1999 Prentice-Hall, Inc. (*@\srclbl{tutorial-start-willkommen}@*)
Welcome! Please type 'help' for a list of valid commands
----------------------------------------
micro-debug> (*@\srclbl{tutorial-start-mdread}@*)
\end{lstlisting}

Ab \srcref{tutorial-start-willkommen} steht die Willkommensnachricht des \md{} gefolgt von der \srcref{tutorial-start-mdread}, die anzeigt, dass der \md{} nun einen Befehl erwartet. Wir können uns nun eine ausführliche Beschreibung der verschiedenen Befehle anzeigen, mit dem Befehl \texttt{help}.

Geben wir den Befehl \texttt{ls-macro-code} ein, erhalten wir den disassemblierten Assembler-Code, wie in \lstref{tutorial-macro-code} zu sehen. Die Ausgabe zeigt zunächst pro Zeile die Assembler-Code-Zeile, dann die Adresse des Befehls im Mikro-Assembler-Code und anschließend den Namen des Befehls mit seinen Argumenten. Die Ausgabe ist nicht identisch mit dem Code, den wir anfangs kompiliert haben -- in \lstref{binary-read-jas} -- in der disassemblierten Variante fehlen Informationen wie Kommentare, Variablennamen und Sprungmarkennamen. Beispielsweise zeigt \srcref{tutorial-macro-code-sprung} einen bedingten Sprung zur Zeile \texttt{0x1B}.

\begin{lstlisting}[language=,caption={Disassemblierter Assembler-Code},label=\lstlbl{tutorial-macro-code}]
     0x0: [ 0x10] BIPUSH  0x0
     0x2: [ 0x36] ISTORE  1
(*@\srclbl{tutorial-macro-code-in}@*)     0x4: [ 0xFC] IN 
     0x5: [ 0x36] ISTORE  0
     0x7: [ 0x15] ILOAD  0
     0x9: [ 0x10] BIPUSH  0xA
(*@\srclbl{tutorial-macro-code-sprung}@*)     0xB: [ 0x9F] IF_ICMPEQ  0x1B
     0xE: [ 0x84] IINC  0 0x0
    0x11: [ 0x15] ILOAD  1
    0x13: [ 0x59] DUP 
    0x14: [ 0x60] IADD 
    0x15: [ 0x15] ILOAD  0
    0x17: [ 0x60] IADD 
    0x18: [ 0xA7] GOTO  0x4
    0x1B: [ 0x15] ILOAD  1
    0x1D: [ 0xFF] HALT 
\end{lstlisting}

Das erwartete Ergebnis eines Programmlaufs ist, dass wir die eingegebene Binärzahl am Ende auf dem Stack und damit im Register \reg{TOS} vorfinden -- geben wir $1010$ ein, soll \reg{TOS} nach einem Programmdurchlauf den Wert $10$ enthalten.

Mit dem Befehl \texttt{run} lassen wir das Programm nun zunächst ohne Breakpoints laufen. Nachdem wir das Programm gestartet haben, wird die \mic{} in \srcref{tutorial-macro-code-in} aus \lstref{tutorial-macro-code} mit dem Befehl \texttt{IN} Zeichen einlesen. Das erkennst Du auf der Konsole an der \srcref{tutorial-mic-eingabe-txt} aus \lstref{tutorial-mic-eingabe} -- statt \texttt{micro-debug>} steht hier nun \texttt{mic1>}.

\begin{lstlisting}[language=,caption={\mic{} erwartet Eingabe},label=\lstlbl{tutorial-mic-eingabe}]
micro-debug> run
mic1> (*@\srclbl{tutorial-mic-eingabe-txt}@*)
\end{lstlisting}

Wir geben nun \texttt{1010} ein und bestätigen die Eingabe mit \texttt{ENTER} -- dadurch werden fünf Zeichen im \md{} gepuffert: Die vier Zeichen, die wir eingegeben haben plus ein Zeilenumbruch. Jedes Mal, wenn die \mic{} nun ein Zeichen benötigt, wird aus diesem Puffer gelesen. Erst wenn dieser leer ist, erscheint erneut die Eingabeaufforderung für den Benutzer.

\begin{lstlisting}[language=,caption={\md{} gibt Anzahl ausgeführter Zyklen aus},label=\lstlbl{tutorial-durchgelaufen}]
Processor executed 441 ticks.
micro-debug> 
\end{lstlisting}

Nachdem wir nun die Zahl eingegeben haben erscheint die Ausgabe aus \lstref{tutorial-durchgelaufen}, die uns anzeigt, dass die \mic{} insgesamt $441$ Zyklen ausgeführt hat. Da unser Programm keine Ausgabe macht, sondern das Ergebnis auf den Stack (und damit in dem Register \reg{TOS}) ablegt, überprüfen wir das nun wie folgt. Mit dem Befehl \texttt{ls-reg} lassen wir uns wie in \lstref{tutorial-tos-falscher-wert} zu sehen den Inhalt des Registers \reg{TOS} anzeigen.

Das Register \reg{TOS} enthält den Wert $0$ und damit einen falschen Wert. Wir können nun noch den Stack anzeigen, um zu überprüfen, ob dort der erwartete Wert $10$ abgelegt ist. Den Stack zeigen wir mit dem Befehl \texttt{ls-stack} an, was die in \lstref{tutorial-stack} gezeigte Ausgabe liefert.

\begin{lstlisting}[language=,caption={\md{} gibt Anzahl ausgeführter Zyklen aus},label=\lstlbl{tutorial-tos-falscher-wert}]
micro-debug> ls-reg TOS
Register TOS : 0x0
micro-debug> 
\end{lstlisting}

Der erwartete Wert $10$ liegt auch nicht auf dem Stack. Wäre der Wert auf dem Stack, aber nicht im Register \reg{TOS} wäre das ein Hinweis auf einen Fehler im Mikro-Assembler-Code, da dieser für die Einhaltung der Regel zuständig ist, dass das Register \reg{TOS} stets den Wert des obersten Elements des Stacks enthält.

\begin{lstlisting}[language=,caption={Inhalt des Stacks nach der Ausführung des Assembler-Programms},label=\lstlbl{tutorial-stack}]
Stack value #1 [  0xC001]: 0x1
Stack value #2 [  0xC002]: 0x0
Stack value #3 [  0xC003]: 0x1
Stack value #4 [  0xC004]: 0x0
Stack value #5 [  0xC005]: 0x0
micro-debug> 
\end{lstlisting}

Damit wir das Programm erneut ablaufen lassen können, müssen wir die \mic{} auf ihren Startzustand zurücksetzen; mit dem Befehl \texttt{reset}.
\chapter{Interaktion per GUI}
\label{bed-gui}

\clearpage

\part{Implementierung}
\chapter{Werkzeuge}
\chplbl{werkzeuge}
In diesem Teil der Arbeit möchte ich besonders eine Frage beantworten: Wie sind die bisher vorgestellten Funktionen des Debuggers implementiert? Mein Ziel ist, die Implementierung und Struktur des Debuggers so nachvollziehbar zu beschreiben, dass Leser mit Programmiererfahrung in der Lage sind, den Debugger weiterzuentwickeln.

Der Debugger soll auch nach Ende dieser Arbeit weiterentwickelt werden können. Daher stelle ich in diesem Kapitel vor, welche Werkzeuge ich zur Entwicklung des Debuggers nutze und wie diese zur Mitarbeit genutzt werden können. Dies ist nicht als Bedienungsanleitung zu verstehen! Ich bin aber der Ansicht, dass es das Verständnis für die Implementierung und meine Arbeit steigert, wenn geklärt ist, welche Werkzeuge ich wie nutze.

\section{Versionskontrollsystem}
Die Projekte \md und \mdg sind bei \name{GitHub} (siehe \cite{Roesch2012,Roesch2012gui}) gehostet und sind somit mit \gls{git} versioniert.

\gls{git} wird beispielsweise in \cite{Ohne1} beschrieben und ist demnach:
\begin{description}
\item[effizient] Auch bei großen Projekten zeigen Vergleiche, dass \gls{git} insgesamt schneller ist, als beispielsweise \gls{svn}.
\item[verteilt] Jeder Entwickler erhält ein lokales Repository und kann damit arbeiten. Es wird kein zentraler Server benötigt, alle Funktionen des Systems und alle Versionen stehen lokal zur Verfügung.
\item[sicher] Bei \gls{git} wird jeder Commit gehasht. Es ist praktisch nicht möglich einen Commit zu manipulieren, da dies die Hash-Summe verändern würde. Auch wenn theoretisch Kollisionen der Hash-Summe möglich sind, ist bisher eine solche Kollision noch nicht konstruierbar -- bei einem Code-Projekt müsste eine solche Kollision zudem kompilierbarer Code sein, was als hinreichend unwahrscheinlich gilt.
\end{description}

Ich habe \gls{git} gewählt, da dadurch später leicht \emph{forks}\footnote{Zu deutsch Abspaltung, ist ein Entwicklungszweig, nachdem sich ein Software-Projekt in zwei oder mehr Teile geteilt hat. Diese Teile werden meist unabhängig voneinander weiterentwickelt. Im täglichen Arbeitsablauf mit \gls{git} ist ein fork aber auch die Möglichkeit Neuerungen für ein Projekt zu Entwickeln und den entstandenen Entwicklungszeig nach Fertigstellung der Neuerung in das Ursprungsprojekt einzupflegen.} gebildet werden können und der Debugger durch Außenstehende weiterentwickelt werden kann. Das dezentrale Entwickeln mit \gls{git} ermöglicht es zusätzlich, dass später viele Entwickler am Debugger mitentwickeln.

Wie checkt man mit \gls{git} den Quelltext der Arbeit aus? Wie unterstützt mich \gls{git}, wenn ich das Projekt weiterentwickeln möchte? Im Folgenden möchte ich auf diese Fragen antworten. Ich möchte keine umfassende Bedienungsanleitung für \gls{git} geben, sondern die Grundlagen für die alltägliche Arbeit mit \gls{git} vermitteln. \gls{git} am \md erklären, für \mdg ist das Vorgehen analog, in den meisten Fällen muss lediglich ein \texttt{-gui} eingefügt werden.

\subsection{Code auschecken}
\seclbl{git-checkout}
Die Adresse des öffentlichen Repositorys lautet: \code{https://github.com/croesch/micro-debug.git}

\begin{lstlisting}[language=sh,caption={\md mit git auschecken},label=\lstlbl{git-auschecken}]
git clone git://github.com/croesch/micro-debug.git
cd micro-debug(*@\srclbl{git-auschecken-cd}@*)
\end{lstlisting}

Durch das Auschecken des Repositorys wird im aktuellen Verzeichnis ein Verzeichnis \texttt{micro-debug} angelegt. In diesem Verzeichnis wird zum einen der Inhalt des Repositorys abgelegt, als auch eine Kopie des Repositorys -- im Verzeichnis \texttt{micro-debug/.git}. \lstref{git-auschecken} zeigt, welcher Befehl zum Auschecken des Repositorys verwendet wird -- in \srcref{git-auschecken-cd} wird in das erzeugte Verzeichnis gewechselt, das Arbeitsverzeichnis.

Im Gegensatz zu \gls{svn} erstellt \gls{git} nur im Wurzelverzeichnis eines Projekts ein Verzeichnis \texttt{.git}. Direkt nach dem Klonen (bei \gls{git} für Auschecken) eines anderen Repositorys enthält es beispielsweise einen Verweis (\emph{origin}) auf dieses Repository. Sobald das geklonte Repository neuere Commits enthält, kann dieser Verweis genutzt werden, um wie in \lstref{git-pull} die neuen Commits in das lokale Repository zu übernehmen.

\begin{lstlisting}[language=sh,caption={Mit git \emph{pull} auf Originalrepository ausführen},label=\lstlbl{git-pull}]
git pull origin
\end{lstlisting}

Mit diesem Befehl werden die neuen Commits heruntergeladen und direkt in den aktuellen Branch eingefügt.

Dieses Vorgehen ist aufgrund der Änderung des aktuellen Branches häufig unerwünscht, daher kann man den Befehl aus \lstref{git-pull} wie in \lstref{git-fetch-merge} in zwei Befehle aufteilen, um zu regeln, ob und welcher Branch die Neuerungen erhält.

\begin{lstlisting}[language=sh,caption={\emph{pull} in zwei Befehlen manuell ausführen},label=\lstlbl{git-fetch-merge}]
git fetch origin(*@\srclbl{git-fetch-fetch}@*)
git merge origin/master(*@\srclbl{git-fetch-merge}@*)
\end{lstlisting}

Mit \emph{fetch} in \srcref{git-fetch-fetch} werden die Aktualisierungen heruntergeladen und zunächst im \texttt{.git}-Verzeichnis gespeichert. Erst der Befehl \emph{merge} in \srcref{git-fetch-merge} verändert die lokalen Dateien im aktuellen Branch.

\subsection{Code beitragen}
Hat man ein Repository geklont und möchte Codeänderungen dem Originalrepository zuführen, gibt es zwei Möglichkeiten: Man benutzt den Standard-Branch oder so genannte Feature-Branches.

Den Standard-Branch zu nutzen, empfehle ich nur für die Entwicklung als Einzelperson. Entwickelt man alleine an einem Projekt, kann man veränderte Dateien wie folgt committen:
\begin{verbatim}
$ git add relativer-pfad-zur-datei
$ git commit -m "Commit-Nachricht"
\end{verbatim}

Für mehrköpfige Entwicklungsteams empfehle ich Feature-Branches zu nutzen, da \gls{git} gute branch-/merge-Qualitäten aufweist. Beispielsweise können Sie so selbst Code zu dem Projekt beitragen, indem Sie folgende Schritte durchführen:
\begin{enumerate}
\item Erstellen Sie unter \code{https://github.com/croesch/micro-debug/issues} ein Thema, das Ihre Entwicklung beschreibt.

\item Erstellen Sie einen lokalen Branch zu dem Thema, wobei die Nummer die Nummer des Themas ist:
\begin{verbatim}
$ git checkout -b 100-themen-titel
\end{verbatim}

\item Entwickeln Sie den Code weiter.

\item Committen Sie die Änderungen:
\begin{verbatim}
$ git add .
$ git commit -m 'Commit-Nachricht'
\end{verbatim}

\item Laden Sie den Stand des öffentlichen Repositories:
\begin{verbatim}
$ git fetch origin
\end{verbatim}

\item Aktualisieren Sie den lokalen master-Branch:
\begin{verbatim}
$ git checkout master
$ git pull origin master
\end{verbatim}

\item Wiederholen Sie die Schritte 2-6, bis Ihre Entwicklung abgeschlossen ist

\item Aktualisieren Sie den Punkt von dem Ihr Branch abgeht:
\begin{verbatim}
$ git checkout 100-themen-titel
$ git rebase master
\end{verbatim}

\item Stellen Sie Ihren Branch öffentlich zur Verfügung.
\begin{verbatim}
$ git push origin 100-themen-titel
\end{verbatim}
\end{enumerate}

Da Sie für dieses Projekt auf github keine Schreibberechtigung haben, ist der letzte Punkt nicht so einfach möglich. Daher wählt man besonders auf gitHub einen leicht anderen Weg, den ich nun kurz vorstellen möchte.

\subsection{Unterstützung durch gitHub}
Die Grundidee ist, ein beliebiges Repository auf github zu klonen, so dass dieser Klon wiederum öffentlich zugänglich ist. Sie klonen dann den Klon von github auf Ihren lokalen Rechner, um daran zu entwickeln. Anschließend können Sie Ihr Ergebnis durch Ihren github-Klon veröffentlichen, so dass diese Änderung wieder in das Originalprojekt einfließen kann. Diesen Prozess möchte ich nun erläutern.

\begin{enumerate}
\item Registrieren Sie sich auf github, um dort Projekte veröffentlichen zu können. Legen Sie ssh-Schlüsselpaare an und laden Sie Ihren öffentlichen Schlüssel auf Ihr Benutzerkonto, dass Sie sich später bei Schreiboperationen authentifizieren können.

\item Klicken Sie auf der Projektseite \code{https://github.com/croesch/micro-debug} den \emph{fork}-Button. Dadurch klont github das Projekt auf Ihr Benutzerkonto, wo es nun unter \code{https://github.com/ihr-name/micro-debug} verfügbar ist.

\item Klonen Sie Ihre Repository nun lokal, wie in \secref{git-checkout} beschrieben:
\begin{verbatim}
$ git clone git://github.com/ihr-name/micro-debug.git
$ cd micro-debug
\end{verbatim}

\item Da Ihr Repository unter \emph{origin} nun Ihren github-Klon referenziert, setzen Sie eine Referenz auf das Hauptprojekt:
\begin{verbatim}
$ git remote add original-projekt git://github.com/croesch/micro-debug.git
\end{verbatim}

\item Erstellen Sie unter \code{https://github.com/croesch/micro-debug/issues} ein Thema, das Ihre Entwicklung beschreibt.

\item Erstellen Sie einen lokalen Branch zu dem Thema, wobei die Nummer die Nummer des Themas ist:
\begin{verbatim}
$ git checkout -b 100-themen-titel
\end{verbatim}

\item Entwickeln Sie den Code weiter.

\item Committen Sie die Änderungen:
\begin{verbatim}
$ git add .
$ git commit -m 'Commit-Nachricht'
\end{verbatim}

\item Laden Sie den Stand des öffentlichen Repositories:
\begin{verbatim}
$ git fetch original-projekt
\end{verbatim}

\item Aktualisieren Sie den lokalen master-Branch:
\begin{verbatim}
$ git checkout master
$ git pull original-projekt master
\end{verbatim}

\item Wiederholen Sie die Schritte 2-6, bis Ihre Entwicklung abgeschlossen ist

\item Aktualisieren Sie den Punkt von dem Ihr Branch abgeht:
\begin{verbatim}
$ git checkout 100-themen-titel
$ git rebase master
\end{verbatim}

\item Veröffentlichen Sie Ihren fertigen Branch über Ihren github-Klon:
\begin{verbatim}
$ git push origin 100-themen-titel
\end{verbatim}

\item Klicken Sie auf Ihrer Projekt-Seite auf \emph{Pull-Request}. Anschließend kann Ihr neuer Branch beispielsweise von mir in das Originalprojekt eingepflegt werden. Dann können Sie und alle anderen, die das Projekt geklont haben, die Änderungen über \emph{fetch} in die geklonten Repositories übernehmen.
\end{enumerate}

Sie haben nun die Möglichkeit den aktuellen Code lokal zu bearbeiten und Änderungen zum Projekt beizutragen. Im Folgenden möchte ich das Build-Management-Werkzeug erläutern, dass ich zum Entwickeln nutze: \gls{mvn}.

\section{Build-Management}
\gls{mvn} ist ein Werkzeug zum Bauen von \gls{java}-Anwendungen. Meines Erachtens gibt es keine signifikanten Gründe für oder gegen die Nutzung von \gls{mvn}; stattdessen hätte ich auch \gls{ant} oder \gls{make} nutzen können. In diesem Abschnitt möchte ich die grundlegenden Befehle für die Arbeit mit \gls{mvn} erläutern.

Ich verwende \gls{mvn}, da hier viel implizites Wissen eingesetzt wird; dadurch wird der Konfigurationsaufwand geringer -- solange man den Konventionen der \gls{mvn}-Entwickler folgt. Zu dem impliziten Wissen gehört unter anderem:
\begin{itemize}
\item \gls{java}-Klassen befinden sich in \code{src/main/java/}
\item \gls{java} Test-Klassen befinden sich in \code{src/test/java/}
\item Kompilate und von \gls{mvn} generierte Klassen befinden sich in \code{target/}
\item Die explizite Konfiguration steht in der Datei \code{pom.xml}
\end{itemize}

\subsection{Code kompilieren}
Mit dem folgenden Befehl können Sie den Programmcode kompilieren; Testklassen werden dadurch weder kompiliert noch ausgeführt:
\begin{verbatim}
$ mvn compile
\end{verbatim}

\gls{mvn} erkennt verschiedene Abhängigkeiten. Das führt dazu, dass beim ersten Ausführen zunächst erstmal alle benötigten Bibliotheken heruntergeladen werden. Zusätzlich erkennt \gls{mvn} auch, ob für ein bestimmtes Ziel die Klassen zunächst kompiliert werden müssen.

Beispielsweise werden zum Packetieren einer Anwendung zunächst alle Klassen kompiliert, anschließend die Tests ausgeführt, bevor die Anwendung packetiert werden kann; dies geschieht mit:
\begin{verbatim}
$ mvn package
\end{verbatim}

\subsection{Tests ausführen}
Wenn Sie alle Tests ausführen möchten, können Sie dies mit folgendem Befehl erreichen:
\begin{verbatim}
$ mvn test
\end{verbatim}

Je nachdem, welche Metriken Sie per Plugins noch mit \gls{mvn} ausführen, kann es nötig sein die von \gls{mvn} generierten Dateien zu löschen. Beispielsweise bei Code-Coverage-Werkzeugen ist dies nötig; alle von \gls{mvn} generierten Dateien -- das Verzeichnis \code{target} -- entfernen Sie mit:
\begin{verbatim}
$ mvn clean
\end{verbatim}

\subsection{Abhängigkeiten ändern}
Die Date \code{pom.xml} enthält die explizite Konfiguration eines Projektes. Darin enthalten sind unter anderem der Name und die Version des Projekts, aber auch abhängige Projekte.

Beispielsweise nutzt diese Arbeit \gls{fest} als Test-Werkzeug, in der \code{pom.xml} steht dazu Folgendes:
\begin{verbatim}
<dependency>
    <groupId>org.easytesting</groupId>
    <artifactId>fest-swing</artifactId>
    <version>1.2.1</version>
    <scope>test</scope>
</dependency>
\end{verbatim}

Dort können Sie nun die Version des Werkzeugs ändern, die genutzt werden soll; oder den Bereich von \emph{test} auf \emph{runtime} erweitern. Anschließend empfehle ich die Dateien zu entfernen, die mit \gls{mvn} zuvor generiert wurden und \gls{mvn} auszuführen.

\chapter{Implementierung der \mic}
\chplbl{prozessor}
Nachdem wir nun die Vorteile der automatisierten Tests kennen gelernt haben, wollen wir uns nun dem Code widmen. Der erste Schritt bei der Implementierung des \md ist die Simulation der \mic, erst wenn dies zufriedenstellend funktioniert, können wir die Debug-Funktionen entwickeln. Wir werden uns daher in diesem Kapitel die Implementierung des Prozessors anschauen und welche Besonderheiten es gibt.

Für die Implementierung der \mic ist es nötig, sich ein Abstraktionsniveau auszusuchen, auf dem implementiert wird. Wir werden in den folgenden Abschnitten sehen, dass ich mich zu einer relativ hardwarenahen Implementierung entschieden habe. Daher habe ich zunächst folgende zu implementierende Komponenten identifiziert: die \gls{alu} inklusive Shifter, die Register, der Mikro-Code-Speicher inklusive der darin abgelegten Instruktionen und der Hauptspeicher.

Aufgrund der \mvn\notiz{maven Referenz?}-Architektur liegt der Java\notiz{Java Referenz?}-Code in dem Verzeichnis \texttt{src/main/java/} und die entsprechenden Tests dazu in \texttt{src/test/java/}. In diesen Verzeichnissen findest du eine \package-Struktur, die unter dem \package \pck{com.github.croesch.micro_debug} beginnt, alle Klassen des \md sind in diesem \package oder in \subpackages zu finden.

Auch der Code für die \mic befindet sich in einem solchen \subpackage: \pck{com.github.croesch.micro_debug.mic1}. In diesem Verzeichnis gibt es die Klasse \klasse{Mic1}, die wir uns in \secref{mic-zusammensetzung} genauer anschauen werden. Alle weiteren Klassen, die den \mic bilden sind in folgenden \subpackages zu finden:
\begin{description}
\item[alu] bildet die \gls{alu} ab und ist in \secref{mic-alu} beschrieben.
\item[api] enthält einige Interfaces, um die Implementierung der \mic unabhängig gestalten zu können.
\item[controlstore] bildet den Mikro-Code-Speicher und die Mikro-Instruktionen ab und ist in \secref{mic-instructions} beschrieben.
\item[io] bildet die Ein- und Ausgabe der \mic ab. Die Klasse \klasse{Input} dient zum Einlesen einzelner Zeichen und enthält einen Puffer, der die noch nicht von der \mic gelesenen Zeichen enthält, da vom Benutzer nur zeilenweise Eingaben gemacht werden können.

Die Klasse \klasse{Output} dient zur Ausgabe einzelner Zeichen und puffert, falls die Ausgabe gepuffert erfolgen soll, die ausgegebenen Zeichen bis ein Zeilenumbruch ausgegeben wird.
\item[mem] bildet den Hauptspeicher ab und ist in \secref{mic-mem} beschrieben.
\item[mpc] bildet die Komponenten ab, die zusammen für die Berechnung des nächsten \gls{mpc} verantwortlich sind und ist in \secref{mic-mpc} beschrieben.
\item[register] bildet die Register der \mic ab und ist in \secref{mic-register} beschrieben.
\item[shifter] bildet den Shifter ab und ist in \secref{mic-alu} beschrieben.
\end{description}

Durch die \package-Struktur solltest Du schon einen groben Einblick haben, wo welche Komponenten implementiert sind. Zum besseren Verständnis möchte ich nun nochmal die Hardware-Komponenten der \mic betrachten und ihre Implementierungen beschreiben.

\section{ALU}
\seclbl{mic-alu}
Die \gls{alu} der \mic setzt sich aus 32~Ein-Bit-\gls{alu}s zusammen, so ist die \gls{alu} auch implementiert. Die Klasse \klasse{Alu} benutzt 32~Instanzen der Klasse \klasse{OneBitAlu}, zur Berechnung der Ausgabewerte -- damit ist die \gls{alu} die Komponente, die am hardwarenahesten implementiert ist.

Die Klasse besitzt eine Methode \texttt{calculate()}, in der aus den Eingangssignalen das Ausgangssignal der \gls{alu} berechnet wird. Dieses Signal wird an den Shifter weitergeleitet, die Klasse \klasse{Shifter} im gleichnamigen \package bildet diese Funktionalität ab und besitzt ebenso eine Methode \texttt{calculate()}, um die Verarbeitung der Signale anzustoßen.

Die Berechnung der \gls{alu} und des Shifters werden bei jedem Zyklus der \mic angestoßen. Durch die sehr hardwarenahe Implementierung bilden sie daher vermutlich ein Performanzengpass zur Laufzeit des \md.

\section{Register}
\seclbl{mic-register}
Die Dateneingänge der \gls{alu} wird mit den Inhalten zweier Register gefüllt und das Ergebnis des Shifters wird wiederum in verschiedene Register geschrieben. Die Register sind im \md als Enumeration \klasse{Register} im gleichnamigen \package implementiert. Sie erfüllen lediglich die Aufgabe einer 32~Bit Variable.

Das Register \reg{MBR} kann im \mic mit oder ohne Vorzeichenerweiterung gelesen werden. Dieses Verhalten ist im \md dadurch realisiert, dass es ein Register \reg{MBR} und ein Register \reg{MBRU} gibt. \reg{MBRU} enthält daher den Wert ohne Vorzeichenerweiterung und das Register \reg{MBR} enthält den Wert mit Vorzeichenerweiterung. Bei der Vorzeichenerweiterung wird wie bei der \mic davon ausgegangen, dass der ursprüngliche Wert in 8~Bit vorlag.

Dass die Register als Enumeration implementiert sind hat den Vorteil, dass man von allen Klassen leicht auf die Register zugreifen kann, aber den Nachteil, dass allein durch die Code-Struktur nicht deutlich wird, zu welcher logischen Einheit die Register gehören.

\section{Mic1-Instruktionen}
\seclbl{mic-instructions}
Welche Berechnung mit welchen Registerwerten ausgeführt wird und in welches Register das Ergebnis geschrieben wird, regeln die Signale der Mikro-Instruktionen. Die Mikro-Instruktionen (\klasse{MicroInstruction}) sind im Mikro-Code-Speicher (\klasse{MicroControlStore}) abgelegt.

Zur besseren Lesbarkeit des Codes nutzt die Implementierung der Mikro-Instruktion Unterklassen von \klasse{SignalSet}, um die verschiedenen Signale zu gruppieren. Die Darstellung der Mikro-Instruktion für den Benutzer ist in der Klasse \klasse{MicroInstructionDecoder} implementiert. Wie wird eine Mikro-Instruktion erzeugt? Der Mikro-Code-Speicher ist in der \datei{mic1} definiert; aus dieser Datei kann die Klasse \klasse{MicroInstructionReader} einzelne Mikro-Instruktionen erzeugen.

Meine Implementierung der Mikro-Instruktion, der Darstellung einer Mikro-Instruktion und dem Einlesen einer Mikro-Instruktion basieren auf der Implementierung von Ray Ontko\notiz{Verweis}, der einen Simulator für die \mic in Java entwickelt hat.

\section{MPC-Berechnung}
\seclbl{mic-mpc}
Der \gls{mpc} bestimmt, welche Mikro-Instruktion im nächsten Zyklus ausgeführt werden soll. In der \mic sind mehrere Komponenten gemeinsam für die Berechnung des \gls{mpc} verantwortlich. Diese verschiedenen Komponenten habe ich in der Klasse \klasse{NextMPCCalculator} zusammengefügt; auch hier gibt es eine Methode \texttt{calculate()} zum Verarbeiten der Eingangssignale.

\section{Hauptspeicher}
\seclbl{mic-mem}
Der Stack, der Assembler-Code und die Konstanten liegen im Hauptspeicher, mit diesem kommuniziert die \mic über die Register \reg{MAR}, \reg{MDR}, \reg{MBR} und \reg{PC}. Der Hauptspeicher ist in der Klasse \klasse{Memory} implementiert, der Speicher selbst ist in einem \emph{int}-Array abgebildet, weswegen der Hauptspeicher im \md wortweise adressiert wird.

Die Assembler-Instruktionen im Hauptspeicher sind nicht so sauber implementiert wie die Mikro-Instruktionen: sie sind im \emph{int}-Array abgelegt. Lediglich für das disassemblieren werden die Zahlenwerte anhand der \texttt{ijvm.conf} durch den \klasse{IJVMConfigReader} in Instruktionsobjekte (\klasse{IJVMCommand} inklusive \klasse{IJVMCommandArgument}) gepackt.

\section{Zusammensetzung der Komponenten}
\seclbl{mic-zusammensetzung}
Die einzelnen Komponenten werden in der Klasse \klasse{Mic1} zusammengeführt und von außen als ein Prozessor gesehen. Lediglich für gewisse Debug-Funktionen ist der Zugriff auf einzelne Komponenten, wie beispielsweise die Register, zugelassen.

Die Klasse \klasse{Mic1} wird mit bekommt zwei Objekte der Klasse \klasse{InputStream} -- eines enthält den Mikro-Assembler- und das andere den Assembler-Bytecode. Mit diesen beiden Objekten wird dann der Mikro-Code-Speicher und der Hauptspeicher erzeugt, die jeweils selbst verantwortlich für das Lesen des Bytecodes sind. Stimmt eine \emph{magic number} nicht, oder sind die Eingabedateien aus anderen Gründen ungültig, wird eine \emph{Exception} geworfen.

Auf die einzelnen Methoden möchte ich hier nicht eingehen, aber eine Methode sei hier erwähnt: In der Methode \texttt{doTick()} wird ein einzelner Zyklus der \mic ausgeführt. Diese Methode ist nur für Testzwecke sichtbar und wird von außen über die Methoden \texttt{run()}, \texttt{step()} und \texttt{microStep()} aufgerufen, die unterschiedlich viele Zyklen aufeinmal abarbeiten lassen.

Aufgrund der gesammelten Funktionalität ist die Klasse \klasse{Mic1} eine der größten und damit komplexesten im \md.
\chapter{Interaktion per Konsole}
\label{konsole}
\chapter{Implementierung der GUI}
\chplbl{gui}
Nachdem wir nun ausführlich den \md betrachtet haben, möchte ich Dir nun vorstellen, wie die \mdg den \md nutzt und welche Klassen hier welche Rolle einnehmen. Ähnlich wie in \chpref{konsole} möchte ich im \secref{gui-ablauf} den Ablauf bei der Nutzung der \mdg vorstellen. In \secref{g-bibliotheken} möchte ich die verwendeten Bibliotheken vorstellen, die der Benutzer am Ende erhält und deren Bedeutung für die \mdg erläuetern. Am Ende dieses Kapitels möchte ich auch hier nochmal alle \packages aufführen und beschreiben, welche Funktionalität dahinter steckt und eventuell einzelne Klassen erwähnen.

Die \mdg ist ein eigenständiges Projekt und vom her ein Benutzer des \md. Alle Implementierungen der \mdg haben also keinerlei Auswirkung auf den \md, umgekehrt allerdings schon. Wie in \secref{gui-nutzung-konsole} beschrieben ist es daher möglich, dass bei vorliegender \mdg auch der \md genutzt werden kann.

Im Unterschied zur Implementierung des \md befinden sich in der \mdg alle Klassen im \package \pck{com.github.croesch.micro_debug.gui} oder in \subpackages davon. Das verhindert, dass gleichnamige \packages oder Klassen zu Problemen führen. Außerdem ist je nach Paketierung die Struktur der paketierten \mdg übersichtlicher.

\section{Programmablauf}
\seclbl{gui-ablauf}
Wie in \secref{k-ablauf} sehen wir uns gleich den Start der \mdg an und ich zeige, welche Klassen an welchen Aktionen beteiligt sind. Gelegentlich werde ich erwähnen, wo Zugriffe auf den \md stattfinden oder wann der Benutzer gefragt ist.

\subsection{Verarbeitung der Argumente}
Analog zum \md gibt es auch in der \mdg eine Klasse \klasse{MicroDebug}, die die \texttt{main()}-Methode enthält. Allerdings ist es hier irrelevant, wie viele Argumente der Benutzer eingegeben hat, oder ob er überhaupt welche eingegeben hat.

Die \mdg nutzt auch den Mechanismus der Klasse \klasse{AArgument}. Da die \mdg allerdings andere Argumente als der \md besitzt, müssen die neuen Argumente zunächste an der Klasse \klasse{AArgument} registriert werden. Die Argumente der \mdg findest Du im \package \pck{argument}; registriert werden diese in der Methode \texttt{createListOfPossibleArguments()} der Klasse \klasse{MicroDebug}.

Die Argumente werden dann auch ausgeführt und geben jeweils einen Wahrheitswert zurück, ob der Debugger starten darf oder nicht.

\subsection{Aufbau der grafischen Oberfläche}
Darf der Debugger starten, so wird eine Instanz der Klasse \klasse{Mic1Starter} erzeugt, die den Debugger starten kann. Zunächst erzeugt die Klasse jedoch das Startfenster, welches in der Klasse \klasse{StartFrame} definiert ist, und zeigt dieses an. Das Startfenster enthält einige Aktionsmöglichkeiten -- hat der Benutzer nun die richtigen Dateien ausgewählt und den Button zum Start gedrückt, wird an der Klasse \klasse{Mic1Starter} die Methode \texttt{create(String,String)} mit den zwei angegebenen Dateipfaden ausgeführt.

Wie in \secref{k-aufbau-debugger} wird nun anhand dieser Dateipfade versucht die \mic zu erzeugen. Gelingt dies nicht, wird das Startfenster erneut gezeigt. Hat der Benutzer zwei korrekte Dateien angegeben, wird das Hauptfenster aufgebaut und angezeigt, was in der Klasse \klasse{MainFrame} definiert ist.

In der Klasse \klasse{MainFrame} werden drei erwähnenswerte Objekte erzeugt: die \emph{View}, der \emph{Controller} und die \emph{Actions}. Die \emph{View} bildet die Oberfläche und ist zunächst in der Klasse \klasse{MainView} definiert, die wiederum weitere Klassen zur Darstellung der einzelnen Bereiche nutzt. Ein ähnliches Konzept nutzt der \emph{Controller}, der in der Klasse \klasse{MainController} definiert ist: Für die einzelnen kleinen \emph{Views} werden entsprechende \emph{Controller} erzeugt, die dadurch jeweils nur einen kleinen Aufgabenbereich erhalten. Das dritte Objekt ist von der Klasse \klasse{ActionProvider} und erzeugt die \emph{Actions}, die der Benutzer über die Oberfläche erreichen kann. Dieser \emph{ActionProvider} hält die Referenzen auf die \emph{Actions}, so dass nur dieses eine Objekt weitergereicht werden muss anstatt dutzender \emph{Actions}.

Diese \emph{Actions} werden dann noch in der \klasse{MainMenuBar} untergebracht und mit Tastenkombinationen versehen, die der Benutzer konfiguriert hat.

\subsection{Verarbeitung von Benutzeraktionen}
\seclbl{verarbeitung-benutzeraktionen-threads}
Nachdem die Oberfläche aufgebaut ist, sind die \emph{Actions} der Ausgangspunkt von weiteren Code-Ausführungen. Bis der Benutzer die \texttt{EXIT}-Aktion ausführt oder das Fenster schließt, läuft die \mdg.

In der Klasse \klasse{ActionProvider} werden die Referenzen auf die \emph{Actions} gehalten und unter den Schlüsseln abgelegt, die die Enumeration \klasse{Actions} bietet.

Bis auf einige Ausnahmen werden die ganzen \emph{Actions} auf dem \gls{edt} ausgeführt. Bei einigen \emph{Actions} wäre dies von Nachteil: spätestens wenn die \mic eine Eingabe erwarten würde, käme es zu einem Deadlock\notiz{Deadlock erklärn!?}. Denn die Eingabe für die \mic wird in der \mdg über ein Textfeld ausgeführt. Liest die \mic nun auf dem \gls{edt} ein Zeichen aus diesem Textfeld, obwohl dort noch keines eingegeben wurde, dann blockiert der aktuelle Thread (in diesem Fall der \gls{edt}) bis der Benutzer eine Eingabe gemacht hat. Eine Eingabe kann aber nur über den \gls{edt} ausgeführt werden, der noch blockiert ist -- ein Deadlock.

Aus diesem Grund gibt es die Klasse \klasse{AbstractExecuteOnWorkerThreadAction}. Jede \emph{Action}, die davon ableitet, wird nicht auf dem \gls{edt} ausgeführt sondern auf einer Instanz der Klasse \klasse{WorkerThread}. Dieser Thread läuft und arbeitet kontinuierlich Objekte der Klasse \klasse{Runnable} ab und wird hier genutzt, um solche Deadlocks zu umgehen.

\subsection{Konfiguration und Lokalisierung}
Das Lesen der Konfiguration und der Textkonstanten funktioniert analog zu den Ausführungen in \secref{k-konfiguration} und \secref{k-lokalisierung}.

Allerdings nutzt die \mdg hier eigene Enumerations-Klassen: die Klasse \klasse{GuiText} für die Textkonstanten und die Klassen im \package \pck{settings} zum Verarbeiten der Konfiguration. Somit können Komponenten der \mdg sowohl auf die Konfiguration des \md als auch der \mdg zugreifen und auch auf Textkonstanten aus beiden Projekten.

Wie in \secref{lokalisierung} bereits erwähnt, nutzt die \mdg zur Lokalisierung einen eigene Datei-Hierarchie: Alle \datei{xml}en, die mit \texttt{text-gui} beginnen, enthalten Textkonstanten für die \mdg. Bei der Konfiguration wird allerdings die selbe Datei verwendet, die der \md auch nutzt. Das hat für den Benutzer den Vorteil, dass er nur eine Datei zu pflegen hat.

Für die Pflege des Projekts \mdg bedeutet dies aber, dass sowohl die \datei{xml}en, die nur mit \texttt{text} beginnen, als auch die Konfigurationsoptionen des \md aktualisiert werden müssen, wenn die Version des benutzten \md geändert wird. Da die Projekte aber derzeit beide zusammen entwickelt werden, ist dies noch kein Problem.

\section{Bibliotheken}
\seclbl{g-bibliotheken}
Wie gerade angemerkt, wird der \md von der \mdg benutzt. Der Code des \md muss also der \mdg vorliegen; derzeit wird der \md der \mdg als \datei{jar} übergeben und dem Klassenpfad hinzugefügt. Dies ermöglicht es dem Benutzer, die verwendete Version des \md auszutauschen.

Die \mdg enthält nur die grafische Oberfläche, das komplette Wissen über die \mic ist in der Bibliothek -- dem \md -- enthalten. Diese Bibliothek ist also sehr wichtig und in der Regel nicht vorgesehen, um durch neuere Versionen ersetzt zu werden.

Eine weitere Bibliothek, die der Benutzer erhält, ist das miglayout\notiz{Referenz und Befehl}. Diese Bibliothek wird benötigt, um die Oberflächenelemente zu positionieren. Das miglayout bietet eine komfortable Schnittstelle, für das Layout von Komponenten.

Die Bibliothek miglayout wird zwar auch zwingend benötigt, kann aber durchaus durch neuere Versionen ersetzt werden. Auch wenn hier prinzipiell das Risiko der Inkompatibilität zu künftigen Versionen besteht, ist das Risiko hier geringer, da nur sehr wenige Schnittstellen genutzt werden.

An dieser Stelle möchte ich auch das Testframework \gls{fest} erwähnen: Die dazugehörigen Bibliotheken werden zwar nicht an den Benutzer ausgeliefert, aber werden zur Entwicklung benötigt. Beim \md wird dieses Framework auch schon verwendet, allerdings nur als andere \gls{api} anstatt \gls{junit}~4. Bei der \mdg wird \gls{fest} genutzt, um \gls{gui}-Tests zu schreiben.

Die \gls{gui}-Tests laufen in der Regel stabil, allerdings neigen sie häufiger zum Fehlschlagen, ohne tatsächlichen Fehler in der \gls{aut} oder dem Test selbst. Daher solltest Du beim Ausführen der Tests damit rechnen, dass gelegentlich der ein oder andere \gls{gui}-Test fehlschlägt. \gls{fest}-Tests\notiz{Oder hier Verweis auf Bericht?} die realen Resourcen deines Computers nutzen, kannst du diesen während die Tests laufen nicht benutzen, sonst schlagen die Tests fehl.

\section{\packages}
Nun hast Du die wichtigsten Klassen der \mdg kennen gelernt. Damit Du aber später weißt, wo etwas implementiert ist, möchte ich auch hier nochmal alle \packages erwähnen und erklären, welche Funktionalität die darin enthaltenen Klassen erfüllen. Die \packages sind im Einzelnen:

\begin{description}
\item[actions] enthält alle \emph{Actions}, die in der \mdg ausgeführt werden können. Hier ist auch die Klasse \klasse{AbstractExecuteOnWorkerThreadAction} enthalten, deren Unterklassen nicht auf dem \gls{edt} ausgeführt werden. In diesem \package gibt es noch ein \subpackage: \pck{api}, das enthält einige Interfaces, um zyklische Abhängigkeiten zu verhindern.
\item[argument] enthält die Unterklassen von \klasse{AArgument}, die die gültigen Argumente der \mdg darstellen. Diese müssen an der Klasse \klasse{AArgument} registriert werden, um dem Benutzer zur Verfügung zu stehen.
\item[commons] enthält bisher nur die Klasse \klasse{WorkerThread}, also wie beim \md Klassen, die zu keinem anderen \package zugeordnet werden konnten.
\item[components] enthält in den \subpackages alle Oberflächenelemente und das Hauptfenster -- die Klasse \klasse{MainFrame}.
\item[components.about] enthält die Klasse \klasse{AboutFrame} und damit alle Komponenten, um den \emph{Über}-Dialog anzuzeigen.
\item[components.api] enthält einige Interfaces, um zyklische Abhängigkeiten zu verhindern.
\item[components.basic] ist ein sehr großes \package mit vielen Klassen. Diese haben aber selten eigene Logik sondern bilden Unterklassen der bekannten Swing-Komponenten, von denen dann alle Komponenten in der \mdg ableiten können. Diese Klassen haben beispielsweise erweiterte Konstruktoren, so dass jedes Oberflächenelement standardmäßig das \texttt{name}-Attribut gesetzt hat. Das \texttt{name}-Attribut ist für die automatisierten \gls{gui}-Tests wichtig.\notiz{Referenz auf Bericht?}
\item[components.code] enthält die Komponenten, die zur Darstellung des (Mikro-)Assembler-Codes benötigt werden:
  \begin{description}
  \item[ACodeArea.java] eine abstrakte Klasse der Textkomponente, die den Code enthält. Hiervon leiten \klasse{MicroCodeArea} und \klasse{MacroCodeArea} ab, die das sind, was ihr Name vermuten lässt.
  \item[ACodeFormatter.java] eine abstrakte Klasse, die für die Syntaxhervorhebung im Code verantwortlich ist. Auch hier gibt es zwei Ableitungen: \klasse{MicroCodeFormatter} und \klasse{MacroCodeFormatter}.
  \item[LineNumberLabel.java] die Komponente zur Darstellung der Zeilennummerierung.
  \item[Ruler.java] die Komponente zur Darstellung der Breakpoints.
  \end{description}
\item[components.controller] enthält die Klasse \klasse{MainController} und die entsprechenden \emph{Controller} mit kleinerem Verantwortungsbereich, beispielsweise \klasse{RegisterController}. 
\item[components.start] enthält die Komponenten, die vor der Klasse \klasse{MainFrame} aktiv sind: \klasse{Mic1Starter} und den \klasse{StartFrame}.
\item[components.view] enthält die Klasse \klasse{MainView} und die entsprechenden \emph{Views}, die nur einen gewissen Teil darstellen, beispielsweise die \klasse{MicroCodeView}. Hier sind auch Komponenten, wie die \klasse{MainMenuBar} und \klasse{NumberStyleSwitcher} vorhanden.
\item[debug] enthält die Klassen, die die Breakpoints behandeln. Hier gibt es eine zusätzliche Abstraktionsschicht zwischen der Klasse \klasse{BreakpointManager} aus dem \md und beispielsweise \klasse{Ruler}, die diese \emph{Handler} benutzt. Diese Schicht ist nötig, um auf die korrekten Zeilennummern zu schließen, das wird mit Hilfe der Klasse \klasse{LineNumberMapper} erreicht. Sie wird dazu genutzt, um zumindest für den Assembler-Code von den fortlaufenden Zeilennummern der Klasse \klasse{Ruler} auf die angezeigten Zeilennummern für den Benutzer zu schließen.

Zusätzlich gibt es die Klasse \klasse{MicroLineBreakpointHandler} und \klasse{MacroLineBreakpointHandler}, außer zur Korrektur der Zeilennummern werden diese genutzt, um die korrekten Methoden an der Klasse \klasse{BreakpointManager} aufzurufen.
\item[i18n] enthält die Klasse \klasse{GuiText}, die die Textkonstanten für die \mdg bereitstellt.
\item[listener] enthält verschiedene \emph{Listener}, die in der \mdg genutzt werden.
\item[settings] enthält Einstellungs-Enumerationen, die die Konfigurationsoptionen für die \mdg bereit stellen. Zusätzlich zu den aus dem \md bekannten Klassen \klasse{IntegerSettings} und \klasse{InternalSettings} gibt es nun auch eine Klasse \klasse{KeyStrokes}, die Tastenkombinationen aus der Konfiguration ausliest.
\end{description}

Im Gegensatz zum \md gibt es bei der \mdg keine wichtige Klasse, hier ist vielmehr das Zusammenspiel der verschiedenen Klassen wichtig zu verstehen. Am komplexesten ist dabei wahrscheinlich die Ausführung auf den zwei Threads, dem \gls{edt} und dem \emph{WorkerThread}. Aber auch die Darstellung des (Mikro-)Assembler-Codes und das Zusammenspiel zwischen Zeilennummern, Breakpoints, Zeilenhervorhebung und Code ist etwas komplexer.\clearpage

\chapter{Zusammenfassung}

Ich habe in der vorliegenden Arbeit einen Einblick in den \md und die \mdg gegeben. Das Ziel einen Debugger für die \mic zu entwerfen und implementieren konnte ich erreichen; in \secref{tutorial-konsole} habe ich ein mögliches Szenario gezeigt, wie der \md eingesetzt werden kann.

Auch die \gls{gui} für den \md -- die \mdg -- ist funktionsfähig und performant ausführbar. Einige Funktionen aus dem \md sind in der \mdg nicht vorhanden, da diese nicht mehr benötigt werden, wie beispielsweise das Anzeigen des \ma oder \ac. Andere Funktionen fehlen noch gänzlich, wie beispielsweise das setzen eines Wertes im Hauptspeicher oder das Verändern der Werte von Registern. Dafür bietet die \mdg allerdings einige Vorteile, zunächst bietet sie eine gute Übersicht -- jederzeit ist der Code, die Register und der Hauptspeicher zu sehen. Aber auch ein funktionaler Vorteil ist nennenswert: die Möglichkeit den laufenden Prozessor zu unterbrechen.

Die beiden Aufgabenteile wurden als eigenständige Projekte entwickelt. Veränderungen an der \gls{gui} haben demnach keinen Einfluss auf den \md. Zwei wichtige Ziele waren die Erweiterbarkeit und Lesbarkeit des Codes, damit andere Personen sich einarbeiten können und den Debugger weiterentwickeln können. Ich habe gezeigt, dass auch dieses Ziel erreicht wurde, unter anderem durch einen hohen Testaufwand: die Codeabdeckung der \mdg liegt über $93\%$\footnote{Gemessen mit \name{sonar} (siehe \cite{Sonar2012}) am 05. Juni 2012.}, des \md sogar über $97\%$\footnote{Gemessen mit \name{sonar} (siehe \cite{Sonar2012}) am 02. Juni 2012.}.

Der Umfang der entwickelten Funktionen erfüllt nicht alle Anforderungen, die während der Entwicklung entstanden. Der Debugger ist aber nun eine gute Basis, um eben solche Funktionalitäten nach und nach zu entwickeln. Diese Arbeit und die Dokumentation im Code sollte ausreichen, dass auch Außenstehende künftig solche Funktionalitäten implementieren.

Ich hoffe nun auf viele Nutzer und Rückmeldungen, was am Debugger verbessert werden kann. Oder welche Funktionalität gut wäre, wenn sie der Debugger hätte.\clearpage

\appendix
\pagenumbering{Roman}

\bibliography{bericht}\clearpage

\chapter*{Erklärung}\addcontentsline{toc}{chapter}{Erklärung}
\markright{Erklärung}
\begin{flushleft}
gemäß \S 5 (2) der "`Studien- und Prüfungsordnung DHBW Technik"' vom 18. Mai 2009.\\

Ich versichere hiermit, dass ich die vorliegende Arbeit mit dem Thema\\
\vspace{10mm}\thema\\\vspace{10mm}
selbstständig verfasst und keine anderen als die angegebenen Quellen und Hilfsmittel verwendet habe.\\
\vspace{30mm}
\begin{tabular}{lp{15mm}p{70mm}}
	Stuttgart, \today && \\\cline{1-1}\cline{3-3}
	{\footnotesize Ort, Datum} && {\footnotesize Christian Rösch} \\
\end{tabular}
\end{flushleft}
\clearpage

\end{document}
