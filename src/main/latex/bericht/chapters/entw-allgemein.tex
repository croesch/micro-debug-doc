\chapter{Automatisierte Tests}
\chplbl{entw-allg}
Nachdem Du nun mit den Werkzeugen vertraut bist, die für das Entwickeln am \md und der \mdg gebraucht werden, möchte ich einen wichtigen Aspekt ansprechen -- automatisierte Tests in dieser Arbeit.

Bei der Implementierung dieser Arbeit sollte besonderen Wert auf die Erweiterbarkeit gelegt werden. Andere Studenten sollen sich in den Code einarbeiten, zusätzliche Funktionen implementieren und bestehende Funktionen abändern können.

Obwohl ich in meinem Studium gelernt habe, wie man ein Softwareprojekt plant und dadurch die optimale Struktur erhält, ist es mir bei der Entwicklung dieser Arbeit nicht gelungen vorher die optimale Struktur zu entwerfen. Immer wieder bin ich auf Probleme gestoßen, die es erforderten, den vorhandenen Code zu refaktorieren -- die Struktur anzupassen:

\begin{quote}
  ``\textbf{Refactoring} (noun): a change made to the internal structure of software to make it easier to understand and cheaper to modify without changing its observable behavior.'' --- \cite[S.~53]{Fowler1999}
\end{quote}

Ich gehe davon aus, dass Refaktorierung am \md und der \mdg auch in Zukunft eine wichtige Rolle spielen wird; besonders, wenn weitere Student mitentwickeln. Martin~Fowler schreibt an anderer Stelle nochmal deutlich, dass die Refaktorierung die Qualität des Codes nicht durch das Ändern des Verhaltens verbessert:

\begin{quote}
  ``So, we want programs that are easy to read, that have all logic specified in one and only one place, that do not allow changes to endanger existing behavior, and that allow conditional logic to be expressed as simply as possible.

  Refactoring is the process of taking a running program and adding to its value, not by changing its behavior but by giving it more of these qualities that enable us to continue developing at speed.'' --- \cite[S.~60]{Fowler1999}
\end{quote}

Wie stellen wir sicher, dass wir durch Refaktorierung das Verhalten des Codes nicht manipulieren? Wir schreiben automatisierte Tests: Ausführlichen automatisierte Tests geben uns die nötige Sicherheit für die Refaktorierung, die wiederum die Lesbarkeit und Erweiterbarkeit des Codes steigert -- unsere zwei wichtigsten Ziele in dieser Arbeit.

Für den \md und die \mdg habe ich für nahezu jede Klasse automatisierte Tests geschrieben, die Du im Verzeichnis \texttt{src/test/java/} findest. Jede Testklasse liegt im selben \package wie die zu testende Klasse und trägt den selben Namen, allerdings um das Suffix \emph{Test} erweitert.

Die Tests sind nicht nur indirekt über die Refaktorierung eine gute Möglichkeit, den Code besser zu verstehen: Möchte man das Verhalten oder die Funktion einer Klasse verstehen, sind die automatisierten Tests der entsprechenden Testklasse ein guter Anlaufpunkt. Denn in jedem Test werden für verschiedene Eingaben die erwarteten Ausgaben beschrieben und dadurch gezeigt, wie sich die Klasse verhalten soll.

Dazu möchte ich ein kleines Beispiel geben: \lstref{integerparser-code} zeigt den Code, den wir versuchen wollen zu verstehen. Die beiden Testmethoden sind in \lstref{integerparser-test1} und \lstref{integerparser-test2} zu sehen. Ohne die JavaDoc\notiz{JavaDoc!?}-Kommentare, die hier nicht aufgeführt sind, ist es nicht ganz einfach den Code aus \lstref{integerparser-code} zu verstehen.

\begin{lstlisting}[caption={\klasse{IntegerParser} -- relevante Methoden},label=\lstlbl{integerparser-code}]
  @Nullable
  public Integer parse(final String toParse) {
    if (toParse == null) {
      return null;
    }

    try {
      // try parsing and if this works, it was a valid number
      return Integer.valueOf(toParse);
    } catch (final NumberFormatException nfe) {
      // number might be 0x.. or .._.., so split the number in radix and the number
      // convert aliases (0x,0b,..) to the correct notation
      final String[] num = convertAliases(toParse).split("_");
      if (num.length != 2) {
        // not a valid number - no idea what to do
        return null;
      }

      // number is a special number
      try {
        // try to parse the number with the specified radix
        return Integer.valueOf(num[0], Integer.parseInt(num[1]));
      } catch (final NumberFormatException nfe2) {
        // didn't work -> invalid number
        return null;
      }
    }
  }

  @NotNull
  private String convertAliases(final String num) {
    if (num.length() > 1 && num.charAt(0) == '0') {
      switch (Character.toLowerCase(num.charAt(1))) {
        case 'b': return num.substring(2) + "_2";
        case 'o': return num.substring(2) + "_8";
        case 'x': return num.substring(2) + "_16";
        default:  return num;
      }
    }
    return num;
  }
\end{lstlisting}

Wenn wir uns aber den Testcode aus \lstref{integerparser-test1} anschauen, dann erhalten wir schnell eine Ahnung, was die Klasse macht. Wir können unsere Vermutung nun an \lstref{integerparser-test2} noch überprüfen. Handelt es sich hier um die Funktionalität, die in \secref{zahlenformat} vorgestellt wurde?

\begin{lstlisting}[caption={\klasse{IntegerParser} -- Test auf Validität},label=\lstlbl{integerparser-test1}]
  @Test
  public void testNumber_Valid() {
    IntegerParser parser = new IntegerParser();
    assertThat(parser.parse("12")).isEqualTo(12);
    assertThat(parser.parse("0x12")).isEqualTo(0x12);
    assertThat(parser.parse("12_16")).isEqualTo(0x12);
    assertThat(parser.parse("1234_10")).isEqualTo(1234);
    assertThat(parser.parse("1010_2")).isEqualTo(10);
    assertThat(parser.parse("0b1010")).isEqualTo(10);
    assertThat(parser.parse("11010_002")).isEqualTo(26);
    assertThat(parser.parse("1g_17")).isEqualTo(33);
    assertThat(parser.parse("-1g_17")).isEqualTo(-33);
    assertThat(parser.parse("-1G_17")).isEqualTo(-33);
    assertThat(parser.parse("-011")).isEqualTo(-11);
    assertThat(parser.parse("-0000")).isEqualTo(0);
    assertThat(parser.parse("0o0")).isEqualTo(0);
    assertThat(parser.parse("0o27")).isEqualTo(23);
    assertThat(parser.parse("27_8")).isEqualTo(23);
    assertThat(parser.parse("0B0")).isEqualTo(0);
    assertThat(parser.parse("0O0")).isEqualTo(0);
    assertThat(parser.parse("0X0")).isEqualTo(0);
  }
\end{lstlisting}

Wie wir sehen, werden der zu testenden Klasse Zeichenketten übergeben und geprüft, ob das Ergebnis die erwartete Zahl ist. Unsere Vermutung war also korrekt. Wir sehen auch, welche Werte die Komponente zurückgibt, wenn wir ungültige Zeichenketten übergeben: \texttt{null}. Dadurch haben wir also erstens eine gute Vorstellung davon, was die Klasse macht und zweitens durch den automatisierten Test gesichert, dass wir bei künftigen Refaktorierungen oder sonstigen Änderungen das Verhalten nicht manipulieren.

\begin{lstlisting}[caption={\klasse{IntegerParser} -- Test auf Invalidität},label=\lstlbl{integerparser-test2}]
  @Test
  public void testNumber_Invalid() {
    IntegerParser parser = new IntegerParser();
    assertThat(parser.parse("")).isNull();
    assertThat(parser.parse(" ")).isNull();
    assertThat(parser.parse("\t")).isNull();
    assertThat(parser.parse("A")).isNull();
    assertThat(parser.parse("1010_a")).isNull();
    assertThat(parser.parse("2010_2")).isNull();
    assertThat(parser.parse("a010_2")).isNull();
  }
\end{lstlisting}

Zusammenfassend lässt sich sagen, dass für mich die ausführlichen automatisierten Tests das Erreichen der beiden wichtigsten Ziele sichern: leichte Erweiterbarkeit und gute Lesbarkeit des Codes. Sicherlich habe ich in das Schreiben ausführlicher Tests viel Zeit investiert, wodurch ich weniger Zeit für die Entwicklung von Programmcode hatte. Ich denke aber, dass durch diese anfängliche Zeiteinbuße und bei kontinuierlicher Pflege der Tests dauerhaft viel Zeit gespart werden kann: in der Praxis habe ich schon oft erfahren, wie viel Zeit verloren gehen kann, wenn ungenügend getestet wird.