\chapter{Werkzeuge}
\chplbl{werkzeuge}
In diesem Teil der Arbeit möchte ich besonders eine Frage beantworten: Wie sind die bisher vorgestellten Funktionen des Debuggers implementiert? Mein Ziel ist, die Implementierung und Struktur des Debuggers so nachvollziehbar zu beschreiben, dass Leser mit Programmiererfahrung in der Lage sind, den Debugger weiterzuentwickeln.

Der Debugger soll auch nach Ende dieser Arbeit weiterentwickelt werden können. Daher stelle ich in diesem Kapitel vor, welche Werkzeuge ich zur Entwicklung des Debuggers nutze und wie diese zur Mitarbeit genutzt werden können. Dies ist nicht als Bedienungsanleitung zu verstehen! Ich bin aber der Ansicht, dass es das Verständnis für die Implementierung und meine Arbeit steigert, wenn geklärt ist, welche Werkzeuge ich wie nutze.

\section{Versionskontrollsystem}
Die Projekte \md und \mdg sind bei \name{GitHub} (siehe \cite{Roesch2012,Roesch2012gui}) gehostet und sind somit mit \gls{git} versioniert.

\gls{git} wird beispielsweise in \cite{Ohne1} beschrieben und ist demnach:
\begin{description}
\item[effizient] Auch bei großen Projekten zeigen Vergleiche, dass \gls{git} insgesamt schneller ist, als beispielsweise \gls{svn}.
\item[verteilt] Jeder Entwickler erhält ein lokales Repository und kann damit arbeiten. Es wird kein zentraler Server benötigt, alle Funktionen des Systems und alle Versionen stehen lokal zur Verfügung.
\item[sicher] Bei \gls{git} wird jeder Commit gehasht. Es ist praktisch nicht möglich einen Commit zu manipulieren, da dies die Hash-Summe verändern würde. Auch wenn theoretisch Kollisionen der Hash-Summe möglich sind, ist bisher eine solche Kollision noch nicht konstruierbar -- bei einem Code-Projekt müsste eine solche Kollision zudem kompilierbarer Code sein, was als hinreichend unwahrscheinlich gilt.
\end{description}

Ich habe \gls{git} gewählt, da dadurch später leicht \emph{forks}\footnote{Zu deutsch Abspaltung, ist ein Entwicklungszweig, nachdem sich ein Software-Projekt in zwei oder mehr Teile geteilt hat. Diese Teile werden meist unabhängig voneinander weiterentwickelt. Im täglichen Arbeitsablauf mit \gls{git} ist ein fork aber auch die Möglichkeit Neuerungen für ein Projekt zu Entwickeln und den entstandenen Entwicklungszeig nach Fertigstellung der Neuerung in das Ursprungsprojekt einzupflegen.} gebildet werden können und der Debugger durch Außenstehende weiterentwickelt werden kann. Das dezentrale Entwickeln mit \gls{git} ermöglicht es zusätzlich, dass später viele Entwickler am Debugger mitentwickeln.

Wie checkt man mit \gls{git} den Quelltext der Arbeit aus? Wie unterstützt mich \gls{git}, wenn ich das Projekt weiterentwickeln möchte? Im Folgenden möchte ich auf diese Fragen antworten. Ich möchte keine umfassende Bedienungsanleitung für \gls{git} geben, sondern die Grundlagen für die alltägliche Arbeit mit \gls{git} vermitteln. \gls{git} am \md erklären, für \mdg ist das Vorgehen analog, in den meisten Fällen muss lediglich ein \texttt{-gui} eingefügt werden.

\subsection{Repository klonen und Code auschecken}
\seclbl{git-checkout}
Die Adresse des öffentlichen Repositorys lautet: \code{https://github.com/croesch/micro-debug.git}

\begin{lstlisting}[language=sh,caption={\md mit git klonen},label=\lstlbl{git-klonen}]
git clone git://github.com/croesch/micro-debug.git
cd micro-debug(*@\srclbl{git-klonen-cd}@*)
\end{lstlisting}

Da \gls{git} ein verteiltes Versionskontrollsystem ist, muss das entfernte Repository zunächst lokal angelegt werden -- das wird klonen genannt. Auschecken bezeichnet den Vorgang, den Inhalt aus dem lokalen Repository im Arbeitsverzeichnis zu realisieren, wo er bearbeitet werden kann.

Klont man ein entferntes Repository, beispielsweise das Repository des \md, wird im aktuellen Verzeichnis ein Verzeichnis \texttt{micro-debug} angelegt. In diesem Verzeichnis wird standardmäßig die Kopie des entfernten Repositorys abgelegt -- im Verzeichnis \texttt{micro-debug/.git}. Außerdem wird der Inhalt des Repositorys ausgecheckt und befindet sich anschließend im Verzeichnis \texttt{micro-debug}. \lstref{git-klonen} zeigt, welcher Befehl zum Klonen (und implizit Auschecken) des \md-Repositorys verwendet wird -- in \srcref{git-klonen-cd} wird in das erzeugte Verzeichnis gewechselt, das Arbeitsverzeichnis.

Im Gegensatz zu \gls{svn} erstellt \gls{git} nur im Wurzelverzeichnis eines Projekts ein Verzeichnis \texttt{.git}. Direkt nach dem Klonen eines anderen Repositorys enthält es beispielsweise einen Verweis (\emph{origin}) auf dieses Repository. Sobald das geklonte Repository neuere Commits enthält, kann dieser Verweis genutzt werden, um wie in \lstref{git-pull} die neuen Commits in das lokale Repository zu übernehmen.

\begin{lstlisting}[language=sh,caption={Mit git \emph{pull} auf Originalrepository ausführen},label=\lstlbl{git-pull}]
git pull origin
\end{lstlisting}

Mit dem Befehl \texttt{pull} werden die neuen Commits heruntergeladen, in das lokale Repository übernommen und ausgecheckt.

Dieses Vorgehen ist aufgrund der Änderung des aktuellen Arbeitsverzeichnisses häufig unerwünscht, daher kann man den Befehl aus \lstref{git-pull} wie in \lstref{git-fetch-merge} in zwei Befehle aufteilen, um zu regeln, ob und welcher Branch die Neuerungen erhält.

\begin{lstlisting}[language=sh,caption={\emph{pull} in zwei Befehlen manuell ausführen},label=\lstlbl{git-fetch-merge}]
git fetch origin(*@\srclbl{git-fetch-fetch}@*)
git merge origin/master(*@\srclbl{git-fetch-merge}@*)
\end{lstlisting}

Mit \emph{fetch} in \srcref{git-fetch-fetch} werden die Aktualisierungen heruntergeladen und zunächst im lokalen Repository gespeichert. Erst der Befehl \emph{merge} in \srcref{git-fetch-merge} verändert die Dateien im Arbeitsverzeichnis.

Es gibt bei \gls{git} auch einen expliziten \texttt{checkout}-Befehl, der verwendet wird, um beispielsweise zwischen verschiedenen Branches zu wechseln.

\subsection{Code beitragen}
Hat man ein Repository geklont und möchte Codeänderungen dem Originalrepository zuführen, gibt es zwei Möglichkeiten: Man benutzt den Standard-Branch oder so genannte Feature-Branches.

\begin{lstlisting}[language=sh,caption={Eine Datei mit git committen},label=\lstlbl{git-commit}]
git add relativer-pfad-zur-datei
git commit -m "Commit-Nachricht"
\end{lstlisting}

Den Standard-Branch (\texttt{master}) zu nutzen, empfehle ich nur, wenn man alleine an einem Projekt entwickelt. \lstref{git-commit} zeigt die Befehle, um eine veränderte Datei zu commiten. Dieser Commit wirkt sich auf den aktuellen Branch aus und kann mit dem \texttt{push}-Befehl in das geklonte Repository transferiert werden. Das entfernte Repository muss dazu den gleichen Stand haben, wie das lokale Repository ohne den zu veröffentlichenden Commit.

Da dies insbesondere bei mehrköpfigen Entwicklungsteams eine Herausforderung ist, empfehle ich Feature-Branches zu nutzen. Durch folgende Schritte kann jeder Leser selbst Code zum \md beitragen und diesen veröffentlichen.
\begin{enumerate}
\item Unter \code{https://github.com/croesch/micro-debug/issues} sind die Bugs und offenen Aufgaben zu finden. Möchte jemand Code beitragen, vermerkt er dies an dem jeweiligen Thema, oder öffnet ein neues, falls kein passendes Thema gefunden werden kann.

\item Im lokalen Repository wird nun ein so genannter Feature-Branch angelegt, in dem der Code geändert werden wird. \lstref{git-branch-erstellen} enthält den Befehl, mit dem ein neuer Branch im lokalen Repository angelegt und ausgecheckt wird -- eine Konvention ist, dass die Nummer des Branches der Nummer des Themas entspricht.

\begin{lstlisting}[language=sh,caption={Einen Branch mit git erstellen},label=\lstlbl{git-branch-erstellen}]
git checkout -b 100-themen-titel
\end{lstlisting}

\item Der entsprechende Code kann nun geändert und wie in \lstref{git-commit} committet werden -- die Commits betreffen den neu erstellten Branch.

\item Regelmäßig oder zumindest am Ende der Änderungen sollte der aktuelle Stand des Originalrepositorys geladen werden. Dazu empfehle ich den Branch \texttt{master} auszuchecken und die aktuellen Commits des Originalrepositorys dort einzupflegen -- wie in \lstref{git-master-aktualisieren} zu sehen.

\begin{lstlisting}[language=sh,caption={Mit git den Branch \texttt{master} auschecken und aktualisieren},label=\lstlbl{git-master-aktualisieren}]
git checkout master
git pull origin master
\end{lstlisting}

\item Nun sollte der neu erstellte Branch so verändert werden, dass seine Commits auf dem aktuellen Stand des Repositorys aufsetzen. Dies kann mit dem Befehl \texttt{rebase} erledigt werden, wie in \lstref{git-neubranch-rebase} beschrieben. Das auschecken des neuen Branches ist nötig, da \texttt{rebase} den aktuellen Branch so verändert, dass er auf dem angegebenen Branch aufsetzt; in diesem Fall \texttt{master}.

\begin{lstlisting}[language=sh,caption={Mit git ein \texttt{rebase} auf einen Branch ausführen},label=\lstlbl{git-neubranch-rebase}]
git checkout 100-themen-titel
git rebase master
\end{lstlisting}

\item Zum Schluss ist das lokale Repository in einem Zustand, der veröffentlicht werden kann. \lstref{git-push} zeigt, wie nun der neu erstellte Branch an das Originalrepository gesendet werden kann.
\begin{lstlisting}[language=sh,caption={Mit git einen lokalen Branch an das Originalrepository senden},label=\lstlbl{git-push}]
git push origin 100-themen-titel
\end{lstlisting}
\end{enumerate}

Die Entwickler des Originalrepositorys können den neuen Branch nun einsehen, überprüfen und in den Branch \texttt{master} einpflegen. Dadurch ist er offiziell Teil des Produkts geworden -- diese Vorgehensweise ermöglicht sehr verteiltes Arbeiten bei gleichzeitig zentraler Verwaltung und Entscheidung, welche Änderungen tatsächlich dem Projekt zugeführt werden.

Bei dem Arbeitsfluss, den ich gerade beschrieben habe, ist das lokale Repository ein Klon des Originalrepositorys. In der Regel hat aber nur ein oder wenige Entwickler Schreibrechte auf dem Originalrepository, das alle lesen können. Daher ist in der Praxis noch ein zusätzlicher Schritt nötig, den ich im Folgenden vorstellen möchte.

\subsection{Unterstützung durch GitHub}
Die Idee ist, dass jeder Entwickler sein eigenen öffentlichen Klon des Originalrepositorys besitzt. Dieser öffentliche Klon wird von dem jeweiligen Entwickler lokal geklont, und daran gearbeitet, wie oben beschrieben. Zur Aktualisierung des lokalen Repositorys muss zusätzlich das öffentliche Repository des Entwicklers aktualisiert werden und nach Fertigstellung der Arbeit kann das Ergebnis per \texttt{push}-Befehl in dem Repository des Entwicklers veröffentlicht werden. Der oder die Entwickler des Originalrepository können die veröffentlichten Änderungen dann in das Originalprojekt einfließen lassen.

Für diesen Arbeitsablauf ist \name{GitHub} sehr gut geeignet -- daher möchte ich diesen Arbeitsablauf am Beispiel von \name{GitHub} verdeutlichen.

\begin{enumerate}
\item Bei \name{GitHub} ist eine Registrierung notwendig, um Repositorys veröffentlichen zu können.

\item Ist man nun angemeldet, erscheint auf der Projektseite \cite{Roesch2012} der \emph{fork}-Button. Dadurch klont \name{GitHub} das Projekt für den Benutzer, wo es nun unter \code{https://github.com/benutzername/micro-debug} öffentlich verfügbar ist.

\item Anschließend kann wie in \lstref{git-benutzer-klonen} das öffentliche Repository geklont werden. Der Unterschied zu \lstref{git-klonen} ist die Adresse des geklonten Repositorys und das Protokoll -- das \gls{git}-Protokoll erlaubt Schreib- und Leseoperationen, bei entsprechenden Rechten auf dem Server.

\begin{lstlisting}[language=sh,caption={\md mit git vom Benutzerrepository klonen},label=\lstlbl{git-benutzer-klonen}]
git clone git://github.com/ihr-name/micro-debug.git
cd micro-debug
\end{lstlisting}

\item Um später Aktualisierungen des Originalrepository zu erhalten, sollte dieses noch im lokalen Repository als \texttt{remote} hinzugefügt werden, wie in \lstref{git-add-original-remote} zu sehen.

\begin{lstlisting}[language=sh,caption={Originalrepository des \md dem lokalen Repository als \texttt{remote} hinzufügen},label=\lstlbl{git-add-original-remote}]
git remote add original-projekt git://github.com/croesch/micro-debug.git
\end{lstlisting}

\item Im lokalen Repository wird nun ein Feature-Branch angelegt, wie in \lstref{git-branch-erstellen} beschrieben.

\item Der entsprechende Code kann nun geändert und wie in \lstref{git-commit} committet werden -- die Commits betreffen den neu erstellten Branch.

\item Da sich das öffentliche Repository des Benutzers nicht ändert, können Aktualisierungen nicht wie in \lstref{git-master-aktualisieren} beschrieben in das lokale Repository geholt werden. Daher benötigt das lokale Repository die Referenz auf das Originalrepository, wie in \lstref{git-add-original-remote} erstellt.

\begin{lstlisting}[language=sh,caption={Mit git den Branch \texttt{master} auschecken und aktualisieren},label=\lstlbl{git-master-original-aktualisieren}]
git checkout master
git pull original-projekt master
\end{lstlisting}

\lstref{git-master-original-aktualisieren} enthält die Befehle, um das lokale Repository mit den Informationen aus dem Originalrepository zu aktualisieren.

\item Abschließend sollte der neue Branch wie in \lstref{git-neubranch-rebase} beschrieben neu auf den \texttt{master}-Branch aufgesetzt werden und dann wie in \lstref{git-push} gezeigt, in dem öffentlichen Repository des Benutzers veröffentlicht werden.

\item \name{GitHub} bietet nun die Möglichkeit die Entwickler des Originalprojekts über den neuen Branch im öffentlichen Repository des Entwickelrs zu informieren. Dazu befindet sich auf der Projektseite des Benutzers ein Button \emph{Pull-Request}. Wenn die Entwickler des Originalrepositorys entscheiden die Änderungen in das Projekt aufzunehmen, sind diese Informationen später über die in \lstref{git-master-original-aktualisieren} beschriebene Aktualisierung im Branch \texttt{master} zu sehen.
\end{enumerate}

Meine Ausführungen über \gls{git} sollten es ermöglichen, den Code des \md auszuchecken, zu ändern und die Änderungen zu veröffentlichen. Im Folgenden möchte ich nun das Build-Management-Werkzeug erläutern, dass ich zum Entwickeln nutze: \gls{mvn}.

\section{Build-Management}
\gls{mvn} ist ein Werkzeug zum Bauen von \gls{java}-Anwendungen. Meines Erachtens gibt es keine signifikanten Gründe für oder gegen die Nutzung von \gls{mvn}; stattdessen hätte ich auch \gls{ant} oder \gls{make} nutzen können.

In diesem Abschnitt möchte ich die grundlegenden Befehle für die Arbeit mit \gls{mvn} erläutern.

Ich verwende \gls{mvn}, da hier viel implizites Wissen eingesetzt wird; dadurch wird der Konfigurationsaufwand geringer -- solange man den Konventionen der \gls{mvn}-Entwickler folgt. Zu dem impliziten Wissen gehört unter anderem:
\begin{itemize}
\item \gls{java}-Klassen befinden sich in \code{src/main/java/}
\item \gls{java} Test-Klassen befinden sich in \code{src/test/java/}
\item Ressourcen befinden sich in \code{src/main/resources/}
\item Ressourcen, die nur für die Tests benötigt werden, befinden sich in \code{src/test/resources/}
\item Kompilate und von \gls{mvn} generierte Klassen befinden sich in \code{target/}
\item Die explizite Konfiguration steht in der Datei \code{pom.xml}
\end{itemize}

Die Datei \texttt{pom.xml} enthält aus \gls{mvn}-Sicht alle nötigen Informationen über das Projekt: Die Version, der Name und benötigte Bibliotheken sind einige dieser Informationen.

Das wohl Wichtigste an einer Build-Management-Software ist, den Code kompilieren zu können. \lstref{maven-compile} zeigt den Befehl, mit dem der Code eines \gls{mvn}-Projekts wie dem \md kompiliert werden kann. Testklassen werden dadurch nicht kompiliert und auch nicht ausgeführt.

\begin{lstlisting}[language=sh,caption={Den Code eines Maven-Projekts kompilieren},label=\lstlbl{maven-compile}]
mvn compile
\end{lstlisting}

\gls{mvn} liest die Abhängigkeiten für ein Projekt aus der \texttt{pom.xml} Datei und lädt die benötigten Bibliotheken selbstständig herunter. Auch für die Ausführung von \gls{mvn} werden gewisse Bibliotheken, dies führt bei der ersten Ausführung von \gls{mvn} dazu, dass zunächst einige Bibliotheken heruntergeladen werden.

Bei der Ausführung von \gls{mvn} gibt es verschiedene Ziele, die aufeinander aufbauen. So ist beispielsweise für das Packetieren eines Projekts notwendig zunächst alle Klassen zu kompilieren und die Tests auszuführen, bevor das Projekt packetiert wird. Das Projekt wird mit dem Befehl aus \lstref{maven-package} packetiert und im Verzeichnis \texttt{target/} abgelegt.

\begin{lstlisting}[language=sh,caption={Den Code eines Maven-Projekts packetieren},label=\lstlbl{maven-package}]
mvn package
\end{lstlisting}

Wenn man Änderungen am Code vornimmt, ist es besonders hilfreich zu wissen, ob die automatisierten Tests fehlschlagen. Möchte man mit \gls{mvn} die Tests ausführen, kann dazu der in \lstref{maven-test} gezeigte Befehl genutzt werden.

\begin{lstlisting}[language=sh,caption={Die automatisierten Tests eines Maven-Projekts ausführen},label=\lstlbl{maven-test}]
mvn test
\end{lstlisting}

Die in der Datei \code{pom.xml} definierten abhängigen Projekte sind in \gls{xml} definiert, wie beispielsweise in \lstref{maven-dependency} am Beispiel von \gls{fest} zu sehen, das vom Debugger genutzte \gls{gui}-Test-Framework.

\begin{lstlisting}[language=sh,caption={Konfiguration eines abhängigen Projekts in Maven},label=\lstlbl{maven-dependency}]
<dependency>
    <groupId>org.easytesting</groupId>
    <artifactId>fest-swing</artifactId>
    <version>1.2.1</version>(*@\srclbl{maven-dependency-version}@*)
    <scope>test</scope>
</dependency>
\end{lstlisting}

Dort kann nun die Version von \gls{fest} geändert werden, die der Debugger nutzen soll.

Die hier gezeigten Befehle sind keine umfassende Beschreibung von \gls{mvn}, genügen aber, um den Debugger zu bauen. Für das Kompilieren können auch andere Werkzeuge verwendet werden, allerdings muss der Entwickler sich dann um die Bibliotheken selbst kümmern. Aber prinzipiell ist \gls{mvn} durch jedes beliebige andere Werkzeug ersetzbar.