\chapter*{Einleitung}
Mit dieser Arbeit stelle ich \md vor, einen Debugger für die \mic. Die \mic ist ein Prozessor und wird von \name{Tanenbaum} in \cite{Tanenbaum1998} vorgestellt -- sie ist ein Prozessor der \gls{cisc} Architektur.

Im Gegensatz zur \gls{risc} Architektur haben \gls{cisc}-Prozessoren einen komplexeren Befehlssatz. Bei der \mic wird dies durch einen \mac realisiert, der für die begrenzte Hardware einen komplexen Befehlssatz bereitstellt. Dieser \mac ermöglicht, Maschinenbefehle auf einem hohen Abstraktionsniveau zu definieren und ohne Veränderung der Hardware anzupassen.

\section*{Aufgabenstellung}
Im Skript der Vorlesung Rechnertechnik \cite[Kapitel 8]{Stroetmann2009} wird von \name{Stroetmann} ein Mikro-Assembler für die \mic vorgestellt. Ziel dieser Arbeit ist der Entwurf und die Implementierung eines Debuggers für diesen Mikro-Assembler. Der Debugger soll die Entwicklung von \ma und \ac für die \mic erleichtern.

Um entsprechenden \ma und \ac debuggen zu können, muss der Debugger die \mic simulieren können. In seiner Studienarbeit \cite{Kutzer2004} beschreibt \name{Kutzer} einen Simulator für die \mic. Der hier entwickelte Debugger soll diesen Simulator ersetzen und um Debug-Funktionen erweitern.

In dieser Arbeit sollen dabei hauptsächlich zwei Teilaufgaben bearbeitet werden.

\subsection*{Debugger für die Konsole}
Zunächst soll eine Version des Debuggers entwickelt werden, die über die Konsole bedient wird. Das Ziel ist die Implementierung des Debuggers unabhängig von der Darstellung der Ein- und Ausgabe.

Der Debugger soll in der Lage sein zwei Binärdateien einzulesen: eine Datei, die den \ma und eine, die den \ac enthält. Der \ac sollte grundsätzlich nur vom \mac abhängig und durch den Benutzer frei definierbar sein; als Referenz für diese Arbeit kann aber der \gls{ijvm}-Assembler dienen.

Der Benutzer muss Breakpoints (Haltepunkte) definieren können, die die Simulation des \mac unterbrechen. Der Debugger ermöglicht dem Benutzer dabei, Breakpoints für bestimmte \mais aber auch für bestimmte \ais zu definieren.

Neben den Werten der Register kann der Benutzer sich auch den Inhalt des Stacks, die gesetzten Breakpoints, den \ma und den \ac anzeigen lassen. Dem Benutzer soll es möglich sein, lokale Variablen, Register, den ausgeführten \ma sowie \ac zu beobachten.

\subsection*{GUI für den Debugger}
Im zweiten Teil der Aufgabe wird für den Debugger eine \gls{gui} entwickelt werden. Die \gls{gui} soll sowohl den \ma als auch den \ac anzeigen; Auch die Register mit ihren Werten und der Stack sollen sichtbar sein.

Die \gls{gui} ermöglicht dem Benutzer, Breakpoints zu setzen und zeigt an, welche \ma und \ai gerade abgearbeitet wird.

Da in dieser Arbeit ein allumfassender Debugger weder geschaffen werden kann noch soll, ist besonders darauf zu achten, die Wartung und Erweiterung des Debuggers außenstehenden Personen zu ermöglichen. Besonders der Code des Debuggers soll daher gut dokumentiert werden; es sollen sich andere Studenten in den Code einarbeiten und neue Funktionalitäten implementieren oder bestehende Funktionalitäten anpassen können.

\section*{Aufbau der Arbeit}
Die Arbeit ist in zwei Teile geteilt: Im ersten Teil erläutere ich die Bedienung und im zweiten Teil die Entwicklung des \md. Ich beschreibe also eher den derzeitigen Zustand des \md, als den Weg der Entstehung des \md. Denn diese Arbeit soll vor allem dazu dienen, dass Außenstehende sich in den \md einarbeiten können und den jetzigen Zustand nachvollziehen können.

Ich werde im Folgenden \md für die Konsolenvariante des Debuggers verwenden und \mdg für die \gls{gui}. Diese beiden Bezeichnungen sind gleichzeitig die Namen der Code-Projekte -- so möchte ich eine saubere Trennung zwischen den Projekten erreichen. Die Bezeichnung \emph{Debugger} werde ich verwenden, um allgemein über den \md und die \mdg zu sprechen; oder für Erklärungen, die beide Projekte betreffen.

\subsection*{Bedienung}
Im \chpref{allgemein} beantworte ich einige allgemeine Fragen über den Debugger: Welche Systemvoraussetzungen gibt es? Wie konfiguriert man den Debugger? Wie kann man den Debugger in andere Sprachen übersetzen? Produziert der Debugger Log-Ausgaben? Wenn ja, kann man diese konfigurieren?

Die Bedienung des \md beschreibe ich im \chpref{bed-konsole}; dort erkläre ich auch die Parameter des \md. Außerdem erläutere ich die verschiedenen Befehle und zeige ein Beispiel, wie mit dem \md ein Fehler im \ac gefunden werden kann.

Im \chpref{bed-gui} beschreibe ich die Bedienung der \mdg; dort nenne ich die Unterschiede zum \md und erkläre die verschiedenen Oberflächenelemente.% und zeige ein Beispiel, wie mit der \mdg ein Fehler gefunden werden kann.

\subsection*{Entwicklung}
In \chpref{werkzeuge} beschreibe ich einige Werkzeuge, die ich zur Entwicklung des Debuggers genutzt habe und deren Verständnis nötig ist, um selbst Änderungen am Projekt vorzunehmen. In \chpref{entw-allg} gehe ich kurz auf die Notwendigkeit der automatisierten Tests ein.

In den nachfolgenden Kapiteln beschreibe ich, welche Funktionalität des Debuggers sich in welchen Klassen befinedt. \chpref{prozessor} nutze ich, um zu beschreiben, wie die \mic simuliert wird und wie ich die verschiedenen Hardwarekomponten implementiert habe. Im \chpref{konsole} erkläre ich den Code des \md und in \chpref{gui} den Code der \mdg.