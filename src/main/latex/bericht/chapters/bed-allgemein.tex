\chapter{Allgemein}
\chplbl{allgemein}
Der Debugger -- sowohl der \md als auch die \mdg{} -- benötigt keine Installation; er kann als \texttt{.zip}-Archiv heruntergeladen werden und ist direkt nach dem Entpacken nutzbar. Die Datei heißt üblicherweise \texttt{micro-debug-version.zip}\footnote{Die Datei der \mdg heißt üblicherweise \texttt{micro-debug-gui-version.zip}.} und enthält mehrere Dateien:

\begin{description}
\item[micro-debug-version.jar] enthält den Code des \md und kann mit dem Befehl \texttt{java~-jar} ausgeführt werden. Diese Datei ist auch in der \mdg enthalten, dort gibt es aber zusätzlich noch die Datei \texttt{micro-debug-gui-version.jar}, die den Code der \mdg enthält.
\item[micro-debug.sh und micro-debug.bat] sind die Startskripte des \md für Windows und Linux\footnote{Auch hier ist die Namensgebung für die Startskripte der \mdg entsprechend: \texttt{micro-debug-gui.sh} und \texttt{micro-debug-gui.bat}}. Der Debugger kann durch Ausführen dieser Skripte gestartet werden -- der Benutzer muss sich dadurch nicht um die Konfiguration des Klassenpfades kümmern. Ich habe die Startskripte so geschrieben, dass sie von beliebigen Orten ausgeführt werden können: Der Benutzer kann den Debugger von vielen Arbeitsverzeichnissen aus aufrufen und hat die Konfigurationsdateien nur an einer Stelle zu pflegen.
\item[config/micro-debug.properties] ist die Konfigurationsdatei für den \md und die \mdg, hier kann beispielsweise die Größe des Hauptspeichers oder Tastenkombinationen konfiguriert werden -- siehe \secref{konfiguration}.
\item[config/logging.properties] ist die Konfigurationsdatei für das Logging des Debuggers -- hier kann der Benutzer definieren, ob und welche Ausgaben auf der Konsole oder in einer Datei erscheinen sollen -- siehe \secref{logs}.
\item[config/lang/] enthält die Textressourcen des Debuggers -- siehe \secref{lokalisierung}.
\item[lib/] enthält die verwendeten Bibliotheken des Debuggers; zur Zeit nur in der \mdg verwendet.
\end{description}

\section{Systemvoraussetzungen}
\seclbl{systemvoraussetzungen}
Der Debugger ist in \gls{java} geschrieben und daher prinzipiell an keine spezielle Plattform gebunden. Voraussetzung für die Nutzung des \md ist lediglich \gls{java}~5\notiz{Referenz?}.

In der Praxis differenzieren sich die verschiedenen Plattformen, auf denen \gls{java} verfügbar ist, in einigen Feinheiten. Daher ist es prinzipiell möglich, dass sich der Debugger auf gewissen Plattformen unerwünscht verhält. Wie anhand der Startskripte zu erkennen ist, habe ich den Debugger im Blick auf \emph{Linux} und \emph{Windows} entwickelt und auf diesen Plattformen\footnote{Ausführlich habe ich mit \emph{Windows~7} (in der 64-Bit-Variante) und \emph{Ubuntu~12.04} (auch 64-Bit) getestet.} getestet. Um den Debugger auf einer weiteren Plattform zu nutzen, muss zumindest das Startskript eigenständig geschrieben werden.

Der \md nutzt bislang einen Thread, die \mdg hingegen mindestens zwei (wie ich in \secref{verarbeitung-benutzeraktionen-threads} näher beschreibe), die allerdings selten parallel arbeiten. Für die Ausführungsgeschwindigkeit des Debuggers ist die Anzahl der verfügbaren Prozessorkerne daher irrelevant.

\section{Konfiguration}
\seclbl{konfiguration}
Die Konfiguration des Debuggers findet in \datei{properties}en statt. Die Konfigurationsdatei \texttt{config/micro-debug.properties} enthält die Konfigurationen sowohl für den \md als auch die \mdg. Wie in \datei{properties}en üblich, werden dort Schlüsselwert-Paare hinterlegt.

\begin{lstlisting}[language=sh,caption={Eintrag in \texttt{conf/micro-debug.properties}},label=\lstlbl{md-props-entry}]
# default value for the register CPP
mic1.register.cpp.defval = 0x4000
\end{lstlisting}

\lstref{md-props-entry} zeigt einen Eintrag aus der Konfigurationsdatei: Den Startwert für das Register \reg{CPP}. Diese Datei kann für jeden Schlüssel genau einen Wert enthalten. Ist für einen Schlüssel kein Wert konfiguriert oder ist ein ungültiger Wert konfiguriert, nutzt der Debugger den internen Standardwert für den jeweiligen Schlüssel.\notiz{Prüfen und ggf. einfügen, dass fallback anständig geloggt wird und das dann hier auch erwähnen.}

Dadurch ist es möglich ein Update des Debuggers durchzuführen, und weiterhin die alte Konfiguration zu nutzen. Ersetzt der Benutzer die \datei{jar} durch eine neuere Version, die zusätzliche Konfigurationsoptionen benötigt, nutzt der Debugger die Standardwerte dieser Konfigurationsoptionen.

Der Benutzer kann seine eigene Konfiguration dadurch von Version zu Version des Debuggers behalten; er muss geänderte Konfigurationsoptionen nicht mühsam nachpflegen.\notiz{Überlegen, ob nicht irgendwo ein Update-Abschnitt sinnvoll ist}

\subsection{Zahlenformat}
\seclbl{zahlenformat}
\notiz{Wahrscheinlich ist es sinnvoll in den properties Dateien einzutragen, von welchem Typ ein Schlüssel sein soll und dies dann hier auch zu notieren.}
In \lstref{md-props-entry} ist zu sehen, wie dem Register \reg{CPP} der Standardwert \texttt{0x4000} zugewiesen wird -- hier hexadezimal notiert. Andere Konfigurationseinträge sind dezimal eingetragen; welches Zahlenformat ist nun wo anzuwenden?

Für den Debugger ist es irrelevant, welches Zahlenformat bei welcher Konfigurationsoption verwendet wird. Er liest die gegebene Zahl und wandelt sie in einen \emph{Integer} um; dadurch kann der Benutzer für jeden Eintrag das passende Zahlenformat wählen. Nicht nur bei Konfigurationsoptionen, sondern bei allen eingegebenen Zahlen des Benutzers ist das Format variabel.

Welche Zahlenformate gibt es außer dezimal und hexadezimal? Der Debugger unterstützt alle Zahlensysteme mit der Grundzahl von $b=2$ bis $b=36$. Generell werden die Zahlen in der Art \texttt{ZAHL\_BASIS} angegeben -- \texttt{ZAHL} wird in dem jeweiligen Zahlensystem und \texttt{BASIS} im Dezimalsystem angegeben. In \tabref{verschiedene-zahlenformate} sind verschiedene Eingabemöglichkeiten für die Zahl $10$ gegeben.

\begin{table}[h]
  \centering
  \begin{tabular}[h]{|rl|}
    \hline
    \textbf{Zahlensystem} & \textbf{Darstellung}\\
    \hline
    Dualsystem        & \texttt{1010\_2}\\
    Ternärsystem      & \texttt{101\_3}\\
    Dezimalsystem     & \texttt{10\_10}\\
    Hexadezimalsystem & \texttt{A\_16}\\
    \hline
  \end{tabular}
  \caption{Auswahl möglicher Eingabearten der Zahl $10$ im Debugger}
  \tablbl{verschiedene-zahlenformate}
\end{table}


Für die meiner Meinung am häufigsten verwendeten Zahlensysteme gibt es die in \tabref{vereinfachungen-zahlenformate} dargestellten Vereinfachungen. Der Benutzer \emph{muss} diese Vereinfachungen aber nicht nutzen, er kann sowohl \texttt{A\_16} als auch \texttt{0xA} eingeben -- oder aber \texttt{10}.

\begin{table}[h]
  \centering
  \begin{tabular}[h]{|rll|}
    \hline
    \textbf{Zahlensystem} & \textbf{ausführlich} & \textbf{vereinfacht}\\
    \hline
    Dualsystem        & \texttt{1010\_2} & \texttt{0b1010} \\
    Oktalsystem       & \texttt{12\_8}   & \texttt{0o12}   \\
    Dezimalsystem     & \texttt{10\_10}  & \texttt{10}     \\
    Hexadezimalsystem & \texttt{A\_16}   & \texttt{0xA}    \\
    \hline
  \end{tabular}
  \caption{Vereinfachungen typischer Zahlenformate im Debugger am Beispiel der Zahl $10$}
  \tablbl{vereinfachungen-zahlenformate}
\end{table}

\subsection{Logs}
\seclbl{logs}
Sucht der Benutzer die Ursache für einen Fehler, den er oder der Debugger begangen hat, sind Log-Ausgaben ein guter Ansatzpunkt. Denn dort sind die vermeintlich wichtigen Vorgänge protokolliert und im Hinblick auf solche Analysen geschrieben.

Der Debugger wird im Startskript so gestartet, dass die Log-Konfiguration aus der Datei \texttt{conf/logging.properties} genutzt wird. Da die Syntax und Semantik der Konfigurationseinträge von \name{Oracle} \cite{Oracle2004} und \name{Vogel} \cite{Vogel2012} detailliert beschrieben ist, lasse ich sie hier aus.

Die Startskripte sind so konfiguriert, dass der Debugger alle Log-Ausgaben mit dem Log-Level \texttt{INFO} oder höher in Dateien schreibt. Diese Dateien liegen im \emph{home}-Verzeichnis des Benutzers und sind von der Form \texttt{micro-debugX.log} -- wobei \texttt{X} ein Laufindex ist. Es existieren neben dem genannten noch weitere Log-Level; je nach Situation kann der Debugger beispielsweise für ausführlichere Log-Ausgaben konfiguriert werden. Die verschiedenen Log-Level sind mit absteigender Wichtigkeit: \texttt{SEVERE}, \texttt{WARNING}, \texttt{INFO}, \texttt{CONFIG}, \texttt{FINE}, \texttt{FINER} und \texttt{FINEST}.

\subsection{ijvm.conf}
\seclbl{ijvm-conf}
Ich habe den Debugger so konstruiert, dass er prinzipiell für jeden beliebigen \ac genutzt werden kann. Allerdings habe ich den Debugger nur mit dem \gls{ijvm}-Assembler getestet, da der Debugger in der Praxis wohl vorwiegend mit dem \gls{ijvm}-Assembler genutzt wird.

Um den \ac disassemblieren zu können, gibt es eine Konfigurationsdatei -- die \texttt{ijvm.conf}. Damit der Debugger einen gewissen Standardsatz an \gls{ijvm}-Befehlen versteht, wird diese Datei in der \datei{jar} des Debuggers mit ausgeliefert.

Möchte der Benutzer die Datei \texttt{ijvm.conf} anpassen, genügt es, die Datei im Verzeichnis \texttt{conf/} abzulegen. Durch das Startskript wird sie dann auf den Klassenpfad geladen; vor der internen \texttt{ijvm.conf}-Datei in der \datei{jar} und ersetzt diese somit.

Die Datei \texttt{ijvm.conf} ist zeilenweise zu lesen, wobei eine Zeile nach folgendem Format aufgebaut ist:\notiz{Beispiel!?}

\setlength{\grammarparsep}{3pt plus 1pt minus 1pt}
\setlength{\grammarindent}{10em} % increase separation between LHS/RHS 
\begin{grammar}

<line>     ::= <comment>
  \alt         <address> <white>+ <identifier> <white>* <argumentlist> <comment>?

<comment>  ::= '//' <text>

<address> ::= '0x' <hexchar> <hexchar>

<hexchar> ::= '0' | '1' | '2' | '3' | '4' | '5' | '6' | '7' | '8' | '9' | 'A' | 'B' | 'C' | 'D' | 'E' | 'F'

<identifier> ::= <word>

<argumentlist> ::= ( <white> <argument> <white>* )*

<argument> ::= 'byte' | 'const' | 'index' | 'label' | 'offset' | 'varnum'

<white>    ::= ' ' | '$\backslash$t'
\end{grammar}

\section{Lokalisierung}
\seclbl{lokalisierung}
Der Debugger wurde in Englisch geschrieben -- weitere Sprachen können über das Verzeichnis \texttt{conf/lang/} hinzugefügt werden.

In diesem Verzeichnis liegen verschiedene \datei{xml}en, die dazu dienen, dem Benutzer Ausgaben in seiner Sprache anzuzeigen. Die Sprache, die der Benutzer angezeigt bekommt, hängt vom \emph{Locale} (siehe dazu ausführlich \cite{Oracle2010Loc}) der \gls{jvm} ab.

Der Debugger sucht für diese Sprache folgende vier Dateien, die Schlüsselwert-Paare für die Textkonstanten enthalten:
\begin{description}
\item[text_lang_CT_var1.xml] wobei \textbf{lang} die Sprache, \textbf{CT} die Länderkennung und \textbf{var1} die Variante ist, die vom \emph{Locale} gegeben sind. Diese Datei ist die spezifischste und wird sofern sie existiert zuerst geladen. Schlüssel, die hier nicht gefunden werden, werden in den folgenden Dateien gesucht.
\item[text_lang_CT.xml] in der Praxis existieren selten Informationen über die Variante des \emph{Locale}, daher kann \texttt{_var1} weggelassen werden.
\item[text_lang.xml] häufig existieren zwar verschiedene Länderkennungen für die gleiche Sprache, aber meist unterscheiden sich die Übersetzungsdateien für die Länderkennungen nicht. Daher werden in der Praxis viele Dateien in dieser Form angelegt werden.
\item[text.xml] die Basis-Datei für die Lokalisierung. Wenn ein Schlüssel in den anderen Dateien nicht gefunden wurde oder die anderen Dateien nicht existieren, dann wird er in dieser Datei gesucht.
\end{description}

Wie ich gerade angedeutet habe, ist der Aufbau dieser vier Dateien hierarschich: ein Schlüssel, der in der spezifischsten Datei gefunden wurde, wird aus den anderen Dateien nicht mehr ausgewertet.

\begin{lstlisting}[language=XML,caption={Beispiel für Datei \texttt{text_de_DE.xml}},label=\lstlbl{locale-file-de-de}]
<?xml version="1.0" encoding="UTF-8"?>
<!DOCTYPE properties SYSTEM "http://java.sun.com/dtd/properties.dtd">
<properties>
	<entry key="border">+*+*+*+*+*</entry>
	<entry key="border2">---</entry>
</properties>
\end{lstlisting}

Anhand eines Beispiels möchte ich dieses Verhalten erläutern: Gegeben seien die drei Dateien, die in \lstref{locale-file-de-de}, \lstref{locale-file-de} und \lstref{locale-file} aufgeführt sind. Die Datei der Form \texttt{text_lang_CT_var1.xml} ist in diesem Beispiel nicht vorhanden.

\begin{lstlisting}[language=XML,caption={Beispiel für Datei \texttt{text_de.xml}},label=\lstlbl{locale-file-de}]
<?xml version="1.0" encoding="UTF-8"?>
<!DOCTYPE properties SYSTEM "http://java.sun.com/dtd/properties.dtd">
<properties>
	<entry key="file-not-found">Datei nicht gefunden {0}</entry>
	<entry key="version">Micro-Debug - Version {0}</entry>
</properties>
\end{lstlisting}

\lstref{locale-result} zeigt das Ergebnis nach der Auswertung dieser drei Dateien am Beispiel des deutschen \emph{Locale}. Es ist zu sehen, wie die Einträge sich überschreiben -- ein Benutzer mit englischem Locale würde in diesem Beispiel den Eintrag für \texttt{border2} nicht erhalten und hätte dafür keinen definierten Wert.

\begin{lstlisting}[language=XML,caption={Beispiel für Datei \texttt{text.xml}},label=\lstlbl{locale-file}]
<?xml version="1.0" encoding="UTF-8"?>
<!DOCTYPE properties SYSTEM "http://java.sun.com/dtd/properties.dtd">
<properties>
	<entry key="version">Micro-Debug: Version {0}</entry>
	<entry key="file-not-found">file not found {0}</entry>
	<entry key="border">----------------------------------------</entry>
</properties>
\end{lstlisting}

Die Reihenfolge der Schlüsselwert-Paare ist willkürlich und hat für den Programmverlauf keine Auswirkung.

\begin{lstlisting}[language=XML,caption={Beispiel für Ergebnis nach Verarbeitung der Lokalisierungsdateien},label=\lstlbl{locale-result}]
<?xml version="1.0" encoding="UTF-8"?>
<!DOCTYPE properties SYSTEM "http://java.sun.com/dtd/properties.dtd">
<properties>
	<entry key="version">Micro-Debug - Version {0}</entry>
	<entry key="file-not-found">Datei nicht gefunden {0}</entry>
	<entry key="border">+*+*+*+*+*</entry>
	<entry key="border2">---</entry>
</properties>
\end{lstlisting}

Dieses System ermöglicht das einfache Hinzufügen neuer Sprachen. Dieser Vorgang ist zwar bisher nicht für den Benutzer vorgesehen, aber prinzipiell kann er in diesen Dateien Anpassungen vornehmen und sich dadurch die Ausgaben des Debuggers an seine Bedürfnisse anpassen. Auch ist denkbar, dass Übersetzungen des Debuggers von Dritten angeboten wird -- der Benutzer kann dann die Sprachdateien für seine Sprache herunterladen und in dem Verzeichnis \texttt{conf/lang/} ablegen.

Im Gegensatz zur Konfigurationsdatei gibt es bei den Lokalisierungsdateien keinen individuellen Fallback für jeden Schlüssel, sodass fehlende Werte mit einem Fehlertext belegt werden. Die Lokalisierungsdateien sind daher auch von einem Update des Debuggers betroffen -- mögliche Änderungen des Benutzers gehen verloren oder müssen in die neuen Dateien übernommen werden.

Einige Einträge in den Lokalisierungsdateien können Platzhalter enthalten. Platzhalter gelten nur für ein bestimmtes Schlüsselwert-Paar und sind von $0$ aufsteigend nummeriert in der Form \texttt{{0}} zu verwenden. Der Inhalt mit dem der jeweilige Platzhalter gefüllt wird kann in den Kommentaren in der Datei nachgelesen werden.

Zusätzlich zu den \texttt{text...xml}-Dateien gibt es bei der \mdg \texttt{text-gui...xml}-Dateien. Die Lokalisierungsdateien des \md und der \mdg sind also voneinander unabhängig.

Die Log-Ausgaben aus \secref{logs} sind immer auf Englisch und können über die \datei{xml}en nicht verändert werden.