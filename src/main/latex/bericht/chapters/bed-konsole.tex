\chapter{Interaktion per Konsole}
\chplbl{bed-konsole}
Wie in \chpref{allgemein} beschrieben, muss der \md nicht installiert werden: Vom Herunterladen bis zum Starten des \md genügen folgende Schritte.

\begin{enumerate}
\item Die Datei \texttt{micro-debug-version.zip} von der Projektseite \cite{Roesch2012} herunterladen
\item Die \datei{zip} in ein beliebiges Verzeichnis entpacken (bspw. \texttt{/opt/micro-debug/})
\item Das Verzeichnis des \md dem \texttt{PATH} hinzufügen
\item Den \md starten -- mit \texttt{micro-debug.sh~-{}-help}
\end{enumerate}

Der \md kann durch einige Parameter gesteuert werden und wird nach dem Start durch Befehle bedient. Ich werde daher zunächst die verschiedenen Parameter des \md und dann die Befehle zur Bedienung des \md erklären. Am Ende des Kapitels zeige ich in einem Tutorial, wie mit dem \md Fehler im \ac gefunden werden können.

\section{Parameter}
Der Standardaufruf für den \md ist in \lstref{aufruf-konsolenversion} zu sehen. Es gibt zwei verpflichtende Parameter: Die Pfade zur \ma und zur \ac -Datei\notiz{Irgendwo den Aufbau der beiden Dateien erwähnen}. Die beiden Pfade können sowohl relativ als auch absolut angegeben werden, wichtig ist allerdings, dass zuerst der Pfad zur \ma und dann der Pfad zur \ac -Datei angegeben wird. Werden die beiden Pfade vertauscht, so startet der \md nicht und bricht mit einer Fehlermeldung ab.

\begin{lstlisting}[language=sh,caption={Aufruf des \md -- Konsolenversion},label=\lstlbl{aufruf-konsolenversion}]
  micro-debug.sh [PARAMETER]... MIC1 IJVM
\end{lstlisting}

Neben den beiden Dateipfaden gibt es im Folgenden erklärte optionale Parameter. Jeder Parameter kann sowohl in der langen (mit doppeltem Minus) als auch in der kurzen (einfaches Minus gefolgt von einem Zeichen) Variante angegeben werden.

\begin{description}
\item[-h, -{}-help]
  ist dieser Parameter gegeben, so wird die Hilfe angezeigt, die neben den verschiedenen Aufrufmöglichkeiten die möglichen Parameter erklärt. Zusätzlich werden noch einige andere Informationen angezeigt, wie beispielsweise Kontaktmöglichkeiten oder ein Hinweis, wo Fehler berichtet werden können.

\item[-o, -{}-output-file FILE]
  wenn dieser Parameter angegeben ist, so wird die Ausgabe der \mic (nicht des \md) in eine Datei umgelenkt. Normalerweise wird die Ausgabe der \mic auf der Konsole ausgegeben.

  Das Argument \texttt{FILE} ist der Pfad zur Datei, in die die Ausgabe geschrieben werden soll. Wenn die Datei bereits existiert, wird die Ausgabe an das Ende der Datei angehängt.

  Unter Linux kann man dann in einer zweiten Konsole diese Datei beispielsweise mit \texttt{tail~-f} anzeigen und die Ausgabe der \mic komfortabel von der Ausgabe des \md trennen. In diesem Szenario ist auch der Parameter \texttt{-{}-unbuffered-output} sinnvoll.

\item[-u, -{}-unbuffered-output]
  verhindert die Pufferung der Ausgabe der \mic. Normalerweise gibt der \md die Ausgabe der \mic zeilenweise aus, also erst bei der Ausgabe eines Zeilenumbruchs. Verwendet man den Parameter \texttt{-{}-output-file} oder möchte man aus sonstigen Gründen jedes ausgegebene Zeichen der \mic direkt auf der Konsole sehen, so ist dieser Parameter die Lösung.

  \emph{Vorsicht!} Wird dieser Parameter ohne \texttt{-{}-output-file} genutzt, so kann ist es schwer, die ausgegebenen Zeichen der \mic ausfindig zu machen, da sie in der Menge der Ausgaben des \md untergehen.

\item[-v, -{}-version]
  gibt die Version des \md aus.
\end{description}

Wenn einer der Parameter \texttt{-{}-help} oder \texttt{-{}-version} angegeben wurde, startet der \md nicht. Dies kann genutzt werden, um ohne vorhandene Bytecode-Dateien Informationen über den \md anzeigen zu können. \lstref{aufrufe-ohne-start} zeigt, wie der \md daher ohne Bytecode-Dateien aufgerufen werden kann.

\begin{lstlisting}[language=sh,caption={Aufruf des \md ohne Start -- Konsolenversion},label=\lstlbl{aufrufe-ohne-start}]
  micro-debug.sh --help
  micro-debug.sh --version
\end{lstlisting}

Bei der Abarbeitung der gegebenen Parameter hängt die Reihenfolge der Parameter nicht von der Reihenfolge ab, in der sie dem \md als Parameter übergeben wurden. Wichtig ist nur, dass die beiden Bytecode-Dateien die letzten beiden Parameter und in der richtigen Reihenfolge aufgeführt sind.

\section{Befehle}
Bei der Bedienung des \md muss der Benutzer zwischen den verschiedenen Befehlen auch Zeichen für die \mic eingeben. Damit der Benutzer sieht, ob die Eingabe für den \md oder die \mic sind, gibt der \md '\texttt{micro-debug> }' und die \mic '\texttt{mic1> }' aus, bevor eine Eingabe erwartet wird. Wenn die \mic eine Eingabe erwartet, kann der Benutzer mehrere Zeichen auf einmal eingeben -- an die \mic werden die eingegebenen Zeichen plus ein Zeilenumbruch gesendet. Sollte der \ac einzelne Zeichen in einer Schleife einlesen, sollte die gesamte Zeile daher auf einmal eingeben werden.

Ist der \md gestartet lässt er sich durch verschiedene Befehle bedienen, die ich jetzt ausführlich beschreiben werde.

Die verschiedenen Befehle können ein oder mehrere Argumente benötigen. Die Argumente haben einen der folgenden Datentypen:
\begin{description}
\item[Register] ist der Name eines Registers, also \reg{CPP}, \reg{H}, \reg{LV}, \reg{MAR}, \reg{MBR}, \reg{MBRU}, \reg{MDR}, \reg{OPC}, \reg{PC}, \reg{SP} oder \reg{TOS}.
\item[Zahl] ist eine Zahl im Wertebereich eines \emph{Integers}\notiz{Verweis?}. Die Formate für die Eingabe der Zahl habe ich in \secref{zahlenformat} beschrieben.
\end{description}

Je nach Befehl kann der zulässige Wertebereich eingeschränkt sein -- eine solche Einschränkung ergibt sich aus der Beschreibung des jeweiligen Befehls.

\begin{description}
\item[break] \emph{Register [Zahl]}

  setzt einen Breakpoint für das gegebene \emph{Register}: Sobald das Register den Wert \emph{Zahl} erhält, hält der \md an. Der \md hält, \textbf{nachdem} das Register den Wert erhalten hat -- da dies unter Umständen zu spät ist, gibt es die Möglichkeit, das Argument \emph{Zahl} wegzulassen. Wird nur ein Register (ohne Wert) angegeben, hält der \md \textbf{bevor} das Register einen neuen Wert zugewiesen bekommt.

\item[exit] \hspace*{\fill}\\

  beendet den \md.

\item[help] \hspace*{\fill}\\

  zeigt die verfügbaren Befehle mit einer kurzen Beschreibung an. Zeigt desweiteren einige Zusatzinformationen über den \md an.

\item[ls-break] \hspace*{\fill}\\

  zeigt alle Breakpoints mit der jeweiligen Bedingung an. Jeder Breakpoint erhält eine Identifikationsnummer, die bei anderen Operationen, wie dem Entfernen, angegeben werden muss.

\item[ls-macro-code] \emph{[Zahl1 [Zahl2]]}

  zeigt den disassemblierten \ac an. Dabei gibt es drei mögliche Konstellationen der Parameter:
  \begin{itemize}
  \item Wird \textbf{kein Parameter} gegeben, wird der vollständige \ac angezeigt.
  \item Wird \textbf{ein Parameter} \emph{Zahl1} gegeben, wird die angegebene Anzahl an Zeilen vor und nach der aktuellen Codezeile angezeigt.
  \item Werden \textbf{zwei Parameter} gegeben, wird der \ac von Zeile \emph{Zahl1} bis zur Zeile \emph{Zahl2} (inklusive) angezeigt.
  \end{itemize}

  Da der \md nur den Bytecode kennt, gibt es eigentlich keine Zeilen. Der \md schreibt pro Zeile einen Befehl (inklusive Argumente), somit existieren Zeilennummern, die für den \md als Referenz dienen.

\item[ls-micro-code] \emph{[Zahl1 [Zahl2]]}

  zeigt den disassemblierten \mac an und arbeitet wie \texttt{ls-macro-code}. Es gibt drei mögliche Konstellationen der Parameter:
  \begin{itemize}
  \item Wird \textbf{kein Parameter} gegeben, wird der vollständige \mac angezeigt.
  \item Wird \textbf{ein Parameter} \emph{Zahl1} gegeben, wird die angegebene Anzahl an Zeilen vor und nach der aktuellen Codezeile angezeigt.
  \item Werden \textbf{beide Parameter} gegeben, wird der Mikro-Assembler-Code von Zeile \emph{Zahl1} bis zur Zeile \emph{Zahl2} (inklusive) angezeigt.
  \end{itemize}

  Wie bei \texttt{ls-macro-code} schreibt der \md pro Zeile eine \mai (die $36~Bit$, die die Instruktion spezifizieren), somit existieren Zeilennummern, die für den \md als Referenz dienen.

\item[ls-mem] \emph{Zahl1 Zahl2}

  zeigt den Inhalt des Hauptspeichers zwischen den Adressen \emph{Zahl1} und \emph{Zahl2} (inklusive) an. Die Adressen des Hauptspeichers sind Wortadressen -- jedes Wort hat eine Größe von $32~Bit$.

\item[ls-reg] \emph{[Register]}

  zeigt den Wert von dem gegebenen \emph{Register} an; wird das optionale Argument weggelassen, werden alle Register und deren Werte angezeigt.

\item[ls-stack] \hspace*{\fill}\\

  zeigt den aktuellen Stack an.

  \emph{Hinweis:} Dieser Befehl wird durch die Konfigurationsoption \texttt{stack.elements.to.hide} beeinflusst. Normalerweise wird der Stack von dem initialen Stackpointer bis zum aktuellen Stackpointer ausgegeben. Da dies unter Umständen mehr Elemente liefert, als der Stack tatsächlich enthält, gibt es die Möglichkeit über die Konfiguration die Ausgabe der ersten Elemente zu unterdrücken.

Möchte man den realen (im Speicher vorhandenen) Stack sehen, sollte man sicherstellen, dass \texttt{stack.elements.to.hide = 0} konfiguriert ist.

\item[macro-break] \emph{Zahl}

  fügt einen Breakpoint hinzu, der den \md anhält, sobald der \ac an der Adresse \emph{Zahl} ausgeführt werden soll. \emph{Adresse} muss Element aus der Menge der von \texttt{ls-macro-code} angezeigten Zeilennummern sein.

\item[micro-break] \emph{Zahl}

  fügt einen Breakpoint hinzu, der den \md anhält, sobald der \mac an der Adresse \emph{Zahl} ausgeführt werden soll. \emph{Adresse} muss Element aus der Menge der von \texttt{ls-micro-code} angezeigten Zeilennummern sein.

\item[micro-step] \emph{[Zahl]}

  führt die nächsten \emph{Zahl} \mais aus; wird kein Argument gegeben, so wird eine \mai ausgeführt.

\item[reset] \hspace*{\fill}\\

  die \mic wird in den Anfangszustand zurückgesetzt: Der Hauptspeicher und die Register werden auf die initialen Werte zurückgesetzt. Auch die Ein- und Ausgabe der \mic wird geleert. Die Informationen des \md, vor allem die Breakpoints, bleiben allerdings erhalten und müssen vom Benutzer nicht erneut gesetzt werden.

\item[rm-break] \emph{Zahl}

  entfernt den Breakpoint mit der Nummer \emph{Zahl}. Die Nummer des Breakpoints ist die Identifikationsnummer, die mit \texttt{ls-break} angezeigt wird.

\item[run] \hspace*{\fill}\\
  
  führt alle Instruktionen bis zum Programmende oder bis zum nächsten Breakpoint aus.

  \emph{Hinweis:} Unter Umständen und ungünstigem Programmcode kann das zu debuggende Programm in eine Schleife geraten, welche ohne Breakpoints nur durch den Programmabbruch verlassen werden kann.

\item[set] \emph{Register Zahl}

  weist dem \emph{Register} den Wert \emph{Zahl} zu.

\item[set-mem] \emph{Zahl1 Zahl2}

  schreibt den Wert \emph{Zahl2} an die Wortadresse \emph{Zahl1} im Hauptspeicher.

  \emph{Hinweis:} Auch wenn dieser Befehl offensichtlich dazu genutzt werden könnte den \ac zu manipulieren, ist er für diesen Zweck nicht vorgesehen.

\item[step] \emph{[Zahl]}

  führt die nächsten \emph{Zahl} \ais aus; wird kein Argument gegeben, so wird eine \ai ausgeführt.

\item[trace-mac] \hspace*{\fill}\\
  
  der \ac wird nun beobachtet. Dadurch wird jede \ai angezeigt, nachdem sie ausgeführt wurde.

\item[trace-mic] \hspace*{\fill}\\

  der \mac wird nun beobachtet. Dadurch wird jede \mai angezeigt, nachdem sie ausgeführt wurde.

\item[trace-reg] \emph{[Register]}

  das gegebene \emph{Register} wird nun beobachtet. Dadurch wird der Wert des Registers angezeigt, wenn er sich ändert. Wird das optionale Argument weggelassen, werden alle \emph{Register} beobachtet.

\item[trace-var] \emph{Zahl}

  die Variable \emph{Zahl} wird nun beobachtet. Dadurch wird der Inhalt der Variable angezeigt, wenn er sich ändert.

  \emph{Zahl} ist die Nummer der lokalen Variable. Wird eine Methode im \ac aufgerufen, ändert sich der Zeiger \reg{LV} und damit auch die Identität der lokalen Variablen. Wird also Variable Nummer~1 in Methode X beobachtet und führt der \md gerade Methode Y aus, wird eine Änderung der jetzigen lokalen Variable~1 nicht ausgegeben (sofern diese nicht auch beobachtet wird).

\item[untrace-mac] \hspace*{\fill}\\
  
  beendet das Beobachten des \ac. Dadurch werden ausgeführte \ai nun nicht mehr ausgegeben.

\item[untrace-mic] \hspace*{\fill}\\

  beendet das Beobachten des \mac. Dadurch werden ausgeführte \mai nun nicht mehr ausgegeben.

\item[untrace-reg] \emph{[Register]}

  beendet das Beobachten des angegebenen \emph{Register}s. Wird das optionale Argument weggelassen, wird nun kein Register mehr beobachtet.

\item[untrace-var] \emph{Zahl}

  beendet das Beobachten der lokalen Variable Nummer \emph{Zahl}.
\end{description}

\section{Tutorial}
\seclbl{tutorial-konsole}
In diesem Abschnitt möchte ich ein Tutorial beschreiben, um in das Arbeiten mit dem \md einzuführen. Ich beschreibe das Debuggen eines Assembler-Programms: ein Programm zum Einlesen von Binärzahlen. \lstref{binary-read-c} zeigt den Code für das Programm in C\notiz{C referenz? erklärn?} und soll hier als Verständnis des Algorithmuses dienen.

\begin{lstlisting}[language=c,caption={C-Programm zum Einlesen einer Binärzahl},label=\lstlbl{binary-read-c}]
int main() {
  int character = 0;
  int result = 0;
  while(1) {
    character = getchar();
    if( c == '\n' ) {
      return result;
    }
    c = c - '0';
    result = 2 * result + c;
  }
}
\end{lstlisting}

Der entsprechende \ac ist in \lstref{binary-read-jas} aufgeführt. Dieses Programm muss nun kompiliert werden -- \name{Ontko} stellt dafür in \cite{Ontko1999} einige Programme bereit: \texttt{mic1asm} zum Kompilieren des \mac und \texttt{ijvmasm} zum Kompilieren des \ac. In \cite{Ontko1999} ist auch ein \mac zu finden, der für dieses Tutorial ausreichend ist; außerdem gibt es dort die zum \mac passende \texttt{ijvm.conf}-Datei.

\begin{lstlisting}[language=,caption={IJVM-Assembler zum Einlesen einer Binärzahl},label=\lstlbl{binary-read-jas}]
.main
.var
    c
    result
.end-var
    bipush      0
    istore      result
loop:
    in
    istore      c
    iload       c
    bipush      10
    if_icmpeq   finish
    iinc        c       -48(*@\srclbl{iinc-increment-value}@*)
    iload       result
    dup
    iadd
    iload       c
    iadd
    goto        loop
finish:
    iload       result
    halt
.end-main
\end{lstlisting}

Zum Debuggen des \ma und \ac empfehle ich, die \texttt{ijvm.conf}-Datei zu verwenden, die zum Kompilieren der \datei{mic1} genutzt wurde. Diese \date{conf} kann entweder in das \texttt{conf/} Verzeichnis des \md gelegt werden, oder jeweils in das Verzeichnis, von dem aus der \md ausgeführt wird. Wie erkenne ich, ob ich die korrekte \texttt{ijvm.conf}-Datei verwende? Beim Ausführen des Befehls \texttt{ls-macro-code}: zeigt der \md unbekannte \ais, enthält der \ac Instruktionen, die in der \texttt{ijvm.conf} nicht oder mit abweichender Adresse definiert sind.

Vor dem Start des \md empfehle ich außerdem, die Konfigurationsoption \texttt{mic1.micro.address.ijvm} zu überprüfen. Diese Option enthält die Adresse der \mai, die von allen \mac -Methoden angesprungen wird, um die nächste Mikro-Instruktion zu \emph{laden}. Ist diese Option falsch konfiguriert, funktioniert später der Befehl \texttt{step} nicht wie erwartet -- der \md hält nicht oder an falscher Stelle.

Die in \lstref{binary-read-jas} aufgeführte Methode soll später eine eigene Methode werden und gibt daher keinen Wert aus, sondern legt das Ergebnis am Ende auf den Stack. \lstref{tutorial-start} zeigt in \srcref{tutorial-startbefehl} den Befehl, um den \md zu starten -- im aktuellen Verzeichnis liegen die Dateien \texttt{mic1ijvm.mic1}, \texttt{binary-read.ijvm} und \texttt{ijvm.conf}.

\begin{lstlisting}[language=,caption={Start des \md},label=\lstlbl{tutorial-start}]
micro-debug.sh mic1ijvm.mic1 binary-read.ijvm(*@\srclbl{tutorial-startbefehl}@*)
MicroDebug - Copyright (C) 2011-2012 Christian Roesch AND 1999 Prentice-Hall, Inc. (*@\srclbl{tutorial-start-willkommen}@*)
Welcome! Please type 'help' for a list of valid commands
----------------------------------------
micro-debug> (*@\srclbl{tutorial-start-mdread}@*)
\end{lstlisting}

Ab \srcref{tutorial-start-willkommen} steht die Willkommensnachricht des \md, gefolgt von der \srcref{tutorial-start-mdread}, die anzeigt, dass der \md nun einen Befehl erwartet. Mit dem Befehl \texttt{help} kann jederzeit eine ausführliche Beschreibung der verfügbaren Befehle anzeigt werden lassen.

Mit dem Befehl \texttt{ls-macro-code} erhält man den disassemblierten \ac -- in \lstref{tutorial-macro-code} ist die Ausgabe des Befehls zu sehen. Die Ausgabe zeigt zunächst pro Zeile die \ac -Zeile, dann die Adresse Instruktion im \mac und anschließend den Namen des Befehls mit seinen Argumenten. Die Ausgabe ist nicht identisch mit dem Quellcode, der kompiliert wurde -- aus \lstref{binary-read-jas} -- in der disassemblierten Variante fehlen Informationen wie Kommentare, Variablennamen und Sprungmarkennamen. Beispielsweise zeigt \srcref{tutorial-macro-code-sprung} einen bedingten Sprung zur Zeile \texttt{0x1B}.

\begin{lstlisting}[language=,caption={Disassemblierter \ac},label=\lstlbl{tutorial-macro-code}]
     0x0: [ 0x10] BIPUSH  0x0
     0x2: [ 0x36] ISTORE  1
(*@\srclbl{tutorial-macro-code-in}@*)     0x4: [ 0xFC] IN 
     0x5: [ 0x36] ISTORE  0(*@\srclbl{tutorial-macro-code-after-in}@*)
     0x7: [ 0x15] ILOAD  0
     0x9: [ 0x10] BIPUSH  0xA
(*@\srclbl{tutorial-macro-code-sprung}@*)     0xB: [ 0x9F] IF_ICMPEQ  0x1B
     0xE: [ 0x84] IINC  0 0xD0
    0x11: [ 0x15] ILOAD  1
    0x13: [ 0x59] DUP 
    0x14: [ 0x60] IADD 
    0x15: [ 0x15] ILOAD  0
    0x17: [ 0x60] IADD 
    0x18: [ 0xA7] GOTO  0x4
    0x1B: [ 0x15] ILOAD  1
    0x1D: [ 0xFF] HALT 
\end{lstlisting}

Das erwartete Ergebnis eines Programmlaufs ist, dass die eingegebene Binärzahl am Ende auf dem Stack und damit im Register \reg{TOS} vorliegt -- wenn ich $1010$ eingebe, soll \reg{TOS} nach einem Programmdurchlauf den Wert $10$ enthalten.

Mit dem Befehl \texttt{run} lasse ich das Programm nun zunächst ohne Breakpoints laufen. Nachdem das Programm gestartet ist, wird die \mic in \srcref{tutorial-macro-code-in} aus \lstref{tutorial-macro-code} mit dem Befehl \texttt{IN} Zeichen einlesen. Das erkennt man auf der Konsole an der \srcref{tutorial-mic-eingabe-txt} aus \lstref{tutorial-mic-eingabe} -- statt \texttt{micro-debug>} steht hier nun \texttt{mic1>}.

\begin{lstlisting}[language=,caption={\mic erwartet Eingabe},label=\lstlbl{tutorial-mic-eingabe}]
micro-debug> run
mic1> (*@\srclbl{tutorial-mic-eingabe-txt}@*)
\end{lstlisting}

Ich gebe nun \texttt{1010} ein und bestätige die Eingabe mit \texttt{ENTER} -- dadurch werden fünf Zeichen im \md gepuffert: Die vier Zeichen, die ich eingegeben habe plus ein Zeilenumbruch. Jedes Mal, wenn die \mic nun ein Zeichen benötigt, wird aus diesem Puffer gelesen. Erst wenn dieser leer ist, erscheint erneut die Eingabeaufforderung für den Benutzer.

\begin{lstlisting}[language=,caption={\md gibt Anzahl ausgeführter Zyklen aus},label=\lstlbl{tutorial-durchgelaufen}]
Processor executed 441 ticks.
micro-debug> 
\end{lstlisting}

Nachdem ich nun die Zahl eingegeben habe erscheint die Ausgabe aus \lstref{tutorial-durchgelaufen}, die anzeigt, dass die \mic insgesamt $441$ Zyklen ausgeführt hat. Da mein Programm keine Ausgabe macht, sondern das Ergebnis auf den Stack (und damit in dem Register \reg{TOS}) ablegt, überprüfe ich das nun wie folgt. Mit dem Befehl \texttt{ls-reg} lasse ich mir den Inhalt des Registers \reg{TOS} anzeigen, das ist in \lstref{tutorial-tos-falscher-wert} zu sehen.

Das Register \reg{TOS} enthält den Wert $0$ und damit einen falschen Wert. Ich kann nun noch den Stack ansehen, um zu überprüfen, ob dort der erwartete Wert $10$ abgelegt ist. Den Stack sehe ich mit dem Befehl \texttt{ls-stack} an, was die in \lstref{tutorial-stack} gezeigte Ausgabe liefert.

\begin{lstlisting}[language=,caption={\md gibt Anzahl ausgeführter Zyklen aus},label=\lstlbl{tutorial-tos-falscher-wert}]
micro-debug> ls-reg TOS
Register TOS : 0x0
micro-debug> 
\end{lstlisting}

Der erwartete Wert $10$ liegt auch nicht auf dem Stack. Wäre der Wert auf dem Stack, aber nicht im Register \reg{TOS} wäre das ein Hinweis auf einen Fehler im \mac. Denn der \mac ist für die Einhaltung der Regel zuständig, dass das Register \reg{TOS} stets den Wert des obersten Elements des Stacks enthalten muss.

\begin{lstlisting}[language=,caption={Inhalt des Stacks nach der Ausführung des \ac},label=\lstlbl{tutorial-stack}]
Stack value #1 [  0xC001]: 0x1
Stack value #2 [  0xC002]: 0x0
Stack value #3 [  0xC003]: 0x1
Stack value #4 [  0xC004]: 0x0
Stack value #5 [  0xC005]: 0x0
micro-debug> 
\end{lstlisting}

Damit das Programm für weitere Analysen erneut ablaufen gelassen werden kann, muss die \mic auf ihren Startzustand zurückgesetzt werden; mit dem Befehl \texttt{reset}. Woran könnte das Problem liegen? Womöglich werden falsche Werte eingelesen -- mit einem Breakpoint auf dem \texttt{IN}-Befehl überprüfe ich diese Vermutung. In \lstref{tutorial-breakpoint-mac} ist aufgeführt, wie ich einen Breakpoint in der Zeile \texttt{0x4} setze und anschließend überprüfe, welche Breakpoints gesetzt sind.

\begin{lstlisting}[language=,caption={Setzen eines Breakpoints im \ac},label=\lstlbl{tutorial-breakpoint-mac}]
micro-debug> macro-break 4
micro-debug> ls-break
Breakpoint #1: at macro code line 0x4
micro-debug> 
\end{lstlisting}

Nachdem der Breakpoint gesetzt ist, kann ich das Programm mit dem Befehl \texttt{run} ausführen. An der Ausgabe erkennt man, dass die \mic nicht das gesamte Programm ausgeführt hat, sondern nur 16~Zyklen. Ich werde nun den \texttt{IN}-Befehl überprüfen und dazu einzeln die \mai des Befehls ausführen. Damit ich nicht jedes Mal den \mac anzeigen lassen muss, beobachte ich den \mac mit dem Befehl \texttt{trace-mic}.

\begin{lstlisting}[language=,caption={Ausgeführte \mais bis zum Sprungbefehl \texttt{goto (MBR)}},label=\lstlbl{tutorial-steps-to-goto}]
micro-debug> micro-step
Executed: fetch;goto 0x4
Processor executed 1 ticks.
micro-debug> micro-step
Executed: PC=PC+1;goto 0x5
Processor executed 1 ticks.
micro-debug> micro-step
Executed: goto (MBR)
Processor executed 1 ticks.
\end{lstlisting}

Je nach \mac kann es mehrere \mais dauern, bis man zur Ausführung des \texttt{IN}-Befehls gelangt. Wichtig ist die Zeile, die \texttt{goto~(MBR)} ausführt -- daher führe ich nun so lange den Befehl \texttt{micro-step} aus, bis dieser Sprungbefehl ausgeführt wird. Bei mir werden von der Stelle, an der der \md gehalten hat, bis zu dem Sprungbefehl drei \mais ausgeführt, wie in \lstref{tutorial-steps-to-goto} zu sehen ist.

\begin{lstlisting}[language=,caption={Mikro-Assembler-Code des Befehls \texttt{IN}},label=\lstlbl{tutorial-in-micro}]
0xFC: H=OPC=-1;goto 0x6A
0x6A: OPC=H+OPC;goto 0x6B
0x6B: MAR=H+OPC;rd;goto 0x6C(*@\srclbl{tutorial-in-micro-rd}@*)
0x6C: SP=MAR=SP+1;goto 0x6D
0x6D: TOS=MDR;wr;goto 0x3
\end{lstlisting}

Bei meiner Implementierung des \mac liegt der Befehl \texttt{IN} an der Adresse \texttt{0xFC}. \lstref{tutorial-in-micro} enthält den \mac für den Befehl \texttt{IN}; bis zur Zeile \texttt{0x6B} wird die Adresse für \emph{memory mapped IO} erzeugt: $-3$. In Zeile \texttt{0x6B} (\srcref{tutorial-in-micro-rd} in \lstref{tutorial-in-micro}) wird der \texttt{rd}-Befehl ausgeführt, der letztendlich das Zeichen über den Hauptspeicher einliest.

\begin{lstlisting}[language=,caption={Überprüfung des \texttt{rd}-Befehls durch den Inhalt des Registers \reg{MDR}},label=\lstlbl{tutorial-check-rd-mdr}]
micro-debug> micro-step 3
Executed: H=OPC=-1;goto 0x6A
Executed: OPC=H+OPC;goto 0x6B
mic1> 1010
Executed: MAR=H+OPC;rd;goto 0x6C
Processor executed 3 ticks.
micro-debug> micro-step 1(*@\srclbl{tutorial-check-rd-mdr-step}@*)
Executed: SP=MAR=SP+1;goto 0x6D
Processor executed 1 ticks.
micro-debug> ls-reg MDR(*@\srclbl{tutorial-check-rd-mdr-cont}@*)
Register MDR : 0x31
\end{lstlisting}

In \lstref{tutorial-check-rd-mdr} ist zu sehen, wie die \mic die drei \mais zum Aufbau der Hauptspeicheradresse und zum Einlesen des Zeichens ausführt. Während der Ausführung fragt der \md im Auftrag der \mic nach einer Eingabe, hier gebe ich wieder \texttt{1010} ein und bestätige mit \texttt{ENTER}.

Die \mic hat nun Zeile \texttt{0x6B} ausgeführt und das erste Zeichen gelesen. Dieses Zeichen wird allerdings erst im nächsten Zyklus in das Register \reg{MDR} geschrieben -- daher habe ich wie in \srcref{tutorial-check-rd-mdr-step} zu sehen eine weiteren \mai ausgeführt. Jetzt muss das gelesene Zeichen im Register \reg{MDR} angekommen sein, was ich mit dem Befehl \texttt{ls-reg~MDR} überprüfe. Wie ab Zeile \srcref{tutorial-check-rd-mdr-cont} zu sehen, enthält das Register \reg{MDR} den Wert $0x31=49$ -- der \emph{ASCII}-Code für das Zeichen \texttt{1}.

Der Fehler scheint demnach nicht im \mac zu liegen -- zumindest nicht, im \texttt{IN}-Befehl. Deswegen konzentriere ich mich jetzt auf den \ac. Mit dem Befehl \texttt{step} führe ich die letzte \mai des Befehls \texttt{IN} aus. Die \mic befindet sich nun in Zeile \texttt{0x5} des \ac -- der \srcref{tutorial-macro-code-after-in} aus \lstref{tutorial-macro-code}.

Damit ich die Veränderungen der lokalen Variablen direkt erkenne, beobachte ich sie. In \lstref{binary-read-jas} ist zu sehen, wie viele lokale Variablen vorhanden sind: zwei. Ich beobachte beide, um zu sehen, ob der \emph{ASCII}-Code korrekt konvertiert wird und welche Werte das Ergebnis annimmt. In \lstref{tutorial-watch-vars} ist zu sehen, wie ich dazu den Befehl \texttt{trace-var} nutze.

\begin{lstlisting}[language=,caption={Beobachten beider lokaler Variablen},label=\lstlbl{tutorial-watch-vars}]
micro-debug> trace-var 0
micro-debug> trace-var 1
\end{lstlisting}

Ich werde das Programm nun schrittweise ausführen -- da ich auf die Korrektheit des \mac vertraue, führe ich jeweils \ais aus. Dann überprüfe ich, ob das erwartete Verhalten eingetreten ist und suche so schrittweise die Ursache für den Fehler. Zur Erinnerung, die \mic hat gerade eine \texttt{1} eingelesen.

Mit dem Befehl \texttt{step} wird die nächste \ai ausgeführt. In \lstref{tutorial-step-trace-mic} ist die Ausgabe dieses Befehls zu sehen -- da ich den \mac noch beobachte, ist diese Ausgabe unübersichtlich. Ich kann aber in \srcref{tutorial-step-trace-mic-varwrite} sehen, dass die lokale Variable \texttt{0} den korrekten Wert erhalten hat.

\begin{lstlisting}[language=,caption={Ausführen einer \ai bei Beobachten des \mac},label=\lstlbl{tutorial-step-trace-mic}]
micro-debug> step
Executed: fetch;goto 0x4
Executed: PC=PC+1;goto 0x5
Executed: goto (MBR)
Executed: H=LV;fetch;goto 0x1F
Executed: MDR=TOS;goto 0x20
Executed: MAR=H+MBRU;wr;goto 0x21
Local variable 0: 49(*@\srclbl{tutorial-step-trace-mic-varwrite}@*)
Executed: SP=MAR=SP-1;rd;goto 0x22
Executed: PC=PC+1;goto 0x23
Executed: TOS=MDR;goto 0x3
Processor executed 9 ticks.
\end{lstlisting}

Damit ich im Folgenden übersichtlichere Ausgaben erhalte, beende ich das Beobachten des \mac durch den Befehl \texttt{untrace-mic}. Zusätzlich Beobachten ich nun den \ac, um bei jeder ausgeführten \ai überprüfen zu können, was gerade ausgeführt wurde. Nachdem ich den Befehl \texttt{trace-mac} eingegeben habe, kann ich das Programm weiter schrittweise ausführen.

\begin{lstlisting}[language=,caption={Ausführen der \ais zum Vergleich zweier lokaler Variablen},label=\lstlbl{tutorial-check-var}]
micro-debug> step
Executed:      0x7: [ 0x15] ILOAD  0
Processor executed 8 ticks.
micro-debug> ls-stack
Stack value #1 [  0xC001]: 0x31
micro-debug> step
Executed:      0x9: [ 0x10] BIPUSH  0xA
Processor executed 6 ticks.
micro-debug> ls-stack
Stack value #1 [  0xC001]: 0x31
Stack value #2 [  0xC002]: 0xA
micro-debug> step
Executed:      0xB: [ 0x9F] IF_ICMPEQ  0x1B
Processor executed 11 ticks.
micro-debug> ls-stack
Stack doesn't contain any elements, nothing to display.(*@\srclbl{tutorial-check-var-stack-empty}@*)
\end{lstlisting}

\lstref{tutorial-check-var} enthält die Konsolenausgabe nach der Ausführung der nächsten drei \ais. Da diese \ais auf dem Stack operieren, habe ich zusätzlich nach jedem Befehl den Stack angesehen. Das Programm prüft hier, ob die eingegebene Zahl ein Zeilenumbruch ist. Dafür legt es die eingegebene Zahl und einen Zeilenumbruch auf den Stack und ruft die Instruktion \texttt{IF_ICMPEQ} auf, die bei Gleichheit der beiden obersten Zahlen auf dem Stack an die angegebene Adresse springt.

An den Konsolenausgaben ist zu erkennen, dass zunächst beide Zahlen korrekt auf den Stack gelegt werden. Der Vergleich sollte die beiden Werte vom Stack entfernen und sie dann vergleichen; in \srcref{tutorial-check-var-stack-empty} in \lstref{tutorial-check-var} sieht man, dass der Stack korrekterweise keine Elemente enthält. Wenn ich nun die nächste \ai ausführe, kann ich an der Adresse der \ai erkennen, ob der bedingte Sprung ausgeführt wurde.

\begin{lstlisting}[language=,caption={Ausführen der \ais zur Berrechnung des temporären Ergebnisses},label=\lstlbl{tutorial-calc-temp-result}]
micro-debug> step
Executed:      0xE: [ 0x84] IINC  0 0xD0
Local variable 0: 1
Processor executed 9 ticks.
micro-debug> step
Executed:     0x11: [ 0x15] ILOAD  1
Processor executed 8 ticks.
micro-debug> step
Executed:     0x13: [ 0x59] DUP 
Processor executed 5 ticks.
micro-debug> step
Executed:     0x14: [ 0x60] IADD 
Processor executed 6 ticks.
micro-debug> step
Executed:     0x15: [ 0x15] ILOAD  0
Processor executed 8 ticks.
micro-debug> step
Executed:     0x17: [ 0x60] IADD 
Processor executed 6 ticks.
micro-debug> ls-stack
Stack value #1 [  0xC001]: 0x1(*@\srclbl{tutorial-calc-temp-result-result}@*)
\end{lstlisting}

Der bedingte Sprung wurde nicht ausgeführt! Stattdessen wurde nun der \emph{ASCII}-Code dekodiert und die lokale Variable enthält den Wert \texttt{1}. Im Folgenden wird das Programm das aktuelle Ergebnis -- das derzeit \texttt{0} ist -- mit zwei multiplizieren und die gerade gelesene Zahl dazu addieren.

\lstref{tutorial-calc-temp-result} zeigt die gerade beschriebenen Schritte: Zunächst wird das derzeitige Ergebnis geladen, auf dem Stack dupliziert und dann aufaddiert, was der Multiplikation mit zwei entspricht. Anschließend wird die gelesene Zahl auf den Stack gelegt und dazu addiert und sollte das aktuelle Ergebnis ergeben. Mit dem Befehl \texttt{ls-stack} kann ich nun sehen, dass diese Annahme korrekt ist: es liegt der Wert \texttt{1} auf dem Stack, wie in \srcref{tutorial-calc-temp-result-result} in \lstref{tutorial-calc-temp-result} zu sehen.

Bisher scheint der \ac auch korrekt zu sein. In \lstref{tutorial-error-reason} ist die Konsolenausgabe nach der Ausführung der nächsten \ais zu sehen. Hier führt die \mic den \texttt{GOTO}-Befehl aus und springt zu Adresse \texttt{0x4}, an der der \texttt{IN}-Befehl liegt.

\begin{lstlisting}[language=,caption={Ausführen des fehlerhaften Schritts},label=\lstlbl{tutorial-error-reason}]
micro-debug> step
Executed:     0x18: [ 0xA7] GOTO  0x4
Processor executed 8 ticks.
\end{lstlisting}

Vermutlich liegt der Fehler des Programms darin, das berechnete Ergebnis nicht in der lokalen Variable zu speichern. Wenn diese Vermutung stimmt, sollte der Stack pro Schleifendurchlauf wachsen, da das Ergebnis am Ende der Schleife auf dem Stack liegt, aber nicht mehr gelesen wird. Dies lässt sich durch die Beobachtung aus \lstref{tutorial-stack} zumindest nicht widerlegen: der Stack enthielt am Ende des Programmlaufs viele Elemente.

Diese Vermutung soll nun näher überprüft werden. Dazu setze ich noch einen Breakpoint auf Zeile \texttt{0x18} des \ac, dem Schleifenende. Da die Zeile \texttt{0x4} unmittelbar nach der Zeile \texttt{0x18} ausgeführt wird, kann ich den Breakpoint in Zeile \texttt{0x4} entfernen.

\lstref{tutorial-trace-vars} zeigt das Setzen des Breakpoints in Zeile \texttt{0x18}; anschließend ist die Anzeige aller Breakpoints zu sehen. In \srcref{tutorial-trace-vars-v0} wird die lokale Variable \texttt{0} beobachtet, die das gelesene Zeichen enthält. Der Vermutung nach wird die lokale Variable \texttt{1}, die das Ergebnis enthalten sollte, nie beschrieben; um meine Vermutung eventuell verwerfen zu können, beobachte ich auch diese in Zeile \srcref{tutorial-trace-vars-v1}.

Anschließend entferne ich den alten Breakpoint aus Zeile \texttt{0x4} -- die Nummer, die ich dabei angegebe, ist die Nummer des Breakpoints, die der Befehl \texttt{ls-break} liefert. Nachdem nun die verschiedenen Werte beobachtet werden und ein neuer Breakpoint gesetzt ist, kann mit dem \texttt{RUN}-Befehl jeweils ein Schleifendurchlauf bis zum Ende des Programms ausgeführt werden.

\begin{lstlisting}[language=,caption={Setzen eines Breakpoints und Beobachten der lokalen Variablen},label=\lstlbl{tutorial-trace-vars}]
micro-debug> macro-break 0x18
micro-debug> ls-break
Breakpoint #1: at macro code line 0x4
Breakpoint #2: at macro code line 0x18
micro-debug> trace-var 0(*@\srclbl{tutorial-trace-vars-v0}@*)
micro-debug> trace-var 1(*@\srclbl{tutorial-trace-vars-v1}@*)
micro-debug> rm-break 1
micro-debug> ls-break
Breakpoint #2: at macro code line 0x18
\end{lstlisting}

\begin{lstlisting}[language=,caption={Schleifendurchlauf Nummer~2},label=\lstlbl{tutorial-loop-2}]
micro-debug> run
Executed:      0x4: [ 0xFC] IN 
Executed:      0x5: [ 0x36] ISTORE  0
Local variable 0: 48(*@\srclbl{tutorial-loop-2-48-in-0}@*)
Executed:      0x7: [ 0x15] ILOAD  0(*@\srclbl{tutorial-loop-2-check-if-lf}@*)
Executed:      0x9: [ 0x10] BIPUSH  0xA
Executed:      0xB: [ 0x9F] IF_ICMPEQ  0x1B
Executed:      0xE: [ 0x84] IINC  0 0xD0(*@\srclbl{tutorial-loop-2-48-to-0}@*)
Local variable 0: 0
Executed:     0x11: [ 0x15] ILOAD  1(*@\srclbl{tutorial-loop-2-calc-result}@*)
Executed:     0x13: [ 0x59] DUP 
Executed:     0x14: [ 0x60] IADD 
Executed:     0x15: [ 0x15] ILOAD  0
Executed:     0x17: [ 0x60] IADD 
Processor executed 84 ticks.
\end{lstlisting}

\lstref{tutorial-loop-2} enthält die Konsolenausgabe des nächsten -- dem zweiten -- Schleifendurchlauf. In \srcref{tutorial-loop-2-48-in-0} gibt der \md aus, dass der Wert \texttt{48} korrekt der lokalen Variable \texttt{0} zugewiesen wurde. Die \texttt{48} ist der \emph{ASCII}-Code für \texttt{0}, was das zweite Zeichen meiner Eingabe \texttt{1010} war.

Ab \srcref{tutorial-loop-2-check-if-lf} wird überprüft, ob das gelesene Zeichen ein Zeilenumbruch ist. Auch hier ist korrekt, dass ab \srclbl{tutorial-loop-2-48-to-0} das gelesene Zeichen in den Zahlenwert konvertiert wird und ab \srcref{tutorial-loop-2-calc-result} das Ergebnis berechnet wird.

\begin{lstlisting}[language=,caption={Stack nach Schleifendurchlauf Nummer~2},label=\lstlbl{tutorial-stack-after-loop-2}]
micro-debug> ls-stack
Stack value #1 [  0xC001]: 0x1
Stack value #2 [  0xC002]: 0x0
\end{lstlisting}

In \lstref{tutorial-stack-after-loop-2} ist der aktuelle Stack nach dem jetzigen zweiten Schleifendurchlauf zu sehen. Wie erwartet ist er um ein Element angewachsen, dass nun \texttt{0} ist. Da das Ergebnis nie in der lokalen Variable abgelegt wird, ist in der Schleife das alte Ergebnis \texttt{0}, welches zur Berechnung des aktuellen Ergebnisses verwendet wird. Wenn das bisherige Ergebnis \texttt{0} ist, dann entspricht das aktuelle Ergebnis gerade der gelesenen Zahl.

Aus der Vermutung über die Fehlerursache kann man den Schluss ziehen, dass am Ende der Stack exakt die eingelesenen Ziffern enthalten sollte. Auch diese Schlussfolgerung konnte bereits in \lstref{tutorial-stack} beobachtet werden. Daher akzeptiere ich nun meine Vermutung und versuche den \ac zu korrigieren; \lstref{binary-read-jas-correct} zeigt die korrigierte Version des \ac.

Die Korrektur besteht darin die Zeile \srcref{binary-read-jas-corrected-line} einzufügen -- das Speichern des aktuellen Ergebnisses in der lokalen Variable.

Nachdem die korrigierte Version des \ac kompiliert ist, kann ich sie debuggen und überprüfen, ob die Korrektur erfolgreich war. Der \md kann den \ma und \ac nicht automatisch aktualisieren und muss daher mit dem Befehl \texttt{EXIT} beendet werden. Anschließend kann er mit dem korrigierten Kompilat gestartet werden.

\begin{lstlisting}[language=,caption={IJVM-Assembler zum Einlesen einer Binärzahl (korrigiert)},label=\lstlbl{binary-read-jas-correct}]
.main
.var
    c
    result
.end-var
    bipush      0
    istore      result
loop:
    in
    istore      c
    iload       c
    bipush      10
    if_icmpeq   finish
    iinc        c       -48
    iload       result
    dup
    iadd
    iload       c
    iadd
    istore      result(*@\srclbl{binary-read-jas-corrected-line}@*)
    goto        loop
finish:
    iload       result
    halt
.end-main
\end{lstlisting}

Mit dem Befehl \texttt{ls-macro-code} kann nun überprüft werden, ob der \ac korrekt aktualisiert wurde. In \lstref{tutorial-macro-code-correct} ist die Ausgabe des Befehls zu sehen -- \srcref{tutorial-macro-code-corrected-line} zeigt die neu eingefügte Zeile.

\begin{lstlisting}[language=,caption={Disassemblierter Assembler-Code (korrigiert)},label=\lstlbl{tutorial-macro-code-correct}]
micro-debug> ls-macro-code
     0x0: [ 0x10] BIPUSH  0x0
     0x2: [ 0x36] ISTORE  1
     0x4: [ 0xFC] IN 
     0x5: [ 0x36] ISTORE  0
     0x7: [ 0x15] ILOAD  0
     0x9: [ 0x10] BIPUSH  0xA
     0xB: [ 0x9F] IF_ICMPEQ  0x1D
     0xE: [ 0x84] IINC  0 0xD0
    0x11: [ 0x15] ILOAD  1
    0x13: [ 0x59] DUP 
    0x14: [ 0x60] IADD 
    0x15: [ 0x15] ILOAD  0
    0x17: [ 0x60] IADD 
    0x18: [ 0x36] ISTORE  1(*@\srclbl{tutorial-macro-code-corrected-line}@*)
    0x1A: [ 0xA7] GOTO  0x4
    0x1D: [ 0x15] ILOAD  1
    0x1F: [ 0xFF] HALT 
\end{lstlisting}

Ich habe nun überprüft, dass der \ac korrekt aktualisiert wurde, daher kann ich jetzt überprüfen, ob die Vermutung korrekt war und der Fehler im \ac nun behoben ist.

\begin{lstlisting}[language=,caption={Beobachten des \ac und der lokalen Variablen},label=\lstlbl{tutorial-watch-code-vars}]
micro-debug> trace-mac
micro-debug> trace-var 0
micro-debug> trace-var 1
\end{lstlisting}

Beim Beenden des \md werden Breakpoints sowie beobachtete Register, Variablen oder \ac verworfen -- zur leichteren Übersicht habe ich diese daher nochmal aktiviert. Die Befehle dazu  sind in \lstref{tutorial-watch-code-vars} aufgeführt. Ein Breakpoint werde ich jetzt nicht mehr benötigen; ich werde das Programm ausführen und die Ausgaben analysieren. \lstref{tutorial-output-of-correct-1} und \lstref{tutorial-output-of-correct-2} enthalten die Konsolenausgaben bei der Ausführung des \ac; wieder habe ich \texttt{1010} eingegeben.

\begin{lstlisting}[language=,caption={Konsolenausgabe der Ausführung des korrekten Assembler-Codes -- Teil~1},label=\lstlbl{tutorial-output-of-correct-1}]
micro-debug> run
Executed:      0x0: [ 0x10] BIPUSH  0x0
Executed:      0x2: [ 0x36] ISTORE  1
Executed:      0x4: [ 0xFC] IN 
mic1> 1010
Executed:      0x5: [ 0x36] ISTORE  0
Local variable 0: 49
Executed:      0x7: [ 0x15] ILOAD  0
Executed:      0x9: [ 0x10] BIPUSH  0xA
Executed:      0xB: [ 0x9F] IF_ICMPEQ  0x1D
Executed:      0xE: [ 0x84] IINC  0 0xD0
Local variable 0: 1
Executed:     0x11: [ 0x15] ILOAD  1
Executed:     0x13: [ 0x59] DUP 
Executed:     0x14: [ 0x60] IADD 
Executed:     0x15: [ 0x15] ILOAD  0
Executed:     0x17: [ 0x60] IADD 
Executed:     0x18: [ 0x36] ISTORE  1
Local variable 1: 1
Executed:     0x1A: [ 0xA7] GOTO  0x4
Executed:      0x4: [ 0xFC] IN 
Executed:      0x5: [ 0x36] ISTORE  0
Local variable 0: 48
Executed:      0x7: [ 0x15] ILOAD  0
Executed:      0x9: [ 0x10] BIPUSH  0xA
Executed:      0xB: [ 0x9F] IF_ICMPEQ  0x1D
Executed:      0xE: [ 0x84] IINC  0 0xD0
Local variable 0: 0
Executed:     0x11: [ 0x15] ILOAD  1
Executed:     0x13: [ 0x59] DUP 
Executed:     0x14: [ 0x60] IADD 
Executed:     0x15: [ 0x15] ILOAD  0
Executed:     0x17: [ 0x60] IADD 
Executed:     0x18: [ 0x36] ISTORE  1
Local variable 1: 2
Executed:     0x1A: [ 0xA7] GOTO  0x4
Executed:      0x4: [ 0xFC] IN 
Executed:      0x5: [ 0x36] ISTORE  0
Local variable 0: 49
Executed:      0x7: [ 0x15] ILOAD  0
Executed:      0x9: [ 0x10] BIPUSH  0xA
Executed:      0xB: [ 0x9F] IF_ICMPEQ  0x1D
Executed:      0xE: [ 0x84] IINC  0 0xD0
Local variable 0: 1
Executed:     0x11: [ 0x15] ILOAD  1
Executed:     0x13: [ 0x59] DUP 
Executed:     0x14: [ 0x60] IADD 
Executed:     0x15: [ 0x15] ILOAD  0
Executed:     0x17: [ 0x60] IADD 
Executed:     0x18: [ 0x36] ISTORE  1
Local variable 1: 5
Executed:     0x1A: [ 0xA7] GOTO  0x4
Executed:      0x4: [ 0xFC] IN 
Executed:      0x5: [ 0x36] ISTORE  0
Local variable 0: 48
Executed:      0x7: [ 0x15] ILOAD  0
Executed:      0x9: [ 0x10] BIPUSH  0xA
Executed:      0xB: [ 0x9F] IF_ICMPEQ  0x1D
Executed:      0xE: [ 0x84] IINC  0 0xD0
Local variable 0: 0
Executed:     0x11: [ 0x15] ILOAD  1
Executed:     0x13: [ 0x59] DUP 
Executed:     0x14: [ 0x60] IADD 
Executed:     0x15: [ 0x15] ILOAD  0
Executed:     0x17: [ 0x60] IADD 
Executed:     0x18: [ 0x36] ISTORE  1
Local variable 1: 10
\end{lstlisting}

Der \ac ist nun korrekt: Der Stack und das Register \reg{TOS} enthalten am Ende das Ergebnis -- \texttt{0xA}.

\begin{lstlisting}[language=,caption={Konsolenausgabe der Ausführung des korrekten Assembler-Codes -- Teil~2},label=\lstlbl{tutorial-output-of-correct-2}]
Executed:     0x1A: [ 0xA7] GOTO  0x4
Executed:      0x4: [ 0xFC] IN 
Executed:      0x5: [ 0x36] ISTORE  0
Local variable 0: 10
Executed:      0x7: [ 0x15] ILOAD  0
Executed:      0x9: [ 0x10] BIPUSH  0xA
Executed:      0xB: [ 0x9F] IF_ICMPEQ  0x1D
Executed:     0x1D: [ 0x15] ILOAD  1
Executed:     0x1F: [ 0xFF] HALT 
Processor executed 477 ticks.
micro-debug> ls-stack
Stack value #1 [  0xC001]: 0xA
micro-debug> ls-reg TOS
Register TOS : 0xA
\end{lstlisting}
