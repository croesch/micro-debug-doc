\selectlanguage{ngerman}
\begin{abstract}
  In dieser Arbeit stelle ich den \md und die \mdg vor. Der \md ist ein konsolenbasierter Debugger für die \mic, die \name{Tanenbaum} in \cite{Tanenbaum1998} beschreibt. Die \mdg ist eine \gls{gui} für den \md.

Sowohl für den \md als auch die \mdg beschreibe ich, wie sie bedient werden. Ich zeige ein Tutorial, wie mit dem \md ein Fehler im \ac gefunden werden kann und weise auf die Unterschiede zwischen dem \md und der \mdg hin.

Im zweiten Teil der Arbeit konzentriere ich mich auf die Entwicklung des \md und der \mdg. Die Werkzeuge, die ich zur Entwicklung genutzt habe, stelle ich vor und zeige, wie ein Außenstehender sich an der Entwicklung des \md oder der \mdg beteiligen kann. Auch den Aufbau des Codes beschreibe ich und beschreibe, welche Funktionalität des \md wo implementiert ist.
\end{abstract}

\bigskip

\selectlanguage{english}
\begin{abstract}
  In this seminar paper I present the \md and the \mdg. The \md is a console based debugger for the \mic, described by \name{Tanenbaum} in \cite{Tanenbaum1998}. The \mdg is a \gls{gui} for the \md.

  For the \md and the \mdg I describe how to use them. Also there is a tutorial where I explain, how you can find bugs in an assembler code with the \md. Furthermore I show the differences between the \md and the \mdg.

  In the second part I concentrate on the development of the \md and the \mdg. I describe the tools I've used for development and explain how you can use them to participate in development of the \md and the \mdg. Also I describe the structure of the code and where which functionality of the \md has been implemented.
\end{abstract}

\selectlanguage{ngerman}
